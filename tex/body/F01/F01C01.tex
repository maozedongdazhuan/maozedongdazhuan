\documentclass[../../dazhuan.tex]{subfiles}

\begin{document}
\clearpage
%\pdfbookmark{第一版自序}{v1pref}
\chapter*{第一版自序}
\addcontentsline{toc}{part}{第一版自序}


亲爱的读者朋友,呈现在您面前的这一部《毛泽东大传》(共4册),分为10卷373章,总计300余万字。这部书既不是小说,也不是演义,更不是传奇,而是一部真实地反映毛泽东一生如何修身,如何做人,如何处事,如何工作,如何奋斗的历史纪实文学。它既可以当作通俗的故事书读,也可以作为历史工作者在研究中国共产党史时最真实可靠的资料运用。\emph{更确切地说,它是一部人生教科书。}如果您真正读懂了这部书,或者是您的后人读懂了这部书,那么对您本人,对您的家庭,对您的后代,将会产生不可想象的重要影响和作用。当然,功利主义者除外,因为功利主义者考虑的只是一己之私。作为一个从自身、从家庭、从民族、从世界、从历史的角度设计自己人生的人,都应该好好地读一读这部书。

笔者在这里首先要感谢毛泽东及其同时代的所有的前辈们,感谢他们在极其艰难困苦的条件下的那种丰富多彩的社会实践活动;感谢口述、记录那个时代的所有前辈和专家、学者;还要感谢笔者在写作过程中所参考的数十部有关专著的那些作者和编著者。没有他们的艰辛创造,没有他们所提供的翔实的历史资料,笔者是不敢也不可能在没有任何依据的情况下,随心所欲地杜撰出反映毛泽东生平的这部故事书的。

说起对毛泽东的认识,笔者也像不少人一样,有一个从基于朴素情感的表象认识到理性认识的转变过程。应该说,\emph{吾生有幸,生逢其时,年青时曾经亲身体验到了那个千载难逢的毛泽东时代的风云与沧桑。}但是,\emph{从历史上看,当代人不了解当代的人和事,这是一种普遍的现象。只有极少数人例外。}所以,几十年以来,笔者时时想了解一个真实的毛泽东。\emph{令人惊奇的是,从上世纪80年代末到90年代初,有关毛泽东生平的专著和历史资料竟然像雨后春笋一般,破土问世,而且大有不可遏止之势,直到2007年,温执政国务院决定一出,这才戛然而止。}笔者从那时起,便常常购买或借阅有关毛泽东的资料,细细研读,兴至浓处,不舍昼夜,时而击节,时而拍案,时而开心一笑,时而掩卷而泣。阅读既多,感触愈深。笔者正是通过这种不间断地阅读和深入研究,才终于完成了对毛泽东由阶级情感的表象认识上升到理性认识这一转变过程的。

毛泽东在1962年第二次庐山会议期间,曾对他身边的工作人员张仙朋说过,他有一个愿望,打算在他有生之年写一本关于他个人的书。他说:“把我的一生写进去,把我的缺点错误统统写进去,让全世界人民去评论我究竟是好人还是坏人。我这个人啊,好处占70\%,坏处占30\%,就很满足了。我不隐瞒自己的观点,我就是这样一个人,我不是圣人。”很可惜,毛泽东的这个愿望最终没能实现。这无疑是人类社会的一个重大损失,同时也是共产主义运动史上的一个重大损失。正是由于这个历史的缺憾,笔者才大量收集史料,研究史料;尽管迄今还有很多历史资料没有公诸于世,而且还有一部分毛泽东暮年的十分珍贵的历史资料已经被某些心怀叵测的人销毁了,但就笔者已经见到的历史资料,写出一部最系统、最真实和比较全面而完整的《毛泽东大传》的条件已经基本成熟,而且成为可能,所以就不揣冒昧,开始酝酿、构思并动手写作了。 
 
笔者认为,中国出了个毛泽东,应该是中国人的骄傲。无论是毛泽东的人品、学品、文品,还是他的武品、诗品、书品,就个性而言,在他的同时代中,无论是哪一方面的人,即正面的或者是反面的,没有一个是可以和他老人家相提并论的。毛泽东的思想、品行、性格、才学和智慧,堪为世人之楷模。也可以说,翻遍5千年中华文明史和数千年世界史,像毛泽东这样的伟人,的确是前无古人,后无来者。\emph{作为中华赤子,他心里始终装着劳动人民,装着改造中国、改造世界、改造人类的伟大理想。}作为军事统帅,他指挥人民武装力量从无到有,从小到大,从弱到强,无论多么强大的敌人和反动的军事力量,在他面前无不灰飞烟灭。别说是蒋介石那几百万大军,就连日本军队、法国军队、以美国为首的联合国军,也都不得不在他面前低头认输。\emph{作为人民领袖,他始终引领着中国人民朝着社会主义、共产主义的方向大踏步地前进,在建立了强大的社会主义新中国之后,成功地改变了二十世纪的世界政治格局。}作为革命导师,他一生都在读书、备课、讲学,教育他的战友、战士和人民,同时也包括他的朋友和敌人。\emph{作为政治家,《共产党宣言》他读了上百遍,晚年的一切政治活动,无一不是在无产阶级专政的条件下,践行马克思、恩格斯提出的两个“最彻底的决裂”(“共产主义革命就是同传统的所有制关系实行最彻底的决裂,毫不奇怪,它在自己的发展进程中要同传统的观念实行最彻底的决裂”)。}作为学者,《资治通鉴》和《二十四史》等等通史和断代史,他比任何一个历史学家读得都多,而且议论精辟,见解独到,冠绝古今,堪称后世“古为今用”之楷模。\emph{作为哲学家,他提出了“物质是无限可分的”著名论断,因此,世界科技大会通过了一项决议,将粒子命名为“毛粒子”。}巴基斯坦前总理布托说:“毫无疑问,毛泽东是巨人中的巨人。他使历史显得渺小。他强有力的影响在全世界亿万男女的心中留下了印记。毛泽东是革命的儿子,是革命的精髓,是革命的旋律和传奇,是震动世界的出色的新秩序的最高缔造者。毛泽东没有死,他永垂不朽。他的思想将继续指导各国人民和各民族的命运,一直到太阳永远不再升起。若果仅仅是从中国的范围来衡量他的划时代的功绩,那将有损于对这位非凡人物的纪念。当然,他为中国及其8亿人民做了了不起的事情,但是毛泽东更是一位崇高的世界领袖。他对当代局势的发展的贡献是没有人可以比拟的。”英国著名女作家韩素音也说:“毛并不是一个不可思议的人物。他是一个完美的人,是全民族的代表,是人民和时代的化身。革命造就了毛,毛也造就了革命。毛的一生不仅是他个人的一生,而且也是中国整整一个历史时期的象征。”由此可以断言:\emph{不了解毛泽东全部的真实历史,不了解毛泽东整个时代的真实性,不了解毛泽东时代所有其他那些风云人物,无论任何人都是不可能真正理解占了大半个二十世纪的“毛泽东现象”,也不可能真正理解毛泽东作为一个自然人的真正的历史含义。}

最后还需要说明的是,笔者在写作过程中,采用的是通俗的章回体的说话(或者叫评话,也叫平话,近代叫说书)形式和《资治通鉴》的编年体例以及《史记》中史论分离的手法。\emph{编撰的原则是尊重历史,像考古学家一样,尽可能地复原历史。}当然,历史绝不是、也不可能是一笔某某人吃喝拉撒睡的流水账。事实上,\emph{任何一部历史都是对于它所囊括的那个时代一些重要人物以及人物之间所发生的事件的生动、典型而又鲜活的记录。}所以\emph{笔者遵循的原则是,对毛泽东和他那个时代所有的历史人物不为尊者讳,不遗尊者贤,不没尊者功,不隐劣者迹,}秉笔直书,力求写出一部真实可信、通俗易读、雅俗共赏、老少咸宜的毛泽东的生平故事,\emph{将毛泽东时代的真相明明白白地昭示于后世。}为此,笔者编撰的方法是:1、以毛泽东为中心,收集所有有关毛泽东的史料和与毛泽东有关的人物、事件的史料。\emph{对所有能见到的史料采取去伪存真、辨正纠错、摈弃议论(特别重要的评论除外)的办法而采用之。2、去芜取精,去概念存形象,使故事本身更生动、更丰满、更感人,更具生活情趣。}

尽管笔者力求这部故事书臻于完美,但由于本人无缘接触到那些事件的亲历者,也无缘接触到历史的第一手资料,所以只能凭借所能见到的有关毛泽东的历史专著和资料进行梳理和重新撰写;更重要的原因还在于笔者学识浅薄,水平有限,所以,在本传中出现一些错误和疏漏,是在所难免的。因此,笔者真诚希望,亲历那个时代的所有健在的前辈们,在“毛学”研究方面卓有成就的专家学者们,以及收藏有笔者尚未见到的有关资料的“毛学”爱好者们,多多赐教和帮助,以便笔者有机会纠正错误和充实内容,使本传能够于再版之时,愈加完美。笔者将不胜感激之至。

\vspace{\baselineskip}

\hspace{18em} 东方直心

\hspace{18em} 2010年5月
    
\end{document}