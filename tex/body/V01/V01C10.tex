\documentclass[../../dazhuan.tex]{subfiles}
% 第一卷
\begin{document}
\chapter*{第十章}
\pdfbookmark{第十章}{V01C10}
\begin{pref}
	现在国民思想狭隘,安得国人有大哲学革命家、大伦理革命家,如	
俄之托尔斯泰其人,以洗涤国民之旧思想,开发新思想。
\end{pref}

话说毛泽东为了总结办学和教学经验,专门设置了一个夜学日志,并带头天天写日志。从他在开学前后所记下的11月5日至11月14日的日志中,可以看出他为办好夜学所进行的精心组织以及对夜学各种大小事务的周密思考:

11月5日晚,召集国民高小和一师部分职员及三四年级同学共十余人开会,就夜学问题“商议实行办法”。

11月6日,出版布告通知学生:“定阴历九月二十五日(11月9日)晚上7点钟,在师范下首国民学校内头次上课,各带笔墨纸砚,齐到为要。”

11月7日,“晚上开会磋商进行办法。学生报名又增二十余人。”由于名额有限,毛泽东只好让周世钊等人截止了报名工作。

11月8日,“晚上教员12人开教务会议”。决定了课程设置、老师分配、授课时间、授课内容等。毛泽东自己担任历史课。

夜学学生按程度高低,分为甲班、乙班,开设国文、算术、常识3科,每班每周3夜,每夜2时,共6时,分配国文3时,算术1时半,常识1时半。

教员每班国文2人,算术1人,常识3人。

教授内容:“分认字、写纸、短文、便条、信札等项。”算术,“先期尽教珠算,入后稍加笔算。”“常识为与以普通之知识及精神之安慰,如历史,教以历代之大势及近年关系最巨之事迹,所以精养其历史的观念及爱国心。”

“常识注重内容,讲义宜少”,“常识注重精神,与国文、算术之近于技能者相别,故作为余兴,每晚以半点钟,用演讲之形式教授之。”常识教授历史、地理、理科、修身、卫生、实业、政法、经济、教育等,“内容多而时间少,宜择其重要及与夜学学生有密切关系者。”教授方法,“大部取注入式,间采启发式。”

11月9日,工人夜学在第一师范附近的国民学校教室里正式开学。

晚六时半,夜学教职员陆续来到国民学校。学生们也纷纷到校报到。“此次报到,既极踊跃,而秩序又甚为整齐,出乎意料之外。此次学生中,十三四岁小儿竟占十分之三数,有在初小读过1年及2年者。夫儿童失学如此其多,使无此夜学稍从补救,将永远废学矣。其中年失学者,前日截止报名后,仍有无数要求补报者。以碍于名数不能许,而彼等固如嗷嗷之待哺也。”

毛泽东和教职员们把一批批穷苦的学生领进了教室,教室里一片欢快的景象。为了教学方便,夜校首先对大部分报到的学生进行文化水平摸底测验。测验的结果,“有清楚不误全行写出者,有略写街名二三个及姓名住所者,有姓名不能写好者。”

毛泽东和周世钊等根据测验成绩,评定甲、乙,把学生分为两个班。甲班44人,乙班41人。

接下来举行开学典礼。学生们“整队向国旗、孔圣行三鞠礼,职教员和学生相向互行一鞠躬礼”。

师范学监主任、代理学友会会长方维夏先生致训词,勉励学生好好学习;夜学主任周渭航也致了训词。

毛泽东兴致勃勃地将夜学上课说明书逐条作了解释。他说:

“讲义及抄本由学校发给。每次上课,衣服听便,不必求好。每次上课须带笔墨。本校已函请警察保护,来往尽管放心。”

11月10日,“购来参考书12种”。晚上召集夜学管理12人在学友会事务室开会,就夜学管理的一些事项作了研究,决定:

“一、教室设灯4盏,头门设灯1盏,三门设灯1盏。二、本校派定工役1名,司送茶水、灯油等事。国民部派定工役1名,应随时呼唤。三、教员与管理员同赴夜学上课,以免学生久待。四、管理取严格主义,以坚持学生信仰。五、学生雨具,令拂干,置于讲堂椅下,自为照料。六、学生大小解,天晴另就街厕所……”

11月11日,“晚上7时补试。计前夕报名未到者44人,今晚到10人。”

11月12日,甲班上课,“课毕,由毛泽东报告:不可喧扰;宜每次上课,3次不到者,开缺不补;解手赴外边厕所;雨时雨具,自置椅下看管;今次有未带笔墨者,下次宜都带来;习字纸带归书好,下次带来,评定甲乙记分等事。”

毛泽东在这一天的日志里还写有这样的内容:“教授两点余钟,学生听之颇能娓娓不倦。”“教室洋油灯4盏,有2盏不明,中间颇暗,应添1盏。”“此次地理讲义不善,字嫌小,又不明白。”

11月13日,乙班上课。“乙班学生程度颇低,国文多不深懂。”

毛泽东还在日志中写道:他告诉学生,前一天晚上,甲班学生放学后“又在铁路旁喧呼者,警察以此为言”;要求大家注意:“省城特别戒严,步哨及社坛岭”,“报告学生提早上课;一面由学校行文省署,邀求保护。”

11月14日,“甲班上课。算术罗宗翰出席,教以数之种类加法大略及阿拉伯数字码。历史常识毛泽东出席,教历朝大势及上古事迹。学生有4人未带算盘,从小学暂借。为戒严早半时下课。管理者李端纶、萧珍元。实验3日矣,觉国文似太多太深。太多宜减其分量;太深宜改用通俗语(介乎白话与文言之间)。常识分量亦嫌太多(指文字),宜少用文字,其讲义宜用白话,简单几句标明。初不发给,单用精神演讲;终取讲义略读一遍足矣。本日历史即改用此法,觉活泼得多。本日算术却觉过浅,学生学过归除者,令其举手,有十几人之多。此则宜逐步加深。”

“说理宜深,语言、文字出之以浅。”

11月16日,张超给甲班上物理常识课,他讲得太深,许多学生坐不住。课后,毛泽东对学生们说:

“物理一科,极有趣味,方才所讲,不过发端,将来电灯之所以能明,轮船、火车之所以能速,其理必皆告汝等知之。”

1917年11月中,毛泽东以一师学友会总务的身份,成功地组织学生志愿军进行了一次英勇的护校斗争。

此时正值护法战争开始不久,桂军谭浩明赶走了驻守长沙的傅良佐部,他的军队尚未进驻长沙城,城里只有一些维持日常秩序的警察。就在这时候,北洋军第8师王汝贤的部队由湘潭、株洲向长沙溃退,溃兵们三五成群,四处奸淫掳掠。长沙城四周顿时大乱。

一天午饭后,第一师范校园里有人传言,说是北洋溃军的一部分已经到了离一师南端只有两里多远的猴子石一带,因为他们不了解长沙城里的虚实,不敢贸然继续前行,就停留在那一带,到附近农民家里抢饭吃。这个消息一传开,校园里一片惊慌,人们纷纷收拾细软,准备带领家人躲到城东5里的阿弥岭去。

毛泽东见师生们如此慌乱,心里暗暗着急,但他毕竟有着半年的军事生涯,他向校领导冷静分析了溃兵的情况,认为溃兵既然不了解城里面的情况,且已疲惫不堪,只要把学校里一年前训练过的学生军组织起来,再请一些警察来前来配合一下,是完全可以防范溃军进扰校园的。

学校当局采纳了毛泽东的意见,决定由他负责指挥,组成学生志愿军,“分夜梭巡,警卫非常”。毛泽东立即指挥同学们把教室里的桌椅统统搬出来,垒做障碍物,构成“防御工事”,进入作战状态。

此时已有零星溃兵来到校园大门口窥探,他们见一师的学生志愿军戒备森严,也就不敢轻易造次了。

毛泽东了解到这一情况,认为可以采取“以弱示强”的办法,主动出战,打溃兵一个措手不及。于是他迅速将胆子较大的学生志愿军200余人集结起来,每人配备一支平时操练用的木枪,带上爆竹,分成3队,潜伏在校后的妙高峰上,形成居高临下分进合围之势。他又派人到附近警察分所联络,请求警所派一部分警察前来助阵,并且让他们埋伏在学生志愿军的前列。

傍晚时分,溃兵们小心翼翼地沿着铁路线向北转移,渐渐出现在学生志愿军潜伏阵地不远的地方。毛泽东一声令下,警察们纷纷鸣枪,学生志愿军也同时燃放起装在煤油桶内的爆竹,齐声高呼:

“缴枪没事!缴枪没事!”

“傅良佐逃走了,桂军已经进城,你们赶快缴枪!”

溃兵们突然听到混合着鞭炮声的枪声和数百人的呐喊声,还以为真的是碰上了敌人的大部队,早已魂飞魄散,不敢抵抗了,纷纷缴械投降。

毛泽东让荷枪实弹的警察将溃兵们押到学校的操坪上露宿,又组织全校同学将缴获的枪支和其它武器抬到学校里面。

第二天,由学校当局出面协商,请长沙商会筹出一部分款项,将溃兵们遣散了。

毛泽东超人的谋略和沉着冷静地指挥,成功地阻止了溃兵们进入校园行劫。师生们纷纷称赞他“一身都是胆”,无愧于“毛奇”的称号。自此以后,“毛奇”这个称号便成为一个双关语,一是赞扬毛泽东是一个“志向非凡、与众不同”的奇特的人,一是把他比作德国历史上一位很有学问又很会打仗的将军——毛奇。

徐特立也为有毛泽东这样的学生而感到自豪,他十分钦佩毛泽东“通身是胆”,说毛泽东是一个“不向恶势力屈服的人”。

后来毛泽东在一次谈话中笑着说:“我搞军事,恐怕这才真是第一次呢。”

这正是:\begin{xemph}提兵布阵试牛耳,以弱示强抓战机。


\hspace{4em}溃兵惊心魂魄散,乱世学子称毛奇。\end{xemph}


且说1917年12月24日,湖南一师放了寒假,毛泽东主持学友会创办的工人及其子弟夜校也将随之结束。他和方维夏等人召集教员们进行了工作总结,并“请餐两大席”,以答谢各位教员的义务教育之功。他们还商定,对那些在夜校里“潜心听讲,缺席甚少学生,分3等发给奖品,以示鼓励。”

毛泽东一处理完夜校的事情,便踏上了寒假出游之路。他步行来到浏阳文家市,在铁炉冲陈绍休同学家住了几天。这几天,白天他和农民一起挑水、种菜,晚上就和农民们坐在一起拉家常。他看到这个地方的农民没有栽种果树的习惯,就耐心劝他们种植果树,造福子孙。他说:

“前人栽树,后人乘凉;前人栽树,后人食果。”

毛泽东还亲自动手栽了几棵板栗树。之后,他又到西乡土桥炭坡大屋老同学陈昌的家乡一带走访了一遭。这次出游时间并不太长,不久,他便回到了长沙。

他的一个本家姑姑毛贵英和姑爹彭华,在长沙三泰街开了一家“彭福泰棉絮店”,那里是他在一师读书期间经常去的地方。姑妈对这位本家侄儿很是喜爱,只当是娘家的亲侄子一样,每次见到润之来了,总要给他做些好吃的。

他们老夫妻没有亲生儿女,膝下唯一的过继女儿是毛泽东的远房妹妹毛福清。毛泽东把这个刚满4岁的毛福清看作是自己的小妹妹一样,只要他一来,便教小福清识字、写字、唱歌。小福清也总是亲切地叫他三哥。

这一天,毛泽东来到姑妈家,看到毛福清正在缠脚,一双小脚丫子脚被裹得紧紧的,像包好的粽子一样,就心疼地问:

“妹妹,疼吗?”

毛福清的脚像火烧一样疼痛难忍,经三哥一问,她的眼泪便扑簌簌地掉了下来。毛泽东说:

“那就莫包嘛!”

毛福清十分为难地摇摇头。毛泽东知道她不敢对姑妈姑爹讲,就直接去找姑妈姑爹为她讲情。他说:

“缠脚是封建社会对妇女的一种束缚,生成的一双脚为什么要缠呢?一个好好的人为什么硬要她变成一个残疾人呢?这个习俗是很残忍的,很不公平的。妹妹的脚疼得那么厉害,还是让她放了吧。”

毛泽东能言善辩,直说得姑妈姑爹点头称是,从此便不再让毛福清缠脚了。

此后,毛泽东趁着在姑妈家小住的机会,又了解了其他一些市民的生活状况。

新的一学期开始了,一师又聘来了一位新老师,名叫孙俍工。孙俍工毕业于北京高师国文部,他在一师讲授的是语文学、文字学、中国文学概论、古文选读。这位先生不仅在授课时旁征博引、妙语连珠,而且他的书法飘逸洒脱、笔力遒劲。毛泽东对这位比自己还小一个多月的先生非常敬重。

有一天,毛泽东去孙俍工寓所讨教书法,他先是欣赏了孙俍工收藏的名人字画,又看了孙俍工临摹的王羲之的《兰亭集序》。他说,他很喜欢书法,但觉得学习书法行书比较容易入门,自己想先学学草书。孙俍工告诉他说:

“其实,行书比楷书隶书都难。在行书中完成那么多的笔锋的变化,不容易呀。要做到行笔而不断,着纸而不刻,轻转重按,如行云流水,无少间断,方能永存乎生意。”

毛泽东听了,觉得先生所言很有道理,就频频点头。他说:

“这就是了。这行书变化如此之多,但不知有无章法可循?”

“有啊!”

孙俍工说着,取笔在手,在笺纸上写下了学习行书的要诀:疏密、大小、长短、粗细、浓淡、干湿、远近、虚实、顾盼、错落、肥瘦、首尾、偃仰、起伏。

毛泽东恭恭敬敬地收起“28字诀”,起身告辞,他说:

“与师一席话,胜读十年书。孙先生,感谢您了。”

孙俍工起身相送,拉着毛泽东的手真诚地说:

“依我看,你现在的字是才气有余,功力不足咧!润之,还是要从练楷书开始。楷如立,行如走,草如奔。你站不稳,又怎么能走和跑呢?”

自此以后,毛泽东和孙俍工先生切磋书法技艺的次数越来越多了,他们竟成了无话不谈的好朋友。

1918年2月19日,毛泽东在一师学友会职员会议上提议继续办夜学,与会者们一致表示赞成。

3月2日,毛泽东草拟了一份《第一师范附设夜学招学广告》,广告中提出,新学生愿入夜学者,不论年纪大小,认字多少,均可报名,听课不收学费且发给讲义。

3月3日,毛泽东主持召开夜学职教员会议,确定本期夜学职教员组织。他本人兼任夜学“管理”。

3月17日,新一期夜学开学了。学校附近工厂里的许多工友都认识毛泽东,亲切地称呼他“毛先生”。

毛泽东通过创办工人夜学,与城市市民有了广泛接触,同他们建立起了深厚的阶级感情,同时也取得了与产业工人打交道的初步经验。

3月间,一师邀请上海《教育》杂志主编李石岑来校进行了一次讲演。这位李石岑先生既是一名学者,又是一位游泳专家。毛泽东了解到这一情况后,就和同学们一起请李石岑先生传授游泳技术。李石岑爽快地答应了。

此时天气还很凉,大家都还穿着棉衣。李石岑先生下水做示范,毛泽东等人也都跟着下了水,一直游了三四十分钟才上岸。

3月下旬的一天,毛泽东在岳麓山下周家台子蔡和森的家——“沩痴寄庐”,再次与同学们聚会,他向大家提出了创立一个“求友互助”正式“团体”的倡议。

本传前面已经说过,这“沩痴寄庐”长期以来就是毛泽东这一批青年人经常聚会的地方,他们在一块儿议论最多的就是关于人生方面的问题以及有关杨先生的教诲。杨昌济先生曾经多次教导他的学生们说:“人患无肯立志身,精神一抖,何事不成。”他又说,人贵在有志,而理想则是立志的基础,树立了根本的理想,才能“立志”。杨昌济先生还曾说过:“人者,有理想之动物也。人生的目的在于实现理想。”在杨昌济先生的教育下,经过两年多的酝酿,毛泽东等人逐渐形成了共同的理想,“如何使个人及全人类的生活向上”,已经成为他们议论的核心问题了。参与讨论这类问题的人大概有15人左右,讨论的次数也有上百次了,但他们从来还没有像现在这样感到迫切!

就在这天的聚会中,毛泽东提议说,“国内的新思想和新文学已经发起了,旧伦理和旧文学,在诸人眼中,已一扫而空,顿觉静的生活和孤独的生活之非。”既然自己的品行要改造,学问要进步,求友互助之心早已“热切到十分”,现在应该有“一个翻转而为动的生活与团体的生活之追求”,何不搞一个正式团体,为大家的互助提供一个联系的桥梁和纽带呢?

此议一出,立刻“就得到大家的赞同了”。众人纷纷议论说,先生总是告诫大家要有一种奋斗的向上的人生观。为不负先生的殷殷之望,组织一个小团体,不是可以更好地奋斗向上吗?就这样,他们终于形成了一个共识,要“集合同志,创造新环境,为共同的活动。”

于是,毛泽东和蔡和森、萧子升等人就开始着手新学会的组织工作。要成立团体,首先需要搞一个章程,众人一致推举毛泽东起草新学会简章。

此时(1917年秋)已在第一师范附属小学任教的萧子暲在日记中记载:

3月31日,“二兄萧子升来坐已久,交阅润之所草新学会简章。二兄意名为新民会云。又述润之等赴日本求学之计划。”

1918年4月8日,“接二兄手书,力主予出洋,付来润之所重草新学会简章。”

4月13日,“夜润之来。明日新民学会开成立会。”

4月14日,这是一个春光明媚的星期天。毛泽东和他的朋友们踏着轻松的脚步,兴致勃勃地来到岳麓山下周家台子(又叫刘家台子)蔡和森的家“沩痴寄庐”,新民学会成立会议将在这里召开。

出席这次会议的人员应到21人,实到14人,他们是:毛泽东、蔡和森、何叔衡、萧子升、陈绍休、萧子暲、邹彝鼎、张昆弟、陈书农、邹蕴真、周名弟、叶兆桢、罗章龙、李维汉。在这14人中,除罗章龙是第一联合中学的学生外,其余的都是第一师范的学友。

因各种原因没有到会的7个人是:陈章甫(即陈昌)、周世钊、罗学瓒、熊焜甫(光楚)、曾以鲁、傅昌钰、彭道良。此后,他们相继加入新民学会。

参加会议的何叔衡,因“自愧年老才退,不配与20岁左右的青年为伍,所以提出不入会。”后来在毛泽东的动员下,他在1918年8月间还是加入了新民学会。

关于新民学会成立之时的详细情况,萧子暲在日记中是这样记载的:

“成立会从上午11时开始,首先讨论会章。人们在屋子里,在河滩上,讨论学会的宗旨、名称、章程。”

“会章系彝鼎、润之起草,条文颇详。”毛泽东首先向大家作了说明和解释,以征求大家的意见。“和平常一样,他的话语浅近、扼要、深刻,意思新颖、明朗、透彻。”他在他起草的《新民学会会务报告》(第1号)中,在谈到新民学会创立缘起时是这样写的:

“现在国民思想狭隘,安得国人有大哲学革命家、大伦理革命家,如俄之托尔斯泰其人,以洗涤国民之旧思想,开发新思想。”“还有一个重要原因,即大家都是杨昌济先生的学生,与闻先生的宏论,做成一种奋斗的和向上的人生观,新民学会乃从此产生了。”

毛泽东在会章草案中写入了他主张大同之世,主张大抵抗、大斗争等内容。萧子升一直不赞成巨大的激烈的变革,他主张点滴的温和的改良。所以,“子升不赞成将现在不见诸行事的条文加入,颇加删削;讨论结果,多数赞成子升。”在这种情况下,毛泽东也只好服从大多数人的意见了。

这萧子升聪明好学,成绩一向很好,能写诗、赋文章,又能写一手好字,其品德也一直为湖南一师的师生们所赞许;他和毛泽东有共同的志向,治学态度和品行修养也多所相似或相近,同为杨昌济先生的得意门生。杨昌济先生曾经把他列在长沙几千名学生中的第一位,蔡和森在第二位,毛泽东则在第三位。因此,萧子升在同学中的影响是比较大的。

中午,蔡畅和大姐蔡庆熙帮助妈妈葛健豪做好了简单的午餐,请大家会餐。

饭后,会议通过的会章将这个团体定名为新民学会,据萧子暲解释说,这是“取自《礼记·大学》“大学之道,在明明德,在新民”,“日日新,又日新”的字样,有一种反对旧制度,主张革新,为人民的意思。” 

会议表决通过的会章共有11条,其条文摘要如次:

“本会以革新学术,砥砺品行,改良人心风俗为宗旨。”

“会员对于本会每年负一次以上的通函之义务,报告自己及所在地状况与研究心得,以资互利;凡经本会会员5人以上之介绍及过半数之承认者,得为本会会员;会员有不正行为,及故违本简章者,经多数会员之决议,令其出会;会员每人于入会时纳会费银1元,每年纳常年费1元;本会设总干事1人,综理会务,干事若干人,协理总干事分理会务,任期3年,由会员投票选充之;本会每年于秋季开常年会一次,遇必要时,得召集临时会;本简章不适应时,经多数会员决议,得修改之。”

会员守则中还规定:“1、不虚伪;2、不懒惰;3、不浪费;4、不赌博;5、不狎妓。”其中不懒惰这一条,是由萧子暲提出的。

萧子暲还回忆说:“大家推毛泽东同志为总干事。他本是学会的发起人,组织者,但他谦虚地只同意做副干事。”因此,会议又推举萧子升为总干事,毛泽东、陈书农为干事。陈章甫任演讲部主任。

这天下午5时,新民学会成立会议结束。“斯时,天气晴朗,微风掀拂江间的碧波和江岸的碧草,送给到会诸人的脑里一种经久不磨的印象。”

后来关于新民学会的性质问题,有的会员因为会章中没有提出远大的革命目标,所以就把它定性为,只“不过是少数读书人的一种读书团体”。

这正是:\begin{xemph}首倡新民会,聚义大斗争。

\hspace{4em}胸中藏万壑,存异求志同。\end{xemph}



在此后的某一天,毛泽东来到一位家庭比较富裕的“新民学会”会员家里,受到主人的热情款待。正当他们谈论到一个政治话题时,这位年轻的主人突然叫来仆人,吩咐仆人去买猪肉,并交代他说,要什么什么价格的和什么什么样的猪肉。毛泽东见此情景,极为恼火,他认为这位朋友以家庭琐事干扰谈论家国前途大事,违反了同学之间约定的三不谈,即不谈金钱、不谈男女之事、不谈家务琐事,一气之下便起身离去,从此再也不和这个富家子弟单独来往了。

还是在这个4月间,湖南省的局势又发生了突变,张敬尧的部队突然开进了长沙城。毛泽东为了学校安全,便和一部分同学组织了一支由他任队长的警备队,在校园内外日夜巡逻,严防散兵游勇来校滋扰,从而保证了一师的正常教学秩序。

张敬尧者,何许人也?原来这张敬尧是属于皖系军阀里的一个不小的头目。早在1917年就开始的直皖联军与湘桂联军之间的混战,到此时已分出了胜负,驻守湖南的湘桂联军司令谭浩明被皖系赶走了,张敬尧便于4月间率部进入长沙,坐上了湖南督军兼湖南省省长的宝座。自此以后,张敬尧和他的弟弟张敬舜、张敬禹、张敬汤哥儿4个无恶不作,罪行累累,给湖南人民造成了严重的灾难。详情容后再叙。

且说此时在新民学会的活动中,会员们聚在一起议论最多的一个问题,是如何出省、出国的问题,因为大多数会员已经毕业或即将毕业,选择一个什么样的职业才能更好地施展抱负?他们不满意湖南这个闭塞的地方,认为湖南交通不便,政治文化又比较落后,所以不应该“堆积”在湖南一地,说是应当分散到中国乃至世界各地去考察,去开辟一个方面,去打开各方面的阵地。不少会员还认为应当去日本留学,因为“日本是辛亥革命的策源地,孙中山先生组织兴中会、同盟会和武昌起义,都是受到日本的影响。其次,日本是东方和西方科学文化的桥梁地带,维新早,接受西方的科学技术早。”

不久,罗章龙即决定去日本留学。陈昌也曾考虑去日本留学,毛泽东则认为他留在国内更能发挥作用,应该以学校为阵地,培养人才。陈昌欣然接受了毛泽东的建议。

罗章龙虽然愿意去日本留学,可他家里经济比较困难,还难以成行。毛泽东和新民学会会员们及尚未加入新民学会的何叔衡,得知这一情况,便纷纷伸出援手资助他。

罗章龙就要走了,毛泽东、何叔衡与新民学会会员们来到长沙北门外的平浪宫,一起聚餐,为罗章龙饯行。尔后,毛泽东一直把罗章龙送至码头上,临别时又交给他一个信封,说是内有一首诗相赠。罗章龙辞了毛泽东,上了船,拆开信封一看,是一首用“二十八画生”笔名写的七古,题目为《送纵宇一郎东行》。诗云:

\begin{couplet}
云开衡岳积阴止,天马凤凰春树里。

年少峥嵘屈贾才,山川奇气曾钟此。

君行吾为发浩歌,鲲鹏击浪从兹始。

洞庭湘水涨连天,艟艨巨舰直东指。

无端散出一天愁,幸被东风吹万里。

丈夫何事足萦怀,要将宇宙看稊米。

沧海横流安足虑,世事纷纭从君理。

管却自家身与心,胸中日月常新美。

名世于今五百年,诸公碌碌皆余子。

平浪宫前友谊多,崇明对马衣带水。

东瀛濯剑有书还,我返自崖君去矣。\end{couplet}

诗的后面另附有4句赠言:“若金发砺,若陶在钧,进德修业,光辉日新。”

罗章龙看罢,自是感激非常。后来罗章龙一行抵达上海之时,恰逢5月7日。这一天正是1915年日本政府向袁世凯政府提出二十一条最后通牒的日子,日本反动军警再一次借机生事。来到上海准备赴日的众学子“忽闻东京发生日警迫害中国侨民风潮”,殴打中国爱国学生,驱赶他们回国。罗章龙一行只好决定暂停东渡,返回湖南。

1918年5月10日,即将毕业的毛泽东将一师学友会的“一切会金、器物、图书及簿据等,移交审计喻恒、皮文光二君,代理保存”。他还对未来的会务工作提出了两条建议,一是扩充会务经费,用以购置图书杂志、办夜学、救济失业穷民,为筹备学友会独立会所,联络毕业同学,以谋全省教育之研究及发展;二是学友会加设交际部,“谋内外之连络,通新故之情愫,图理论与经验之结合”。毛泽东说罢,向喻恒、皮文光二人提供了自己草拟的交际部细则7条。最后,他又诚恳地对喻恒、皮文光说:

“今愚等将去矣,惟有望后来同学诸兄,竭力以图其成而已。”

就在毛泽东移交后不久,即在这5月间里,湖南第一师范就被张敬尧的弟弟张敬汤所部的一个旅占领了,学校只好被迫停课,学生们大部分都离开了学校。

毛泽东趁此机会又一次出游了,他同蔡和森一起,从周家台子“沩痴寄庐”出发,徒步沿洞庭湖南岸和东岸,经湘阴、岳阳、平江、浏阳等县,游历半个多月,进一步了解了农村的政治、经济等状况。

蔡和森回到长沙后曾对家人说:

“这次游学虽然身无分文,但润之会写字,替人写横幅对联,人家就给点酬金。我俩在‘见人说话,遇事帮忙’的8个字之下,得益不少。我想,只要乐于助人,走遍天下就不难了。”

1918年6月,湖南一师因皖系军阀张敬汤部依然强占着校舍,只得提前放了假。就这样,毛泽东也不得不提前从一师毕业了,从而结束了他“6年孔夫子,7年洋学堂”的整个学生时代。

毛泽东在1936年同斯诺谈话时曾经总结说:

“我在湖南省立第一师范度过的生活中,发生了很多事情,我的政治思想在这个时期开始形成。我也是在这里获得社会行动的初步经验的。”

欲知毛泽东毕业后的去向和活动,请看本传第二卷——《倚天抽剑》。

东方翁曰:毛泽东在新民学会会员“多数赞成子升”的情况下,以少数服从多数,忍痛对会章内容“颇加删削”,删去了他所主张的大同之世、大抵抗、大斗争等等重要内容。这是毛泽东在他一生中第一次体现其个性中的猴气。应该说这是一次理智的服从。将大目标藏之于胸,待时而发,乃是成大事者必须具有的胸襟。正所谓“小不忍则乱大谋”者是也。 
\end{document}
