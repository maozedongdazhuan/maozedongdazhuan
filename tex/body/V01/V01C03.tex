\documentclass[../../dazhuan.tex]{subfiles}
% 第一卷
\begin{document}
\chapter*{第三章}
\pdfbookmark{第三章}{V01C03}

\begin{pref}
中国也要有这样的人物。我们应该讲求富国强兵之道,才不致重蹈安南、朝鲜、印度的覆辙。你知道,中国有句古话:‘前车之覆,后车之鉴。’我们每个国民都应该努力。
\end{pref}

话说正当毛泽东在毛槐林的宽裕枯粮行当学徒的时候,表兄文运昌在湘乡东山高等小学堂放了假,来韶山看望姑父姑母。他同毛泽东谈起在学校的情况,告诉表弟说,东山高等小学堂讲授的是新学,在那里可以学到许多新鲜的东西,这在韶山是学不到的。毛泽东听了很动心,当即表示要跟表哥去东山读书。文运昌深表赞同,愿意做他的担保人。

毛泽东很有礼貌地向父亲说明了自己的想法,可是毛顺生听后只是哑然失笑。父亲的态度严重地伤害了他,在这之后,他和父亲的隔阂更深了,好长一段时间父子俩谁也不理睬谁。

毛泽东不再去宽裕枯粮行当学徒了,在外婆家一个亲戚的帮助下,他到湘潭县城一位失业的法科学生家里自学了半年。尽管迫于父亲的压力不得不重新回到韶山,但这半年的读书生活以及在湘潭的见闻,已经使他变得不再是父亲所期待的宝贝儿子了。

毛泽东决心要到湘乡东山去就学,他制定了一个稳妥可行的计划,首先从自家的朋友们和外婆家的亲戚们那里借钱,东借5块钱,西借10块钱,筹措好一切必要的经费;接着又同母亲商量好,请在家开馆授徒的八舅父文玉钦,还有比自己大9岁的表哥文运昌,在长沙读书的二姨家二表哥王季范,以及同族的长辈和老师毛麓钟、毛宇居、李漱清,一起来到家里当说客。

这一天,毛泽东起得特别早,请厨师办了一席饭菜招待客人。他请来的这一批说客都是读书人,对封建社会“万般皆下品、唯有读书高”、“学而优则仕”那一套,个个是滚瓜烂熟。客人们在席间你一言我一语,纷纷劝说毛顺生:

“咏芝天资聪颖,记忆力过人,读书必有大成。”

“家无读书子,官从何处来?”

“满朝朱紫贵,尽是读书人。”

“书中自有千钟粟,书中自有黄金屋,书中自有颜如玉……”

他们还告诉毛顺生说:

“如果让咏芝去上洋学堂,将来不仅前程远大,也可以赚大钱的。”

在亲友们的劝说下,毛顺生终于动了心,他同意让儿子到东山去求学。客人们见他答应了,都非常高兴,便起身告辞。

毛泽东把八舅父文玉钦、表兄文运昌、王季范等人送了一程又一程,对他们的帮助一再表示感谢。

眼看就要开学了,在入学前的一天晚上,毛泽东边吃饭边告诉父亲说:

“过几天我就要去东山高小读书了。”

毛顺生虽然答应让儿子去读书,但听说儿子为上学已经借了一笔钱,还是有些生气,便没好气地说:

“你到湘乡去读书,必须再弄到一笔钱来,好开支顶替你的长工的工钱。”

父亲的态度,与其说是他那贪婪的一面再一次暴露,倒不如说是他故意对儿子施加压力。毛泽东不便再与父亲发生冲突,只好硬着头皮问:

“雇一个长工一年需要多少钱?”

父亲说要12块。毛泽东二话不说,第二天便去找一位尊重学问曾经资助过族人上学的亲戚,从他那儿借了一些钱来。他把一个纸袋子放在父亲粗糙的大手上,说:

“这里是12块钱,我明天早上就去东山。”

这是1910年秋天的一个黎明时分,毛泽东早早起了床,开始收拾自己的行装。母亲文七妹看着正在忙活的儿子,只是问他要不要再多带一点东西。毛泽东看着母亲那抑郁的眼神,安慰她说,不用了。母亲见他收拾好了,又问:

“你要去跟你爹道个别吗?”

毛泽东倔强地说:

“不,我不去。”

这是一个凉爽的金秋时节,毛泽东把他那根早已用惯了的扁担挑在了肩上。但这一次他挑的不再是那两个粪筐了,而是一副非常简单的行囊,一头是一个书包,里面装着一件长袍、两条床单和一顶表兄文运昌夫人杨达昌送给他的青色素纱蚊帐,另一头则是装满了《水浒传》、《三国演义》这一类的课外书。就这样,毛泽东离开了非常闭塞的湘潭县韶山冲,和他在东山小学堂读书的表兄文运昌一起,去报考湘乡县东山高等小学堂。自此以后,他再也难得回到这里生活了。

且说毛泽东行至湘乡县境内,遇到了一个姓王的邻居,那位邻居看着眼前穿着新鞋新袜子的小伙子,感到很新鲜。在韶山,人们平日里可不是这副打扮。

“石三伢子,你穿上新鞋子真精神。”

“我要去上学了。”

毛泽东自豪地回答着。他向这位邻居说明了他那神圣的抱负。王姓邻居听后笑了起来,嘲笑他要去“洋学堂”读书的念头,还问他是否征得了父亲的同意。毛泽东火了,冲着他喊道:

“你简直是个老古董!你过时了!”

毛泽东挑着行李走了25公里,终于走进了东山高等小学堂的黑漆大门。这里那些砖瓦结构的建筑物,被护城河和高高的院墙围着,毛泽东感觉好像是走进了一座大寺庙。学校的规模之大,让他吃惊,长这么大还从未见过这么多孩子们聚在一起。

他在表哥文运昌陪同下报了名。在肃穆的考场上,毛泽东面对着《言志》这一试题,稍加思索,把个人的志向与祖国的命运连在一起,直抒胸臆,一篇文章一气呵成了。几个主考教师不约而同地把他的文章判为第一。然而也有人提出毛泽东不能到东山读书。毛泽东找到了校长办公室,很有礼貌地以乡下人憨直的方式问道:

“先生,你会让我在你的学校里读书吗?”

校长李元甫手里握着一根长长的镶铜竹烟袋,望着眼前这个瘦高个子神情庄重的青年,沉默了片刻,问他叫什么名字。

“先生,我叫毛泽东。”

他的镇定让校长扬了一下眉毛,但还是举出了几个不准毛泽东入学的理由:一是毛泽东年龄过大,二是毛泽东没有学过算术和地理,更重要的是东山是湘乡县立小学堂,这里是地主豪绅培养自己子弟的地方,学费和膳食费都有相当的津贴。毛泽东既不是本县人,又没有特殊关系,湘乡少数士绅出于地方观念,是不会同意录取他的。

国文教师谭咏春是极力推崇毛泽东文章的主考老师之一,他出于爱才之心,立即出面相助。他让儿子谭世瑛先安排毛泽东在学堂西学斋住下,自己则去找校长李元甫和其他教员,一面向大家推荐毛泽东的文才,夸毛泽东具有“建国之大用”,一面据理反驳说:

“中国学生可以到外国留学,湘潭人为什么不能到湘乡来读书?”

一位经学教员兼学校董事的张先生,仍然固执己见,不同意录取毛泽东。谭咏春愤然起身道:

“学校当以培养人才为重任,若张先生诸人故意作梗,我只好自请辞退!”

李元甫校长也指着毛泽东的文章大加赞赏,他说:

“我们学堂里取了一名建国才!”

在校长李元甫和谭咏春等教师的支持下,学校终于录取了毛泽东。毛泽东交了5个月的膳食费和学杂费,一共是1400个铜元。文运昌又帮表弟办好了入校注册手续,并做了他入学的担保人。就这样,毛泽东被分配到湘乡县立东山高等小学堂戊班,成了谭世瑛的同班同学。

东山高等小学堂的确是实行“新法教育”,这里除了教授经书外,还教授西方的自然科学和其它新学科,并且倡导改革。在早点名时,老师都要讲述中国在西方列强压迫下所受的苦难,以唤醒和培养学生的民族感情。毛泽东到这里以后才知道,慈禧太后和光绪皇帝两年前就死了。他感觉到,由偏僻的韶山来到东山需要弥补的差距实在太大了。

学校里有一名曾留学日本的萧老师教授音乐和英文,许多学生因为他戴着一根假辫子而讨厌他,背地里都叫他“假洋鬼子”。毛泽东却喜欢听萧老师讲日本的一些事情。萧老师所教的歌曲中有一首日本歌曲名叫《黄海之战》,其歌词大意是:

“麻雀歌唱,夜莺跳舞,春天的绿色田野多可爱,石榴花红,杨柳叶绿,展现一幅新画图。”

1936年毛泽东对斯诺说:“我还记得里面的一些迷人的歌词。”并当场背诵了这一段词。他还说:“我当时知道并感到日本的美,并且从这首歌颂日本战胜俄国的歌曲里感觉到她的骄傲和强大。我没有想到还有一个野蛮的日本——今天我们所认识的日本。”

那时候东山学校的学生大多数都是富家子弟,是一些衣着讲究的小绅士;而毛泽东却穿着一身农家裤褂,他的手也比那些富家同学的手粗糙一些,脸也因为太阳经常晒而比大部分同学要黑得多。他的身材瘦长,走起路来大步流星,按年龄段来说他的个头已经算是高的了,而站在那些比他小四五岁的同学中间,看起来就像小塔一般。他讲话慢条斯理,而周围那些伶俐孩子讲起话来,简直就像打机枪。他同样梳着辫子,可他的辫子显得特别长且有些蓬乱,一副不修边幅的样子倒显得有些潇洒。在体质和气质上已经具备了自己特征的他,加上满嘴的外县口音,因此颇受那些纨绔子弟们的嘲笑和冷遇。

毛泽东曾经对斯诺说:“我以前没有见过那么多孩子聚在一起,他们大多数是地主子弟,穿着讲究,很少农民供得起上这样的学堂。我的穿着比别人都寒酸。我只有一套像样的短衫裤……我平常总是穿一身破旧的衫裤,许多阔学生因此看不起我。”“人家不喜欢我也因为我不是湘乡人。在这个学堂,是不是湘乡本地人非常重要,而且还要看是湘乡哪一乡的。湘乡有上、中、下三里,而上、下两里,纯粹出于地域观念而殴斗不休,彼此势不两立。我在这场斗争中采取中立的态度,因为我不是本地人。结果,三派都看不起我,我精神上感到很压抑。”

有一天,毛泽东从书本上看到了清朝末年湖北英山县名士郑正鹄的一则故事,说是郑正鹄初授天水县令,一些官吏巨富见他五短身材,其貌不扬,有意奚落他一番,便请画工画了一幅《青蛙图》送给他。郑正鹄一看便知其意,稍加思索,当众挥笔在图上题了一首七绝诗《咏蛙》:

小小青蛙似虎形,河边大树好遮荫。明春我不先开口,哪个虫儿敢做声。

毛泽东看罢很受感染,便提笔改写了13个字,将它抄写在自己的作业本上:

独坐池塘如虎踞,绿荫树下养精神。春来我不先开口,哪个虫儿敢作声。

国文老师谭咏春先生看了这一首诗,非常兴奋,也未辨是毛泽东的“活剥”诗还是自作诗,大笔一挥,赞扬道:

“诗似君身有仙骨,寰观气宇。似黄河之水,一泻千里。”

当然,毛泽东在东山小学堂也并不是没有一个好朋友的。他的同班同学毛森品和他一样,因为个头高,就和他一起坐在了教室的最后一排。他们俩关系甚为密切,而且还时常开开玩笑逗逗乐子。

毛森品原名叫毛生炳,有一次,毛泽东跟他开了一个不大不小的玩笑。他见国文教员在毛森品的作文卷子上将“毛生炳”的“炳”字误写成了“柄”,便在旁边写了“毛内生出柄来”几个字。此后,毛森品就被同学们取笑为“毛生把”。毛森品非常生气,与他同在这个学校学习的哥哥毛钦明闻知此事,也为弟弟抱不平,责备毛泽东不该开这样的玩笑。毛泽东这才觉得自己的玩笑开过头了,就向毛钦明兄弟俩做了解释,说他的本意是想嘲笑老师竟然把学生的名字写错了,没想到却成了同学们嘲弄的把柄,真是该死。他当即向毛钦明和毛森品表示了歉意,还跟老师提议,将毛生炳的名字改名为毛“森品”,从而得到了毛森品兄弟俩的谅解。

且说毛泽东在东山小学堂学习期间,的确开阔了眼界。学校有一个藏书楼,他经常去那里借阅历史、地理方面的书籍,在掌握地理知识的同时,了解到了尧、舜、禹、秦始皇、汉武帝等等传说中的和历史上的杰出人物的故事,并且对这些古人非常仰慕。在藏书楼里,他如鱼得水,学识猛进,除了中国历史、地理以外,还学到了不少外国文学和自然科学方面的知识。

国文教员贺岚岗看到毛泽东如饥似渴地阅读历史书,很是欣赏,就特地买了一本《了凡纲鉴》送给他。这本书是明朝人袁黄仿照朱熹《通鉴纲目》体例编写的一种通史。

毛泽东在课余时间和同学们谈话时,常常对中国古典小说津津乐道。同学们都很钦佩他对《三国演义》这部小说的熟悉程度,喜欢听他复述其中的精彩片段。可毛泽东坚持认为,《三国演义》中所描绘的故事都是历史上曾经发生过的真实事情。为此,他在争论中用椅子打了一位讽刺他的同学,他也和历史老师争辩这一问题,还极力驳斥那些同意老师观点的同学。他是一个秉性刚直的人,他不会通过变通来保护自己。为了这事,他又找到校长那儿,校长也不同意他的观点。毛泽东生气极了,就给湘乡县令写了一封申辩书,还让一些同学在上面签了名。尽管为了这些小事闹得很不愉快,但因他写得一手好古文,还是在全校出了名,受到了校长和教员们特别是国文教员的青睐。

毛泽东写的《立志》、《救国图存论》、《宋襄公论》等作文,全校闻名。《宋襄公论》一文是用“康梁体”写成的,有一位教员认为只能给20分,可谭咏春先生看了以后,拍案叫绝,他说:

“毛咏芝的文章不仅思想进步,文笔泼辣,而且立志高远,见解精辟,令人折服呀!康、梁的文章有什么不好!好,好得很!”

他竟然破例给毛泽东打了个105分。

毛泽东的文章每次都被老师批上“传观”二字,贴在“揭示栏”内,作为同学们学习的范本。因此许多富家子弟又乐意和他接近交往了,其中就有萧子升和萧子暲兄弟俩。

萧子暲,1896年10月10日出生于湖南省湘乡县萧家冲一个书香门第,原名萧植蕃,号子暲,谱名克森,排行第三,所以他后来就以萧三为笔名。

萧子升,又名瑜,字旭东,1894年8月22日出生,排行第二,是萧子暲的二哥。

一天黄昏,毛泽东和萧子暲刚刚做完运动,一听到晚自修的铃声,就和同学们一起顺着学校第二道大门走向教室,他看到萧子暲手里拿着一本书,就问道:

“你那是什么书?”

“《世界英雄豪杰传》。”

这书名对于毛泽东有着极强的吸引力,他问萧子暲是否可以借阅一下。萧子暲笑嘻嘻地说:

“我借书给别人,向来是有点讲究的。”

“小弟愿闻其详。”

“我的书,有三种人不能借。”

“不知是哪三种人?”

“无真才实学者不借;庸庸小人者不借;三嘛,嘿嘿,我出上联而不能对出下联者不借。毛君自然不在前两条之列,但你要想读得此书,须对出我的上联方可。”

毛泽东听萧子暲如此说,便微微一笑,说道:

“小弟不敢自命才高博学,但阅书心切,请仁兄出上联如何?”

萧子暲说:

“我这书里讲的可都是英雄豪杰呀,请毛君听好了,我的上联是:‘目旁是贵,瞶眼不会识贵人;’快对,快对!”

这一上联是一句拆、析、合字的联语,用这种方法对出下联,自然难不倒毛泽东。毛泽东心里想的却是:借一本书尚且这等啰嗦还不说,而你这上联分明是把书中人物说成是贵人,相反却讥讽借书者不识英雄豪杰。于是他就想以“口边有亚,哑嘴岂能呼亚圣”来对萧子暲的上联。可他转念一想,这样对来,有伤同学,诙谐有余,庄重不足,于是就说:

“我就冒昧对下联,赠与仁兄。请听了:‘门内有才,闭门岂能纳才子?’”

萧子暲一琢磨,已知毛泽东下联的讽谏之意,没想到毛泽东才思如此敏捷,便诚恳地说:

“请恕小弟无礼,贤兄大才,愿为知己。”

说罢,他立即把《世界英雄豪杰传》借给了毛泽东,毛泽东连声道谢。

在此后的几天里,毛泽东手不释卷读完了这本书。他读到了拿破仑、彼得大帝、叶卡特琳娜女皇、惠灵顿、格莱斯顿、卢梭、孟德斯鸠和林肯的事迹,对这些人物的历史功绩深表钦佩。他盼望中国也有类似的人物出现,以拯救民族危亡。从此毛泽东开始注意西方了。后来他曾对萧子升这样说:

“中国也要有这样的人物。我们应该讲求富国强兵之道,才不致重蹈安南(即越南——笔者注)、朝鲜、印度的覆辙。你知道,中国有句古话:‘前车之覆,后车之鉴。’我们每个国民都应该努力。顾炎武说得好:‘天下兴亡,匹夫有责’。中国积弱不振,要使他富强独立起来,要有很长的时间。但是,时间长不要紧,你看,华盛顿不是经过8年艰苦战争之后才得到胜利,建立了美国吗?我们也要准备长期的奋斗。”

且说毛泽东读完了《世界英雄豪杰传》,在还书时,抱歉地对萧子暲说:

“很对不起,我把你的书弄脏了!”

萧子暲翻开书一看,毛泽东在整本书里用墨笔画了许多圈点,而圈点最密的是华盛顿、林肯、拿破仑、彼得大帝、叶卡特琳娜女皇、惠灵顿、卢梭、孟德斯鸠等人的传记。他感叹道:

“咏芝兄,但愿将来能望兄之项背。”

毛泽东在东山学校还读了从表兄文运昌那里借来的一些介绍康有为、梁启超变法的书报,读了梁启超主编的《新民丛报》,读了康有为撰写的《戊戌变法》。他很喜欢这些书报,反复阅读,有的甚至可以背诵出来。

此时的毛泽东非常崇拜康有为和梁启超。他在读梁启超写的《新民说》一文时,写了一些批语。在“论国家思想”一节旁,他分析了两种君主制国家:

“正式而成立者,立宪之国也,宪法为人民所制定,君主为人民所拥戴;不以正式而成立者,专制之国家也,法令为君主所制定,君主非人民所心悦诚服者。前者,如现今之英、日诸国;后者,如中国数千年来盗窃得国之列朝也。”

这一时期,毛泽东并不反对君主制度,他认为皇帝像大多数官吏一样,都是诚实、善良和聪明的人;只是需要由康有为、梁启超那样的维新派帮助变法改革。

毛泽东在东山学堂除了努力学习,还坚持锻炼身体。他每天早晨一起床就在围墙外面的环校跑道上练习长跑;环绕学校有一条河,水深且宽,他坚持在河中游泳,搏击风浪;学校南面有一座海拔500米的东台山,他经常沿着一条崎岖陡峭的小路,登上顶峰,在山上的八角亭里休息一会儿,然后再跑步下山。

转眼间,一个学期将要结束了,谭咏春看到毛泽东成绩优异,胸怀大志,为了让他更好的成长,便与李元甫校长、贺岚岗先生商量着推荐毛泽东到长沙驻省湘乡中学去学习。谭咏春先生将此美意告诉了毛泽东,毛泽东听了自然是很高兴的,可他不得不说出家里无钱供应的苦恼,谭咏春忙说:

“不要紧,我和几位先生推荐你去,吃公费。”

毛泽东欣喜异常,连声道谢。

后来在1936年,毛泽东在陕北的窑洞里对斯诺说:“我开始向往到长沙。长沙是一个大城市,是湖南省的省会,离我家120里。听说这个城市很大,有许许多多的人,不少的学堂,抚台衙门也在那里。总之,那是个繁华的地方。”

且说毛泽东在春节前回到了韶山,父亲毛顺生温和多了,他问儿子:

“你什么时间才能完成学业,回来当上先生,光宗耀祖?”

毛泽东在寒假期间少不了要到外婆家去,把自己买来的笔杆糖平均分给表兄弟们,最小的表弟吵着还想要。毛泽东就说:

“农民领袖李自成起义,与昏庸的官府作斗争,为的是使天下人均田均富。我们兄弟之间也要按均田均富的原则来分糖,谁也不能多分。”

这样一说,小表弟也就不再闹了。

毛泽东还和二姑家表弟贺晓秋相约一块儿去长沙读书。贺晓秋是和他一块儿长大的,还曾在同一个私塾里读过书,同练一种字体,二人十分相投。可二姑不同意让贺晓秋去长沙读书,她说:

“还是留在家中吧,现在形势是一天一变,三天风四天雨,不能指望着读书来糊口,不能指望着把文章放进锅里煮了当饭吃。”

贺晓秋只好作罢。

后来毛泽东在离开家乡时,改写了前人的一首诗,并将这一首修改后的七言诗夹在父亲常看的账本里,以表达他一心向学、志在四方的抱负。诗曰:

\begin{emph}
\centering 孩儿立志出乡关,学不成名誓不还。
		
		埋骨何须桑梓地,人生无处不青山。\par
\end{emph}

毛顺生拿着这首诗去向李漱清请教。李漱清先生读罢,竖起大拇指称赞道:

“好诗,好诗。石三伢子硬是有志气。”

据徐涛考证,毛泽东改写的原诗,是中国宋代以后一位名叫月性的和尚所写的《题壁诗》:男儿立志出乡关,学若不成死不还。埋骨何期坟墓地,人间到处有青山。

而据逄先知以及高菊村等人说,毛泽东改写的原诗,是日本明治维新时期著名的政治活动家西乡隆盛所作:男儿立志出乡关,学不成名死不还。埋骨何须桑梓地,人生无处不青山。

另据郑松生考证,原诗作者是与西乡隆盛同时代的日本和尚释月性,他在27岁离开家乡时写了一首《题壁》:男儿立志出乡关,学若无成不复还。埋骨何期坟墓地,人间到处有青山。

此时的毛泽东读书甚多,他所看到的原诗究竟是哪一首,目前尚难断定,只有留待以后新史料的发现了。

这种改写别人诗句的创作手法,其作品在文学上被称之为“剥体诗”。前面提到的《咏蛙》便属于这一种。后来毛泽东在他一生60多年的诗词创作中,也常常使用这种创作手法。

这正是:\begin{xemph}身入异乡为异客,伯乐誉为建国才。

\hspace{4em}一识中外豪杰范,雏凤清声凌空来。\end{xemph}

欲知毛泽东到长沙如何发展,请看下一章叙述便知。

东方翁曰:我在上世纪九十年代第一次看到《中国青年报》上有一篇文章说,《咏蛙》一诗为毛泽东所作。没隔几天,我又在这个报上看到一位专家举证说,此诗乃是清人郑正鹄所作。可后来还有一些相互传抄的资料依然说《咏蛙》是毛泽东的诗作。因此,在开始写作本传时,我一直不敢采用这首诗。其原因一是还没有把事情的原委弄清楚,二是一些人正在议论毛泽东如何如何专制,甚至还有意识地挖掘其专制的根源。如果贸然采用了这首诗,不正好给那些持有这种观点的人提供了依据吗?待到后来,我终于弄明白了,这首诗原是毛泽东在东山小学堂时写在作业上的对郑正鹄《咏蛙》诗的活剥之作,这才敢于采信了。斯时斯地斯人,如此活剥,完全合乎毛泽东的性格。而对于那些借此污蔑毛泽东的人,完全可以断言,不是无知便是别有用心的了。


\end{document}
