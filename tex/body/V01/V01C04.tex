\documentclass[../../dazhuan.tex]{subfiles}
% 第一卷
\begin{document}
\chapter*{第四章}
\pdfbookmark{第四章}{V01C04}
\begin{pref}
	我就像一头牛闯进了菜园子,见到遍地青菜,拼命地大嚼大吃,嚼个不停。
\end{pref}

话说在1911年春天,毛泽东怀着极其兴奋的心情离开了东山高等小学堂,跟着赴长沙任教的贺岚岗先生来到省城长沙,考入了驻省湘乡中学,开始了他人生中至关重要的一段历程。

毛泽东在驻省湘乡中学一边刻苦学习,一边以极大的热情关注着社会动态。在这里,他第一次看到了资产阶级革命家们创办的《民立报》。从此,读报成了他终生的爱好。

《民立报》是著名的同盟会会员宋教仁、于右任主编的。毛泽东到长沙不久,从这份报纸上看到了4月27日黄兴在广州领导的黄花岗起义;了解了在这次武装起义中英勇殉难的七十二烈士的事迹。从这份报纸上,他还知道了孙中山的革命活动和同盟会的纲领,开始拥护孙中山等革命党人。

1911年5月,清政府颁布了“铁路国有”的政策,强行把商办的川汉、粤汉铁路改为官办,并把筑路权出卖给帝国主义,换取帝国主义的借款。

清政府的卖国行为引起了全国人民的强烈反对。保路风潮首先从湖南掀起,四川、湖北、广东等省也立即行动,一个轰轰烈烈的保路运动开始了。

毛泽东和驻省湘乡中学的同学们一起投入了这场斗争。他首先倡议并和另一名同学带头剪掉了自己的辫子,以此表达他们对清王朝的强烈不满。可没想到原先“相约剪辫子”的十几个同学又后悔了,毛泽东和他那位朋友对这十几个不守约的同学进行了突然袭击,强行剪掉了他们的辫子。

在剪辫子问题上,毛泽东还和一个政法学堂里的朋友发生了争论。这位政法学堂学生引用经书来为自己的论点找论据,认为身体发肤是受之父母,不可毁伤;毛泽东则和反对留辫子的朋友站在反清的立场上,提出相反的理论,驳得那位朋友哑口无言。

这个时候的中国,正处于辛亥革命的前夜,清王朝已经四面楚歌,很难照旧统治下去了。毛泽东的心情非常激动,激奋之下,他写了一篇文章贴在学校门口的墙壁上。这是他第一次发表自己的政治见解。他在这篇文章中提出,要把孙中山从日本请回来,当新政府的总统,由康有为当国务总理,梁启超当外交部长。

1911年10月10日,武昌新军突然发动武装起义,辛亥革命爆发了。离武昌不远的长沙城受到了强烈震动,形势变得异常紧张起来,湖南巡抚立即宣布全城戒严。

革命党人并没有停止他们的活动。有一天,一个革命党的宣传家得到驻省湘乡中学校长的许可,来校作了一次激动人心的演讲。他介绍了武昌起义的经过和武汉三镇的形势,鼓动大家立即行动起来,为了祖国,为了革命,投笔从戎救援湖北起义军。有七八个学生当场站起来支持他的主张,强烈抨击清政府,主张推翻满人统治,建立民国。

毛泽东也非常激动,他用了四五天的时间规划自己的未来,决心投笔从戎,参加革命军。于是他和几位同学商量说:

“武昌起义是成功了,可是离革命的胜利还远得很哩。我想,革命不能光靠嘴巴讲,要靠实际行动。我们到底应该用什么样的实际行动来帮助湖北起义军呢?我已经想好了,当兵去!”

“当兵?”

同学们都睁大了眼睛望着毛泽东。毛泽东坚定地说:

“对!当兵。到湖北去当革命军!现在武汉三镇的形势很危急,袁世凯的北洋兵分两路南下,采取包抄战术,企图钳住武汉。革命党人势单力薄,困难一定不小。我们既然要革命,就不能空手讲白话。光喊口号是打不倒宣统皇帝的。要革命就要到战场上去,真刀真枪地干!”

在他的鼓动下,有4个同学决定和他一道投笔从戎,到汉口去。毛泽东听说汉口的街道潮湿,需要穿雨鞋,就去在城外军队里干事的一位朋友那里借一双雨鞋。

10月22日,毛泽东从小吴门出城,向协操坪走去。他来到兵营门口,被卫兵拦住了,那卫兵指指悬挂在门口“禁止通行”的牌子,告诉他说:

“你不能进去了,赶快回去吧!我们马上就要起义。你看,已经补发了子弹。”

毛泽东心头一震,只好转身回城,还没走几步,就听见兵营里突然响起了一阵急促的哨子声,回头一看,全副武装的士兵们正在操坪上紧急集合。他不敢久留,快步离去,到了经武门,见城门已经关闭了,只好沿着护城河向前走,待走到云阳门口,见这里的城门也关闭了。此时的毛泽东还不知道,在长沙城里配合新军行动的暴动已经发生了,各个城门都被暴动者所占领。

毛泽东进不了城,只好跑上留芳岭,站在一个高地上观望。不多时,噼哩啪啦的枪声接连不断,城外的起义军向云阳门猛扑过来。城内的接应者打开城门,城外的新军潮水般地拥了进去。就这样,湖南新军在焦达峰、陈作新的率领下,攻占了巡抚衙门,升起“汉”字白布旗子。长沙起义成功了。

毛泽东兴致勃勃地回到学校,看到学校里也挂起了汉字白旗,门口还站了几个士兵。

10月22日深夜,在焦达峰、陈作新领导下,起义军建立了“中华民国湖南军政府”。焦达峰任都督,陈作新任副都督。

长沙城里一片欢乐气氛。许多学生投入到起义军中。很快,一支学生军组织起来了,可毛泽东不喜欢这支学生军,他认为学生军的成份太混杂了。他决定参加正规军,因为正规军里的士兵都是社会上的穷苦人出身,他愿意和这些人战斗在一起。

10月底,毛泽东和他的两个同学一起参加了起义军。他被编入湖南新军二十五混成协五十标第一营左队,当了一名列兵。不久,他就同排长和大多数士兵交上了朋友。在这些朋友中,毛泽东最喜欢的是湖南籍的一个矿工和一个铁匠。由于他有文化能写东西,可以帮助别人写写信,所以大家都认为他很有知识,敬佩他有“大学问”。

毛泽东在新军中认真地接受军事训练,同时研究时事和一些社会问题。他每月7元饷银,除了用于伙食2元和买水花去一点外,剩下的大都用在购买报纸上。每天操练结束以后,他就坐下来看报,成了军营里一个好读报的人。

毛泽东买的都是一些左翼报纸,并将这些报纸奉为至宝。他从鼓吹革命并正在讨论社会主义的《湘江新闻》上,第一次看到了一位曾经留学日本的湖南人创立了一个“社会主义”党。这是毛泽东第一次接触到“社会主义”这个新名词。他还读了江亢虎写的一些关于社会主义及其原理的小册子。由此,他对社会主义问题产生了浓厚的兴趣。他热情洋溢地给以前的同班同学写信,向他们介绍社会主义这个颇有吸引力的新概念;同时,他还和那些士兵们一起讨论“社会主义”问题。

此时的革命形势正在急剧变化。

1912年2月,袁世凯在帝国主义支持下,采取以革命吓唬清政府的手段,迫使清政府取消了皇族内阁,由他出任内阁总理大臣,组织内阁。他又玩弄“拉”的一手,对革命党的上层实施诱和;同时使出“打”的一手,用反革命武力直接对革命党人进行威胁。

1912年春,正当湖南新军准备采取行动反对清政府、反对袁世凯的时候,在南京的中华民国临时大总统孙中山,却与袁世凯达成了妥协协议,辞去临时大总统职务。几乎与此同时,满清皇室接受了袁世凯的优待条件,溥仪退位了。

1912年3月间,袁世凯在北京就任临时大总统。

此时的毛泽东以为革命已经结束了,便决定退出军队,继续求学深造。连长、排长和好朋友们都劝他留下来,他谢绝了长官和战友们的挽留。当初他之所以当兵,是认为军队将会在革命中起到重要作用;现在他认为军队已经不再是时代的先锋了,所以就丝毫也不留恋军队的生活了,毅然决然地离开了兵营,结束了半年的军旅生涯。

这正是:\begin{xemph}才看大地惊春雷,又闻故都起阴霾。\end{xemph}


且说毛泽东回到长沙城里,住进了很便宜的新安巷“湘乡会馆”。长沙有各种各类的学校,各个学校都通过报纸招徕新生。他不断查阅《湘汉新闻》和其它报纸上的招生广告。革命既然已经完成了,那么革命以后应该选择什么学校、什么专业呢?毛泽东对自己将来究竟想做什么还没有明确的主见,所以他在学校和专业的选择上举棋不定。就在这时,韶山家里捎来信说,父亲以后不再从经济上支持他了。事到如今,已经没有了退路了,他必须选择一个能谋一份差事的学校。

毛泽东最先看到一个警察学堂的广告,说是将来可以当警官,于是他就花了1元钱去报了名。但在考试之前,他又被一所肥皂制造学校的广告打动了。这则广告说,该校不收学费,供给膳宿,并有津贴;还说制造肥皂对社会大有好处,可以富国利民。他便改变了报考警察学堂的念头,决定将来做一个肥皂制造家。他又花了1元钱的报名费,报考了肥皂制造学校。后来一个在法政学堂学习的朋友,劝他进法政学堂。这所学堂也在广告上许下诺言说,只要学生在3年内学完全部法律课程,保证期满之后马上可以当官。毛泽东第3次付了1元钱的报名费,报考了法政学堂。他给家里写了一封信,重述了广告上许诺的一切,并描绘了“将来当法律学家和做官的美好图景”,要求家里寄学费来。正当他等候父母回音的时候,另一位朋友劝他报考经济学校。朋友说,现在祖国正处于经济战争之中,最需要的人才是能建设祖国经济的经济学家。他又动心了,又付了1元报名费。这是他报考的第4所学校,考试后还真被校方录取并注了册。

此时的毛泽东并没有专心下来,还在继续注意广告。一则把一所公立高级商业学校说得天花乱坠的广告吸引了他,广告上说这所学校是政府主办的,设有很多课程,而且教员都是很有才能的人。他想,如果在那里能够学成一个商业专家是最好的了。于是他又写信把这一决定告诉了父亲,父亲着实高兴了一场,同意支付学费。因为父亲是知道善于经商的好处的。毛泽东又交了1元报名费,考入了这所公立高级商业学校。可他在这里只学习了一个月,就又退学了,原因是这所商业学校的课程大多都是用英语讲授的,而毛泽东的英语基础并不好,只在东山高等小学堂学了一点入门知识,现在学起来很艰难;更主要的原因还在于学校里缺乏比较好的英语教师。

毛泽东回忆说:“这种情况使我不满,到了月底,我就退学了,并且继续留心报上的广告。”

他身上已经一文不名了,邋里邋遢地站在长沙街边的木茶棚里,瞪大眼睛盯着报纸看广告。即便如此窘迫,他却还是以嘲弄的态度对待周围的生活,俯瞰着忙忙碌碌的芸芸众生。

“我即宇宙!”

毛泽东从道家的冥想得出结论。

湖南省政府军的火药库爆炸,烈焰熊熊,毛泽东和他的朋友们一道去观赏。一年前他曾经激情满怀地参加了这支军队,可现在他却以旁观者的口吻嘲讽说:

“这比放爆竹要好看多了。”

有一天,有3个学友在天心阁的顶楼上碰见了毛泽东。他正独自平静地在城墙上这个七层高塔上俯瞰长沙。于是4个人便相约一起去喝茶、吃瓜子。他这3个同学在社会地位上都比他高一等,其中一个是常常借钱给他,一个姓谭的是大官的儿子。那位姓谭的同学说:

“君主制的废除,就意味着我们都可能当总统。”

旁边一个同学用一些俏皮话揶揄姓谭的。而毛泽东却显得有些激动,说:

“让他说,我很感兴趣,让他说罢!”

姓谭的解释说:

“对于一个政治领袖来说,学问是次要的,而重要的是斗争意志。”

毛泽东被他这种看法深深地吸引住了,他开始思考这一问题。

不久,毛泽东报考了湖南全省高等中学校,入学考试的作文题目是《民国肇始,百废待兴,教育、实业何者更为重要》。他以资产阶级维新派的理论为依据,下笔滔滔,如长江大河一泻千里,言之凿凿地提出了一种教育为主说。张榜结果,他名列第一。

比毛泽东年长16岁的校长符定一,很想知道夺得头名的青年人是个什么样子,于是,毛泽东就被人带到了校长办公室。符定一定睛一看,这个青年一双眼睛炯炯有神,高高的个子显得有些瘦弱,身着一件旧长衫,脚上一双圆口旧布鞋,虽是一副农家子弟打扮,却显得稳重大方,气质不凡,说起话来还慢条斯理的,就非常满意。

湖南全省高等中学校秋季更名为省立第一中学。这个学校很大,学生也很多,名师也不少,首任校长符定一还亲自授课。毛泽东曾经聆听了符定一有关古典文学、历史、伦理等方面的教导。在一次作文竞赛中,毛泽东以刚健充实的内容,纵横捭阖的气势,严谨善辩的推理,获得了全校第一名。符定一没有想到这位来自湘潭乡村的后生,竟能在莘莘学子之中一举夺魁,便要面试一下毛泽东。他把毛泽东叫到办公室里,叫毛泽东写一篇作文。毛泽东见先生如此器重自己,顷刻间文思泉涌,一挥而就。符定一拿过来一看,果然是妙笔惊人,文采斐然,心中大喜,便决心认真栽培这位弟子。

毛泽东成了这所学校的高材生,他在这里还写了不少好作文,其中《商鞅徙木立信论》是最著名的一篇,受到国文教员柳潜的高度赞赏。

《商鞅徙木立信论》全文如下:
\begin{xquote}

吾读史至商鞅徙木立信一事,而叹吾国国民之愚也,而叹执政者之煞费苦心也,而叹数千年来民智之不开、国几蹈于沦亡之惨也。谓予不信,请罄其说。

法令者,代谋幸福之具也。法令而善,其幸福吾民也必多,吾民方恐其不布此法令,或布而恐其不生效力,必竭全力以保障之,维护之,务使达到完善之目的而止。国民政府互相倚系,安有不信之理?法令而不善,则不惟无幸福可言,且有危害之足惧,吾民又必竭全力以阻止此法令。虽欲吾信,又安有信之之理?乃若商鞅之于秦民适成此比例之反对,抑又何哉?商鞅之法,良法也。今试一披吾国四千余年之记载,而求其利国福民伟大之政治,商鞅不首屈一指乎?鞅当孝公之世,中原鼎沸,战事正殷,举国疲劳,不堪言状。于是而欲战胜诸国,统一中原,不綦难哉?于是而变法之令出,其法惩奸宄以保人民之权利,务耕织以增进国民之富力,尚军功以树国威,孥贫怠以绝消耗。此诚我国从来未有之大政策,民何惮而不信?乃必徙木以立信者,吾于是知执政者之具费苦心也,吾于是知吾国国民之愚也,吾于是知数千年来民智黑暗国几蹈于沦亡之惨境有由来也。虽然,非常之原,黎民惧焉。民是此民矣,法是彼法矣,吾又何怪焉?吾特恐此徙木立信一事,若令彼东西各国文明国民闻之,当必捧腹而笑,噭舌而讥矣。呜呼!吾欲无言。
\end{xquote}

柳潜看罢此文,禁不住拍案叫绝,给这篇作文打了个100分。他还在这篇571字的短文中写下了150个字的眉批和总评:

“实切社会立论,目光如炬,落墨大方,恰似报笔,而义法亦骎骎入古。”

“精理名言,故未曾有。逆折而入,笔力挺拔。”“议论潇洒,积理宏富”;“力能扛鼎。”

“有法律知识,具哲理思想,借题发挥,纯以唱叹之笔出之,是为压题法,至推论商君之法为从来未有之大政策,言之凿凿,绝无浮烟涨墨绕其笔端,是有功于社会文字。”“历观生作,练成一色文字,自是伟大之器,再加功候,吾不知其所至。”

末了,柳潜又在这篇作文题目的上方写了“传观”二字。

这国文教员柳潜原是清朝末年的一名秀才,早年酷爱读书,学识渊博,颇有才华。他在青壮年以后目睹官场腐败,遂放弃仕途,以教书为业,被湖南全省公立高等中学校(后改名省立第一中学)首任校长符定一聘请为国文教师。后来柳潜因生活困顿,曾先后在福建和长沙等地做过几年幕僚,后又返回学校从事教师职业,但终因积劳成疾,贫病交加,于1930年在长沙去世,终年52岁。

且说符定一校长知道毛泽东喜欢读课外书籍,便主动借给他一部上自远古下至明朝、一共有120卷的中国历代编年史《御批历代通鉴辑览》。

尽管此时的毛泽东在省立第一中学声名鹊起,可他感到这个学校设置的课程太多太繁,校规也繁琐呆板,在学校学习还不如自学更好一些,所以他决定去定王台图书馆自由读书,独自研究学问。符定一得知此事,甚感惋惜,便极力劝阻他。毛泽东去意已决,婉言谢绝了先生的好意,毅然离开了学习生活了半年的省立第一中学,寄居在长沙新安巷的湘乡会馆,开始了在定王台的自修生活。

定王台位于长沙东南角,相传为西汉景帝之子长沙定王刘发所筑。那时候,他每年都要挑选出上好的大米,命专人专骑送往长安孝敬母亲,再运回长安的泥土,在长沙筑台,年复一年,终于用从长安运来的泥土筑成了一座高台。每当夕阳西下之时,刘发便登台北望,遥寄对母亲的思念之情。后来,长沙人就把这个土台子叫作定王台,又称“望母台”。到了清朝末年,土台子已经荡然无存了,有人就在这个地方盖起了一栋两层楼的洋房。辛亥革命后,省政府当局接受一些学者的建议,利用这栋房子办起了湖南图书馆,购置了不少新书,其藏书量居全省之首。这个地方比较偏僻,周围树木葱茏,环境幽静,的确是一个难得的读书好场所。

定王台图书馆每天一开门,毛泽东总是第一个进去,中午仅仅休息片刻,到街上买两块米糕当午饭,下午依然是一动不动在书桌旁埋头苦读,好像一尊低着头的雕塑一般。晚上闭馆时,他又是最后一个出来。

毛泽东在这里广泛涉猎了18、19世纪欧洲资产阶级的社会科学和自然科学著作。他读了达尔文的《物种起源》,亚当·斯密的《原富》,赫胥黎的《天演论》,穆勒的《名学》,斯宾塞尔的《群学肄言》,孟德斯鸠的《法学》,卢梭的《社会契约论》等等。他还认真研读了一些俄、美、英、法等国的历史、地理书籍。同时他也阅读一些诗歌、小说和古代希腊、罗马的故事。

毛泽东在定王台图书馆第一次看到了一幅世界地图,叫作《世界坤舆大地图》,不禁惊叹道:“世界原来这么大!”他每天经过地图前,都要停下脚步细细地看上一阵,以极大的兴趣认真研究一番。

40年后的一个晚上,毛泽东和几个同学说起他在定王台看世界大地图后的思想状况,他说:“说来也是笑话,我读过小学、中学,也当过兵,却不曾看见过世界地图,因此就不知道世界有多大。湖南省图书馆的墙壁上,挂有一张世界大地图,我每天经过那里,总是站着看一看。过去我认为湘潭县大,湖南省更大,中国自古就称为天下,当然大得了不得。但从这个地图上看来,中国只占世界的一小部分,湖南省更小,湘潭县在地图上没有看见,韶山当然就没有影子了。世界原来这么大。世界既大,人就特别多。这么多的人怎么过生活,难道不值得我们注意吗?从韶山冲的情形来看,那里的人大都过着痛苦的生活,不是挨饿,就是挨冻。有无钱治病眼看着病死的;有交不起租谷钱粮被关进监狱活活折磨死的;还有家庭里、乡邻间为着大大小小的纠纷吵嘴、打架,闹得鸡犬不宁,甚至弄得投塘、吊颈的;至于没有书读,做一世睁眼瞎子的就更多了。在韶山冲,我就没有看见几个生活过得快活的人。韶山冲的情形是这样,全湘潭县、全湖南省、全中国、全世界的情形,恐怕也差不多。我真怀疑,人生在世间,难道都注定要过痛苦的生活吗?决不!为什么会有这种现象呢?这是制度不好,政治不好,是因为世界上存在着人剥削人、人压迫人的制度,所以使世界上大多数的人都陷入痛苦的深潭。这种不合理现象,是不应该永远存在的,是应该彻底推翻、彻底改造的!总有一天,世界会起变化,一切痛苦的人,都会变成快活的人,幸福的人。世界的变化,不会自己发生,必须通过革命,通过人的努力。我因此想到,我们青年的责任真是重大,我们应该做的事情真多,要走的路真长。从这时候起,我就决心要为全中国痛苦的人、全世界痛苦的人贡献自己的全部力量。”

毛泽东在评价他在省立图书馆这一段的学习生活的时候,还曾笑着对萧三说:

“我就像一头牛闯进了菜园子,见到遍地青菜,拼命地大嚼大吃,嚼个不停。”

且说毛泽东在定王台图书馆学习期间,每晚都要回到“湘乡会馆”里住。这个会馆里还住着其他一些穷学生,住着过路客、流浪汉和一些当过兵的人。那些当过兵的人都是一些退伍军人或者是被遣散的士兵,他们既没有工作,也没有钱,性情暴躁,喜怒无常,经常因为一些琐事和学生们吵架斗殴。有一天晚上,那些士兵和学生再次发生了冲突,酿成了一场血腥的武斗。毛泽东一直躲在厕所里,直到殴斗结束后才走了出来。这样的环境实在是待不下去了,也就在这个时候,父亲来了一封信,说是不赞成他自修。父亲认为他是不务正业,因此拒绝再提供费用,除非他继续进学校读书。就这样,毛泽东被迫结束了半年的他认为是“极有价值”的自修生活。

这正是:\begin{xemph}方入菜园饥如牛,立有乱兵扰梦来。\end{xemph}


欲知毛泽东去向何处?且看下一章分解。

东方翁曰:毛泽东投笔从戎,是他第一次的人生抉择,第一次的社会革命实践,仅仅半年就结束了,可谓是来也匆匆,去也匆匆。他又三番五次地报考各类学校,在人生坐标的再选择上,彷徨复彷徨,几番求索、几番曲折之后,这才似乎看到了前途的一丝光亮。不光毛泽东是这样,恐怕有不少人在年轻的时候都会有如此这般的一番经历罢。

\end{document}
