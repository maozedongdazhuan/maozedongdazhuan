\documentclass[../../dazhuan.tex]{subfiles}
% 第一卷
\begin{document}
\chapter*{第五章}
\pdfbookmark{第五章}{V01C05}
\begin{pref}
	给我印象最深的教员是杨昌济,他是从英国回来的留学生,后来我同他的	
生活有密切关系。他教授伦理学,是一个唯心主义者,一个道德高尚的人。
\end{pref}

话说毛泽东被迫离开定王台后,又开始重新考虑自己的前程,在费用已无保障的情况下,他又去查阅广告栏。湖南省立第四师范学校登出一则不收学费、膳食费用很低的招生广告,自然引起了他的兴趣。此时他的两个朋友也来鼓励他报考四师,他们是指望毛泽东在入学考试的时候帮帮他们。毛泽东答应了,于是便写了3篇文章备考。他并不认为自己捉刀替朋友写文章是不道德的行为,他说,这种小活动只不过是事关友谊的不得已而为之的小事情而已。就这样,年届二旬的毛泽东报考了第四师范学校。校长看到他的作文考卷大加赞许,对同事们说:

“这样的文章,我辈同事中有几个能够做得出来?”

毛泽东和他的那两个朋友都被录取了,顺利地进入了第四师范学校。

毛顺生同意了儿子的这一选择,又给他寄来了钱。从此以后,毛泽东抵制了一切吹嘘未来前途的广告的诱惑,在第四师范学校预科一班学习了一年。

毛泽东在四师结识了不少良师益友,其中一个是他的同班同学周世钊,两人关系非常好。另一个也是最重要的一个,就是他在第四师范结识的恩师杨昌济。他在第四师范所做的课堂笔记《讲堂录》中所记录的修身课,就是杨昌济所教。杨昌济看到农家出身的毛泽东好学深思,成绩优异,十分赞赏。毛泽东对杨先生的渊博学识和诲人不倦的精神也非常敬佩。

周世钊,1897年出生于湖南宁乡,字惇元,又名敦元、东园。

杨昌济,又名怀中,字华生。1871年6月8日(清同治十年四月廿一日)出生于湖南省长沙县清泰乡隐储山下板仓冲下屋的一个书香门第,其高祖、曾祖父都是“太学生”,祖父杨万英是一个“邑庠生”,父亲杨书祥捐过一个“例贡生”。他自幼受到中国正统文化的熏陶,1889年参加长沙县学试,一举考上“邑庠生”,1890年、1893年两试“举人”不第,1898年进入岳麓书院读书,积极参加了谭嗣同、唐才常等人在湖南组织的维新改良活动。1903年3月,杨昌济同陈天华、刘揆一、李傥、石醉六等人一起去日本留学,启程前改名为杨怀中,表示自己虽然身在异邦,却心怀中华。他先后入东京弘文学院、东京高等师范学校学习。1909年,他赴英国进修,专攻哲学、伦理学和心理学;1912年获文学学士学位。随后,他先后在德国、瑞士考察了9个月。考察活动结束后,他回到了阔别10年的祖国。在省城,这位过激的绅士因为提倡寡妇改嫁而震动了整个长沙。他谢绝了湖南都督谭延闿聘他为湖南省教育司长的邀请,立志于教育救国事业。他当时写的一副对联是:

\begin{couplet}
自避桃源作太古;欲栽大木拄长天。
\end{couplet}

1913年,杨昌济成为湖南高等师范学校教授,又应第四师范之邀,兼任该校的修身课和心理课。

1936年毛泽东对斯诺说:“给我印象最深的教员是杨昌济,他是从英国回来的留学生,后来我同他的生活有密切关系。他教授伦理学,是一个唯心主义者,一个道德高尚的人。他对自己的伦理学有强烈信仰,努力鼓励学生立志做有益于社会的正大光明的人。”

那时候的杨昌济把全家人从乡下迁到了长沙,住在天鹅塘。他在自己寓所的门上,挂着一块1尺来长、3寸多宽的铜牌,上面用隶书镌刻着“板仓杨”3个大字。在长沙5年多的时间里,这块铜牌跟随他先后换了四五个地方,不管这块铜牌挂在哪里,都有一批青年学子寻踪而来,登门求教。

在天鹅塘杨宅,毛泽东认识了杨昌济的爱女杨开慧。他第一次到杨昌济老师家,一进屋看见天真文静的杨开慧对着自己微笑,便不由自主地说:“你就是小霞?”他们之间似乎是早已认识了一般。

杨开慧,1901年11月6日出生于湖南省长沙县板仓,乳名霞,字云锦。父亲希望她像朝霞一样灿烂、火红、美丽。在父亲怀着救国救民的抱负远涉重洋留学日本时,杨开慧还不满3岁,她跟着母亲向振熙在板仓乡下度过了童年。稍长,她便帮助妈妈干些家务活,还跟着哥哥杨开智上山扒柴。

杨开慧7岁那年,父亲从国外来信,嘱咐向振熙让女儿上学读书。山冲里本来是不让女孩子们读书的,在向振熙的斡旋下,与杨家斜对门的杨公庙官立第40初级小学破例为杨开慧和另外6个女孩子办了一个班,首开了板仓女孩子上学读书的先河。

杨开慧在杨公庙只读了3个学期,便转到了离板仓5里多路的隐储学校。隐储学校比杨公庙小学大,图书也多。

辛亥革命后,杨开慧根据父亲来信的要求,和妈妈向振熙一起到离家20多里的衡粹实业女校读书,母女俩同校学习,妈妈读实习班,她读附设小学班。后来她们又转到麻林桥附近的县立第一女子高小,一直读到毕业。

杨开慧聪颖好学,10岁时已读了许多诗歌小说,她最喜爱的是古文诗歌《木兰辞》。后来,她又阅读了不少社会科学、自然科学方面的书籍,打下了良好的文字功底,并练出了一手好毛笔字。1912年深秋,她曾写过这样一封信,可见她的文字功底之一斑:

\begin{xquote}
\noindent 我最爱之姐姐鉴:

许久未晤,甚以为念。近维起居多祜,学业日增为颂!妹现发头昏,且生痱子,请医诊治,总难见效。校中的课堆积,偶一思及,颇为之焦灼也。妹与吾姊至好,素承规劝,有暇望赐教行,以慰系念。天气将寒,惟珍重不一。此问大安。

\sign{愚妹\quad 杨开慧\quad 书上}

\end{xquote}

杨开慧随着父母迁入长沙后,就没有再进入学校,而是在父亲的辅导下进行自学了。

且说在1914年2月,湖南第四师范学校并入了第一师范,毛泽东、周世钊和其他三四百名身穿蓝色毛纺制服的学生,跟着恩师杨昌济及历史教员黎锦熙、国文教员袁仲谦等人一起转入了第一师范学校。此时的杨昌济除了在湖南高等师范学校和第一师范授课外,还到长沙第一联合中学、湖南商业专门学校任过教。

湖南省立第一师范学校坐落在湖南长沙南门外妙高峰下的书院坪,它的对面是滚滚北上的湘江水,隔江放眼,西岸便是那有名的郁郁葱葱的岳麓山。

这里是一个历史悠久的教育基地,历史上叫作城南书院,是南宋理学家张南轩讲学的地方,与朱熹讲学的岳麓书院隔湘江相望。1903年,有关方面在城南书院的基础上创办了湖南师范馆,辛亥革命以后,湖南师范馆这才正式定名为第一师范学校。校园内树木葱郁,环境十分优雅。校舍是清一色的仿西式建筑风格的两层楼房,圆柱、拱顶;正中一座大楼的顶端镌刻着“湖南第一师范”几个醒目的大字。

这是一所免费的公立中等学校,主要培养小学教师。它和中国的其他中高等学府一样,是一所比较民主的开明的新式学校。学校外墙上书写的校训是“实事求是”;在管理方面,校方制定的教育方针是包括“道德实践”、“身体活动”、“社会活动”在内的“三育并重”;学校还有校歌,还制定了校旗、制服。每当集会的时候,身着一色制服的莘莘学子那雄浑的校歌之声,就飘荡在校园的上空:

衡山西,岳麓东,城南讲学峙其中。

人可铸,金可熔,丽泽绍高风。

多材自昔夸熊封。

男儿努力,蔚为万夫雄。

此时的湖南第一师范学校可谓是人才荟萃,不仅拥有一批学识渊博、思想进步、品德高尚的教师,而且还招收了一大批追求进步的热血青年,堪称湖南省培养新青年的摇篮。

一师同四师的春季招生不同,是秋季招生。因此毛泽东和四师转来的同学一样,需要读半年预科,这样他就被编入了预科第3班。到了1914年秋,毛泽东才和周世钊一同被编入了本科第8班。

此时的毛泽东仍然很瘦,常常穿一件灰色长袍,留着长长的头发,一双大眼睛显得炯炯有神。他的言谈举止还是慢条斯理的。他不是那种讲起话来滔滔不绝、指手画脚的学生,尤其是在师生或同学们聚会时,他更是很少说话。在学习上,他依然非常用功,在课堂上静心听讲,认真做笔记;在课余时间或是进图书馆或是进阅览室,或是找同学和老师交谈学问。他与学友蔡和森、陈昌、张昆弟、罗学瓒、萧子升、萧子暲、周世钊等交往密切,志趣相投。他们中的多数都是来自农村,非常了解农民的疾苦。他们聚在一起,研究治学做人的道理,讨论个人和祖国的前途问题。

罗学瓒,号荣熙,1894年出生于湖南湘潭县马家河南岸一个农民家庭。1913年,他在长沙明德中学毕业后,考入湖南第四师范,不久转入第一师范,和毛泽东同在第8班。

蔡和森,湖南湘乡县永丰镇人,1895年3月30日出生于上海,原复姓蔡林,名和仙,字润寰,号泽膺,学名彬。1913年秋,他考入湖南省立第一师范,被编入第6班,和毛泽东一样,长于作文,且词高意远,有独特见解。

陈昌,又名章甫,1894年7月14日出生于广西梧州,1902年随父回到故乡湖南省浏阳县陈家大屋。他是毛泽东一师的同学和好朋友。陈昌学习刻苦,自强不息。课堂之外,不是在阅报室读报,就是在自习室读书。他好学深思,读书思考有所得,必记笔记,并写进日记,深得杨昌济老师的厚爱。

张昆弟,1894年出生于湖南省桃江县板溪乡一个农民家庭。1913年考入湖南第一师范,和蔡和森同在第6班。

萧子升、萧子暲兄弟则是在1911年、1912年先后从东山高等小学堂考入湖南省立第一师范。毛泽东在长沙湘乡中学时,萧子暲曾去看望过他。

毛泽东在一师期间,受杨昌济、徐特立、袁仲谦、黎锦熙、方维夏、王季范、王正枢等教员的影响非常大。除了王季范的家,杨昌济、徐特立、黎锦熙、方维夏这几位老师的住处,也是毛泽东常去求学问教的地方。

前面已经说过,王季范是毛泽东的表哥。他1884年出生于一个小康家庭,父亲王文生曾在东北当过小官吏,母亲文氏乃文芝仪的次女,排行第六,是毛泽东母亲的同胞姐姐、毛泽东的二姨妈。王季范在家族同辈中排行第九,所以毛泽东一直叫他九哥。王季范从小熟读经书,打下了良好的古典文学基础。他在长沙优级师范毕业后被聘到第一师范任教。毛泽东来到一师,他们既是亲表兄弟,又是师生关系。王季范对表弟不仅在经济上常常给予帮助,而且还在学业上认真指导,他先后辅导表弟阅读了《楚辞》、《昭明文选》、《韩昌黎全集》、《资治通鉴》以及《曾国藩家书》等经典书籍,为表弟不断提高古典文学素养打下了基础。

黎锦熙,字劭西,亦作邵西,号鹏庵,1890年出生于湖南省湘潭县晓霞山下一个人称“长塘黎氏”的书香世家。他曾师从于齐白石。1906年,黎锦熙在长沙组织了“德育会”。辛亥革命前,他参加了同盟会,辛亥革命后,在长沙办过《湖南公报》、《公言》等报刊。因此,湘省人士中曾把他与齐白石、张平子并称为“三英”,有“湘潭之东,秀毓钟灵,周围十里,代出三英。白石艺术,邵西语文;平子(湖南《大公报》社长张平子)办报,《大公》风行”之说。

毛泽东和黎锦熙常在一起谈论历史,臧否古今人物。二人真诚相交,情同挚友,形成了一种介乎师友之间的友谊。

在第一师范这一批教师中,杨昌济对毛泽东的影响最深。杨先生的口才并不是很好,但他人品高尚,学识渊博,又比较注重实际,所以他受到了毛泽东和其他学生的普遍尊重。毛泽东和蔡和森等经常一起到天鹅塘杨宅向杨昌济先生请教各种问题。杨昌济打心眼里喜欢这些好学的青年,特别是毛泽东这位农民出身的学生,两代人之间几乎没有什么隔阂,纵论天下大事,自由自在,无拘无束,其乐融融。

徐特立对毛泽东的影响也非常重要。

徐特立,一名懋恂,又名立华,字师陶,1877年2月1日出生于湖南长沙县五美乡一个贫农家庭。他只读过6年私塾,又读了4个月的“宁乡师范”,尔后在长沙城东创办了“梨江学校”,后应好友朱剑凡邀请,到长沙周南女校教书,1910年赴日本考察教育。辛亥革命后,他当选为湖南省临时议会副议长,不久辞职,在非常困难的条件下,在长沙创办了两个高小、一个初小、一个男子师范学校、一个女子师范学校,被誉为教育界的“长沙王”,而有些人则讥笑他傻,说他是“徐二镥锅”。后来从1913年10月到1919年8月,徐特立一直在一师任教。

此时,在长沙的中等学校教师们,多数都还保持着绅士派头,冬裘夏绸,衣冠楚楚,到学校去上课还要坐轿子;而徐特立却总是夹着讲义,从荷花池到书院坪,徒步往返10余里,就是遇到风风雨雨,他也从不以轿代步。徐先生最初给毛泽东所在的8班上课时,就以他创办长沙师范的经历作为艰苦奋斗的例证。下课后,同学们你一言我一语地议论起来,毛泽东说:

“徐先生办长沙师范,不顾利害,不怕困难,牺牲自己的一切,干别人不敢干的事情,这是那些自命聪明善于计算的人所不肯做的,所以笑他傻。徐先生常常把方便让给别人,把困难担在自己肩上,惯于摆烂摊子,在没有一个钱的情况下,居然能创办出一个规模不小的师范学校,这真有镥锅的精神!这种对他的讥笑,实际上是对他的赞扬。”

有一次下课后,徐特立回到休息室看书,毛泽东走了进来,问道:

“徐先生,要想知识渊博,就得博览群书,要博览群书,主要是一个快字,而快了,又不能吃透内容,理解不了,等于白读。有什么方法才能快读书,读好书呢?您读书的经验可以谈一些出来让我们仿效吗?”

徐特立亲切地说:

“润之,我认为读书要守一个‘少’字诀,不怕书看得少,但必须看懂看透。要通过自己的思考,来估量书籍的价值,要用笔标记书中的要点,要在书眉上写出自己的意见和感想,要用一个本子摘录书中精彩的地方。总之,我是坚持不动笔墨不看书的。这样读书的结果,虽然进程慢一点,但读一句算一句,读一本算一本,不但能记得牢固,而且懂得透彻,可以达到学以致用的目的,效果自然比贪多图快好。”

毛泽东连连点头称是,把徐先生的话牢牢记在心上,以至于影响了他一生。而他与国文教员袁仲谦的师生关系就有些不同了。

袁仲谦,又名袁吉六,人称袁大胡子。袁吉六先生是湖南保靖县人,清末考中进士,写得一手好字,做得一手好古文,在第一师范教书多年,也连续教了几年毛泽东所在班的国文。

袁吉六先生教风甚严,他要求学生的作文必须写桐城派的古文。毛泽东原来极推崇梁启超的文体,因先生反对梁启超的半文半白的文体,他这才不得不改变文风,改作古文,为此就需要熟读韩愈的文章。他从旧书店里买来了一套《韩昌黎全集》,可这套书纸张既不好,错字又很多,他不得不又借来了一个好的版本,一一校正,将书补充好。

毛泽东极称赞袁吉六先生提倡的“四多”,即多谈、多写、多想、多问,还赞赏他“文章妙来无过熟”的读书方法。袁先生也最喜欢他这个学生,夸奖毛泽东的文章“大有孔融之意”,周末还找他到自己的书房里谈话。

毛泽东起初还比较听袁先生的话,后来却对他守旧而又专制的作风越来越反感了。有一次刚开学,袁先生在教室里监督学生们做作文,毛泽东在题目下面写了一句:“某年某月某日第一次作文”。袁先生看见了,极不赞成,他说,我没有要你写这句话就不要写。他命毛泽东重抄一页,催了两次,毛泽东都没有理会。袁先生气冲冲地跑过来,一把将那一页撕掉了。毛泽东也生气了,站起身质问先生,并要同他一起到校长那里去讲理。袁先生无言以对。毛泽东重写了一遍,仍然坚持保留了那一句话。

尽管如此,由于毛泽东的古文写得好,袁吉六先生对他还是很器重的。他认为毛泽东是一个有才华、有胆识的“特殊学生”,所以还是照常辅导毛泽东做作文,并借书给这位特殊学生阅读。建国后,袁家还保留着毛泽东当年借书的两张便条。

毛泽东学习十分认真、刻苦,他听课时做有大量的课堂笔记,课后自修时写有读书录,还全文抄录过一些他喜爱的书籍。这些课堂笔记、读书笔记累积下来有几网篮之多。后来,他把这些笔记和抄写的书籍都送回了韶山。1929年国民党军队到韶山要抄毛泽东的家,附近的族人听到风声后,立即将这些本子和书籍搬到后山焚毁。一位曾经教过毛泽东的私塾老先生,从火堆中抢出了一个笔记本和两册教科书,并保存了下来。现存的《讲堂录》就是他抢出来的那本笔记,成为现在唯一可以看到的毛泽东学生时代的课堂笔记了。

《讲堂录》共47页,后36页主要是听课笔记,也有一些读书札记,内容涉及哲学、史地、古文、数理等;其中对古今名人治学、处世、治国和有关伦理道德的言论记录比较多一些。如:

“士要转移世风,当重两义;曰厚曰实。厚者勿忌人;实则不说大话,不好虚名,不行驾空之事,不谈过高之理。”

“少年须有朝气,否则暮气中之。暮气之来,乘疏懈之隙也,故曰怠惰者,生之坟墓。”

“闭门求学,其学无用。欲从天下家国万事万物而学之,则汗漫九垓,遍游四宇尚已。”

“惟明而后可断,既明而断矣,事未有不成者。”

“高尚其理想,其后一言一动皆期合此理想。”“理想者,事实之母也。”

 “为学之道不得不重现在。”“不为浮誉所惑,则所以养其力者厚;不为流欲相竞,则所以制其气者重。”

毛泽东对于他喜欢的课程,如撰写文学或伦理主题的文章和社会科学课程,学得津津有味,而且有独到见解,常常得100分。可他对于那些不喜欢的课程,如静物写生和自然科学方面的课程,却很少去光顾,所以经常得零分或接近零分。当然,完全放弃枯燥的静物写生,也不是毛泽东的做事风格,因此在写生课堂上不得不硬着头皮应付一下。一次绘画考试,他在试卷上潦草地画了一个椭圆,题名为“鸡蛋”,就交了卷。还有一次,他在上课时一个简洁的构思使他得以提前离开了教室。他在图画纸上先画了一条水平线,接着在水平线上画了个半圆,又将李白的一句诗题写在上面:“半壁见海日”。

这幅题名为“半壁见海日”的简笔画,虽然是应景之作,却大有深意——象征着“光辉和希望”的半个太阳,跃出了地平线!生动地展现了一位横空出世的思想家对未来的热切期盼。

毛泽东在生活上也有一些鲜为人知的小事。有一次,他读书读到深夜,由于被子离油灯太近,就引发了一场小火灾,烧坏了几张床铺。还有一次,一位同学因父母为他包办婚姻而苦恼,毛泽东深表同情,就到这位同学家里劝说他的父母放弃这种安排,不用说他是一位不受欢迎的人。

毛泽东还喜欢和友人一起游览山河、谈诗论文,有时也要联联句作作诗。1914年深秋的一天傍晚,他和萧瑜也就是萧子升一起到湘江畔散步,共同欣赏黄昏时分的风景。萧瑜突然来了诗兴,首先吟道:

晚霭峰间起,归人江上行。 云流千里远,

毛泽东也不甘落后,随口吟出:

人对一帆轻。落日荒林暗,

萧瑜接吟:

寒钟古寺生。深林归倦鸟,

毛泽东就来了一个结束语:

高阁倚佳人。

在1915年春节前夕,第一师范放寒假了,毛泽东这才回到了故乡韶山。一天傍晚,他满面笑容地问菊妹子:

“你认得字吗?”

这个叫菊妹子的小姑娘,是母亲收养的一个干女儿。菊妹子摇摇头,不解地说:

“三哥,大家都讲,女人家读书有啥子用?”

“那你取了大名没有?”毛泽东又问。菊妹子不好意思地摇摇头。毛泽东思索了一会儿,在桌子上写下了“毛泽建”3个字,说:

“菊妹子,今后,你就叫‘毛泽建’吧!”

菊妹子瞪大眼睛,好奇地望着三哥。

“‘毛’是我们的姓,‘泽’是排辈,‘建’是你的名,是设立和建筑的意思。”毛泽东耐心地解释着,然后又语重心长地说:“你可要为穷苦人建功立业啊!”

“毛泽建,多好听的名字呀!”

菊妹子从心底里感激三哥为她取了个正式的名字,可她毕竟还只是一个不到10岁的小姑娘,对三哥讲的道理似懂非懂,只是瞪大眼睛点了点头,然后兴冲冲地跑了出去,口里喊道:

“我有大名了,以后我就叫毛泽建了!”

这天夜里,毛泽建兴奋得在床上翻来覆去地睡不着觉,她想起了自己的身世。

毛泽建是毛泽东堂叔毛尉生的女儿,1905年10月出生于韶山冲东茅塘,比毛泽覃小将近一个月。因为她是在金风送爽菊花飘香的季节里降临人世的,所以父母就给她取了个“菊妹子”的乳名。毛尉生年轻时,因为劳累过度得了肺病,经常咳血;菊妹子的母亲毛陈氏也有眼疾,一家人的生活贫困不堪。后来毛尉生39岁就去世了。1912年韶山冲发大水,庄稼颗粒无收,毛陈氏只好带着仅有7岁的菊妹子和菊妹子的3个小弟弟,沿街乞讨。善良的文七妹看到他们一家如此艰辛,就把菊妹子从东茅塘接到上屋场,作为自己的女儿收养了。毛顺生和文七妹把菊妹子视作亲生一般,疼爱有加。毛泽东兄弟3人也把这个苦命的堂妹当作亲妹妹一样看待。

且说这天夜里,毛泽建心里一直在念着“毛泽建”三个字,还用手指一遍又一遍地模仿着三哥写的这3个字。第二天一早,毛泽东见了她,笑着问:

“菊妹子,你的名字还记得不?”

“当然记得了,我叫毛泽建。”

“哦,不错。”

“三哥,我还会写了呢。” 

菊妹子兴奋地说。

“是吗?那你写给我看看。”

毛泽东颇感意外,只见菊妹子拿起一个小木棍,在地上歪歪斜斜地写出了“毛泽建”3个字,居然是一字不差。毛泽东高兴地笑了,说:

“菊妹子真有出息,以后我们就叫你毛泽建了。”

1915年的春节很快就过去了,2月24日是正月十一日,毛泽东再次来到外婆家,看望了七舅父、八舅父,顺便归还了借文运昌表哥的书,这是他在手里放了整整5年已经读过无数遍的《盛世危言》、《新民丛报》及其它书籍,因文运昌不在家,他便留下了一张“还书便条”:

咏昌先生:

书11本,内《盛世危言》失布匣,《新民丛报》损去首页,抱歉之至,尚希原谅。

泽东 敬白

正月十一日

又国文教科2本,信一封。

眼看就要开学了,毛泽东离开韶山又回到了湖南一师,投入到紧张的学习之中。就是从这个时候开始,杨昌济先生倡导同事黎锦熙和学生毛泽东、蔡和森、萧子暲、陈昌、罗学瓒等人在自己的住处——浏正街李氏芋园——组织了一个哲学研究小组,他们每逢星期六或星期日,就在杨先生家中由两位先生轮流介绍和推荐读物,然后大家一起讨论有关读书心得和哲学,探讨人生哲理,纵论天下大势,畅谈相契政见,谋事今天,计虑将来,就这样,他们之间形成了一个以杨昌济为核心的思想学术团体。这个研究小组的活动一直坚持到1915年9月间。萧子暲曾经回忆说:

“杨先生并不善于辞令,也不装腔作势,但他能得到听讲者很大的注意与尊敬,大家都佩服他的道德学问。毛泽东、蔡和森、陈昌等每逢星期日,便相率到杨先生家里去讲学问道,杨先生也很器重毛、蔡、陈等学生。”

杨昌济通过与这些学生的接触,对他们的家庭出身、个人经历和品德学问都十分了解。

1915年春,毛泽东的同窗好友廖廷璇和皮述莲结婚,毛泽东为他们撰写了一副贺联:

二月梅香清友;春风桃灼佳人。

“二月(农历)”、“春风”指廖廷璇、皮述莲结婚的时间,“梅”、“桃”分别形容他们的品貌,联语表达了对学友结成连理的赞美和祝愿。

1915年4月5日,毛泽东曾和杨昌济先生谈起了自己的家世和经历。杨昌济问道:

“韶山,就是南岳72峰之一呀,距湘乡不远吧?”

毛泽东回答说:

“对,我的家离湘乡一山之隔,我外婆家就在湘乡。”

杨昌济微笑着说:

“难怪你的语音里带有湘乡口音。”

“是的。”

毛泽东笑着点了点头,又说:

“我们那里语言差别很大,当地有句俗话叫作:‘隔山不懂话,隔江难辨音’。”

这一天,杨昌济先生在日记中作了这样的记载:

“毛生泽东,其言所居之地为湘潭与湘乡连界之地,仅隔一山,而两地之语言各异。其地在高山之中,聚族而居,人多务农,易于致富,富则往湘乡买田。风俗纯朴,烟赌甚稀。渠之父先亦务农,现业转贩;其弟亦务农;其外家为湘乡人,亦农家也。然资质俊秀若此,殊为难得。余因以农家多出异材,引曾涤生、梁任公之例以勉之。毛生曾务农2年,民国反正时又曾当兵半年,亦有趣味之履历也。”

黎锦熙先生也在日记中这样记载了他和毛泽东的交往:“4月4日星期日,上午润之来,阅其日记,告以读书方法。”“4月11日星期日,上午萧子升、润之及焜甫至,讲读书法。”“4月18日星期日,上午润之等相继至,共话宏文书社事。”“4月25日星期日,上午游园,润之来,告以在校研究科学之术。”

毛泽东在学习上喜欢广泛而虚心地向他人请教,共同探讨问题,他认为,“学问”两个字组成一词是很有道理的,既要虚心好学,独立思考,又要好问,与人交谈讨论,只有这样才能真正取得学问。

为了认识真理和追求真理,毛泽东决心学习“颜子之箪瓢与范公之划粥”的精神,“将全副功夫,向大本大源探讨。”他提出“文明其精神”,要刻苦学习,不断充实自己。他与同学约定三不谈,即不谈金钱,不谈男女之事,不谈家务琐事。只在一起谈论大事,即“人的天性,人类社会,中国,世界,宇宙”。

1936年,毛泽东对斯诺说,年轻时“没有时间谈情说爱,认为时局危急,求知的需要迫切,不允许去谈论女人或私人问题。”“我对女人不感兴趣。”

毛泽东的宏大志向深深地感染着他周围的好友们,罗学瓒为此曾作诗一首,表明心志:

开怀天下事,不谈家与身。登高翘首望,万物杂然陈。光芒垂万丈,

何畏鬼妖精。奋我匣中剑,斩此妖孽根。立志在匡时,欲为国之英。

5月7日下午3时,日本政府向袁世凯发出最后通牒,在5月9日下午6时之前的48小时内,答复他们早在1915年1月18日提出的“二十一条”要求。

原来在1915年5月间,第一次世界大战中的两大军事集团“同盟国”和“协约国”为重新瓜分世界,争夺殖民地,正打得不可开交,日本帝国主义就趁着西方列强无暇东顾之机,加紧了对中国的侵略。这个“二十一条”要求,是日本帝国主义趁袁世凯想当皇帝之机,向他抛出的诱饵,以解决中日“悬案”为名,提出的旨在独霸中国、灭亡中国的秘密条约。

袁世凯为了换取日本帝国主义对其复辟帝制的支持,在5月9日答复中,除了“二十一条”第五号(7条):中国政府必须聘请日本人做政治、军事、财政顾问,中国警政和兵工厂由中日合办,日本有在武昌和九江、南昌间,南昌和杭州间,南昌和潮州间的筑路权,有在福建省内进行铁路、矿山等投资的优先权这些要求“容日后协商”外,其它的要求全部接受了。

袁世凯这一卖国的消息一经传到湖南,湖南一师的师生们无不义愤填膺,他们为了揭露袁世凯接受“二十一条”的修正案,集资编印了有关日本帝国主义侵略中国的几篇文章和资料,题目叫《明耻篇》。毛泽东仔细阅读了这些文章和资料。他在封面上写道:

“五月七日,民国奇耻。何以报仇?在我学子!”

他又在《明耻篇》卷首《感言》篇里批注道:

“此文为第一师范学校教习石润山先生作。先生名广权,宝庆人。当中日交涉之倾,举校愤激,先生尤痛慨,至辍寝忘食,同学等爰集资刊印此篇,先生则为序其端而编次之,云云。”

他还在《明耻篇》那些文章和资料中的许多地方加了圈点和着重号,并作了批注,在不少地方写着:“此文作得好!”“说得痛快!”等等。

后来在全国人民的强烈反对下,日本帝国主义的阴谋最终未能得逞,袁世凯的皇帝梦也破产了。

且说5月23日,湖南一师校长张干、学监王季范、教员杨昌济等人倡导并主持了一个为病逝学生召开的追悼会。这个学生就是易昌陶。

易昌陶又名易永畦,湖南衡阳人。他品学兼优,是毛泽东的同班同学和挚友,此前在3月份病逝于衡山县家中。

在这次追悼会上,杨昌济先生为易昌陶题写了一副挽联:

\begin{couplet}
遗书箧满,铁笔痕留。积瘁损年华,深悲未遂平生志;

湖水长流,岳云依旧。英灵怀故国,没世宁灰壮士心。
\end{couplet}

毛泽东痛失交往密切的良友,又感到祖国命运艰难,生者责任重大,他悲痛交织,也提笔为好友写了一副挽联:

\begin{couplet}
胡虏多反复,千里度龙山,腥秽待湔,独令我来何济世;

生死安足论,百年会有殁,奇花初茁,特因君去尚非时。
\end{couplet}

他还写了一首五言古风长诗,悼念亡友:

\begin{couplet}
去去思君深,思君君不来。愁煞芳年友,悲叹有余哀。

衡阳雁声彻,湘滨春溜回。感物念所欢,踯躅南城隈。

城隈草萋萋,涔泪侵双腮。采采余孤景,日落衡云西。

方期沆瀁游,零落匪所思。永诀从今始,午夜惊鸣鸡。

鸣鸡一声唱,汗漫东皋上。冉冉望君来,握手珠眶涨。

关山蹇骥足,飞飙拂灵帐。我怀郁如焚,放歌倚列嶂。

列嶂青且茜,愿言试长剑。东海有岛夷,北山尽仇怨。

荡涤谁氏子,安得辞浮贱。子期竟早亡,牙琴从此绝。

琴绝最伤情,朱华春不荣。后来有千日,谁与共平生?

望灵荐杯酒,惨淡看铭旌。惆怅中何寄,江天水一泓。
\end{couplet}



毛泽东的悲怆和惋惜跃然纸上,这挽联,这悼亡诗,无不淋漓尽致地抒发了他对亡友的怀念和爱国之情,可谓是文理俱佳,真挚动人,令人垂泪,催人上进。

后来,这些挽联和挽诗均被收录在一师的《易君永畦追悼录》中。

从1915年夏季开始,毛泽东参与组织了第一师范进步师生开展反日、反袁斗争。

当袁世凯复辟帝制之声甚嚣尘上时,毛泽东团结进步师生,公开进行反袁演说,写文章,和帝制派劝进的丑恶行为进行针锋相对的斗争。

杨昌济、徐特立等先生也在毛泽东的影响下,参加了反袁斗争。先生们一起写信,斥责了一位鼓吹帝制的教师,扫除了在校内宣传帝制的邪气。

后来张勋复辟,军阀割据一方,混战连绵不断;在思想文化领域,也掀起了一场尊孔复古的逆流。专制与共和、复辟与反复辟、民主与法制、尊孔与反孔新旧两派之间的斗争日益激烈。  

在这样一个新的方生、旧的未死的充满矛盾的年代里,毛泽东怀着一颗强烈的爱国心,时刻关注着中国和世界局势的发展变化,思索着中华民族的前途和命运。他主要是依靠报刊杂志了解国际和国内形势的发展变化。一师有一个可容纳几十人的阅览室,那里是他天天晚上都要去的地方,那里面有湖南、北京、上海等地出版的报刊。他在阅览室里全神贯注地翻阅长沙和上海的报纸,直至深夜。如果有人要找他,别人就会说:“可能在报纸阅览室里。”

同学们都喜欢听毛泽东讲一周的动荡局势和第一次世界大战的最新动向,一时间,他成了一师校园内闻名的“时事通”。

面对复杂多变的局势,同学们经常议论纷纷,对一些问题也常常困惑不解,而毛泽东则看得清楚,而且能讲清楚事情的来龙去脉。萧子暲说:

“我也是天天看报,每次到阅览室,差不多都会遇见毛泽东也正在那里看报。有一个星期天,我在街上同他相遇。返校途中,我们边走边谈,他给我详细分析了奥国皇太子为什么在塞尔维亚被刺杀;德国皇帝威廉二世为什么出兵;德俄、德法、德英为什么宣战;凡尔登如何难攻;英法如何联盟;美国如何乘机大发横财;日本又如何趁火打劫,提出灭亡中国的“二十一条”。他讲的有时间、有地点、有根有据。使我听了又钦佩,又惭愧。”

毛泽东除了在阅览室阅读以外,还要自己花钱买报刊杂志,他曾回忆说:

“我在长沙师范学校的几年,总共只花了160块钱——其中包括我的许多次报名费。在这笔钱里,大概有三分之一花在报纸上,订阅费每月约1元。我还常常买报摊上的书籍和杂志。我父亲骂我浪费。他说这是把钱挥霍在废纸上。可是我养成了读报的习惯,从1911年到1927年我上井冈山时为止,我从没有中断过阅读北京、上海和湖南的日报。”

毛泽东读报刊非常认真,常常随身带着中国和世界地图、字典和笔记本。凡是不熟悉的地名,他就对着地图,找出位置;凡是重要的消息、文章、资料,不论长短,总是从头到尾认真读完,并做摘记,写出心得。 

毛泽东常常把自己订的报纸中的重要资料,剪辑成册。他有时还在报纸的空白边上,写出报纸中提到的城市、港口、山岳、江河等地理名词,并写出英文。他说,这样既了解了时事,又熟悉了地理,还学习了英文,是一举三得。

且说毛泽东在不断接触进步思想,积极参加国内革命运动和救亡活动的同时,也因为一些小事惹出了麻烦。

1915年春期末,湖南省议会发布了一项公告:决定从下学期开始,师范类学校的每个学生须交纳10元学杂费。第一师范校长张干自然是忠于当局的,他对这一决定表示坚决拥护。张干的态度首先遭到了家境贫寒和那些得不到家庭接济的学生们的强烈反对。校园里纷纷传言,这个“增费”的决定是张干为了讨好当局向省议会提出的建议。于是,学生们纷纷罢课,在校内掀起了一场声势浩大的“驱张运动”。

张干其人,原是数学教师,为人精明能干,言辞练达,很有社会活动能力,且善于与上司结交,不到30岁就当上了校长。

一天傍晚,毛泽东和萧子暲看到同学们在布告栏里贴了不少传单,揭发张干“不忠、不孝、不仁、不悌”等“劣绩”。毛泽东说:

“子暲,我看这些传单没有击中张干的要害。”

“为什么?”

“因为我们现在并非反对张干当我们的家长,而是反对张干当我们的校长。”

“依你之见呢?”

“要赶走这位校长,就要制造舆论,说他如何未将我们一师办好。走!到君子亭去。”

两人来到君子亭,毛泽东拿出纸和笔,不一会儿就草成了一份《驱张宣言》。萧子暲拿过来一看,只见上面写的是:“张干自到我们一师任校长以来,对上逢迎,对下专横,办学无方,贻误青年……”,他连连称好。毛泽东说:

“马上派人进城去印,今晚一定要印好。”

第二天清晨,《驱张宣言》已经撒遍了一师校园,整个学校沸腾起来了,“驱除张干”的口号声此起彼伏。此事惊动了湖南省教育厅,很快就来了一名督学。督学让张干把学生们集中起来训话,他说:

“你们不要再胡闹了,立即给我复课!”

毛泽东让同学传递一个纸条给台上的督学。那督学接过纸条,见上面写着:“张干一日不离校,我们一日不上课!请督学马上给我们答复!”督学这才知道这些学生是不好对付的,他怕事情再闹大,就换了一种口气说:

“诸位,这个学期很快就要结束了,你们还是上课吧,下学期张干不来了。”

督学走后,袁吉六与同事们评价毛泽东说:

“挽天下危亡者,必斯人也。”

一位学监却向张干告密说,《驱张宣言》是毛泽东带头写的。张干大怒道:

“反了!在学校只有校长开除学生,学生要开除老师这还是第一次。”

他下令彻查带头闹事的学生,挂牌开除以毛泽东为首的17名主要闹事者。身为学监和数学教员的王季范非常着急,他找到杨昌济、徐特立、方维夏、袁吉六、符定一、王正枢等教员商量对策,又一同出面召集了一个全校教职员会议,为学生们鸣不平,对张干施加压力。这样一来,张干不得不收回成命,对毛泽东等改为记大过处分。

此后,杨昌济辞去了一师的教学,整整一年没有到校上课。

欲知毛泽东后来在一师的情况如何,请接着往下看。

东方翁曰:毛泽东不仅以德报恩,而且也曾以德报怨。前面所说的那位在湖南一师要开除他学籍的校长张干,在新中国建立后亦被邀至京城,待以座上宾,便是一例。此事载于本传第六卷,诸君不可不读。古人云:“受人滴水之恩,当以涌泉相报。”此乃人之常情,世之常理;而以德报怨者,在历朝历代几乎是不可多见的!这种流芳百代之举,自然是那些“得志便猖狂”的“中山狼”一类人物不屑为之的。那些曾经受教受惠于毛泽东却在毛泽东身后肆无忌惮地往老人家身上泼脏水的人,便属于此类。

\end{document}
