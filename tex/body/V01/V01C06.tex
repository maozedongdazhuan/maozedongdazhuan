\documentclass[../../dazhuan.tex]{subfiles}
% 第一卷
\begin{document}
\chapter*{第六章}
\pdfbookmark{第六章}{V01C06}
\begin{pref}
	日人诚我国劲敌。二十年内,非一战不足以图存!	
国人犹沉酣未觉,注意东事少。
\end{pref}

话说1915年6月25日,毛泽东在致湘生的回信中写道;

“读君诗,调高意厚,非我所能。同学易昌陶君已病死。昌陶君工书善文,与弟甚厚,死殊可惜。校中追悼,吾挽以诗,乞为斧正。”

此前,毛泽东得到了长沙高等师范的一封招生函件,得知该校设文史两种,而且是“重自习,不数上讲堂。”于是他便准备从一师退学,报考高等师范,所以他在这封信中就谈到了关于学习方面的认识和转校的想法:

“今闻于师友,且齿已长,而识稍进。于是决定为学之道,先博而后约,先中而后西,先普通而后专门。从前拿错主意,为学无头绪,而于学堂科学,尤厌其繁碎。学校分数奖励之虚荣,尤其鄙弃,今乃知其不是。吾今日舍治科学,求分数,尚有何事?适得高等师范信,下期设招文史两科,皆为矫近时学绝道丧之弊。其制大要与书院相似,重自习,不数上讲堂,真研古好处也。”

毛泽东十分赞赏康有为、梁启超的治学方法,他说:

“康尝言:‘吾四十岁以前,学遍中国学问;四十岁以后,又吸收西国学问之精华’。梁固早慧,观其自述,亦是先业词章,后治各科。盖文学为百学之源,吾前言诗赋无用,实失言也。足下有志于此乎?来日之中国,艰难百倍于昔,非有奇杰不足言救济,足下幸无暴弃。”

“屠沽贾衒之中,必有非常之人……人非圣贤,不能孑然有所成就,亲师而外,取友为急。”

这时候,眼看就要放暑假了,毛泽东因为手头拮据,就没有回韶山,住入李氏芋园,在杨昌济、黎锦熙等先生的指导下进行自学。

随着时局的发展和对社会问题的认真思索,毛泽东逐渐感觉到,要实现救国救民的愿望,就需要更广泛地结交有志救国青年,联合更多的同志。

6月末的一天,毛泽东来到了校园后山的君子亭上。这是一座四柱八角琉璃瓦小凉亭,四边都设有坐凳。他坐在凳子上起草了一个约二三百字的《征友启事》,末尾借用《诗经》里的“嘤其鸣矣,求其友声”说:“愿嘤鸣以求友,敢步将伯之呼。”启事中还特别强调,要结交对救国感兴趣的青年,特别是能刻苦耐劳、意志坚定、随时准备为国捐躯的青年。最后的署名以“毛泽东”3个字的繁体28画,署为“二十八画生”。

毛泽东回到住处,用蜡版刻好油印好,分装在各个信封里,在信封上注明:“请张贴在大家看得见的地方”,又写上“来信由第一师范附属小学陈章甫转交”;尔后向长沙各主要学校都发了一份。

原来此时的陈昌已经从一师毕业了,被聘请在一师附属小学任教。毛泽东以天下为己任的远大抱负和无所畏惧的进取精神,博览群书、独立思考的治学态度,和他那超凡的胆略和才华,使陈昌无限钦佩。他们二人志同道合,朝夕相处,交往甚密,是非常要好的朋友,因之毛泽东将这件事托付给这位密友。

且说毛泽东的《征友启事》一经贴出,还真是引起了不小的反应,一般人看了,都很难理解,认为“二十八画生”一定是个怪人;有的人甚至还认为这种征友是不怀好意的。湖南第一女子师范姓马的女校长,就认为这个启事是在找女学生谈恋爱,她按照启事上写的通信处,找到了陈章甫,还找到了一师校长那里。陈昌和一师校长告诉她说,“二十八画生”的真实姓名叫毛泽东,是个品学兼优、胸怀大志的学生,马校长这才消除了疑虑。

毛泽东后来在和斯诺的谈话中,说到他这次征友活动,最初“只得到3个半人的响应。”在这3个半人中,一个是罗章龙,另外两个后来变成了极端反动分子,而那半个人则是李立三。

罗章龙,湖南浏阳人,生于1896年,化名纵宇一郎。

李立三,原名李隆郅,曾用名李能至,李成、柏山、李明、李敏然,1899年出生于湖南省醴陵县阳三石,1915年在长沙长郡联立中学读书。

据罗章龙回忆说:“1915年5月中旬(应为农历五月中,即公历6月底;毛泽东在这一年11月9日写给黎锦熙的信中也说是在‘夏假后’,即放暑假后;二者是相吻合的——笔者注)的一天,我赴司马里第一中学访友,在该校会客室门外墙上,偶然发现署名‘二十八画生征友启事’一则。启事是用八裁湘纸油印的,古典文体,书法挺秀。启事引句为《诗经》语:‘愿嘤鸣以求友,敢步将伯之呼’。内容为求志同道合的朋友,其文情真挚,词语典丽可诵,看后颇为感动。返校后,我立作一书应之,署名纵宇一郎。我将信寄出去3天后,果然得到了毛泽东的回信,略云:接大示,空谷足音,跫然色喜(《庄子》语——笔者注),愿趋前晤教云云。”

此后,罗章龙经浏阳籍学友陈昌、陈绍休(赞周)联络约定,邀同好友李立三在下一个星期日(当在7月4日)上午到定王台湖南省立图书馆与毛泽东见面,手持报纸为记。“同学陈圣皋也欣然前往(罗章龙语——笔者注)。” 

这一天,天气晴和。毛泽东和罗章龙、李立三、陈圣皋在图书馆见了面。

罗章龙回忆说:“上午9时左右,我们到达定王台省立图书馆。”“在走廊处有一少年仪表端庄,气宇轩昂,心知即所欲晤见之人。我们乃趋前为礼,彼此互通姓名,方知少年姓毛名泽东,字润之。二十八画生乃其名字的笔画数。略谈数语后,圣皋去阅览室看书,润之建议到院内觅一僻静处倾谈。进得院内,寂静无哗,我们就坐在一长条石上。”

李立三在谈话开始后不久没有发表意见就告辞了。毛泽东和罗章龙却整整谈了3个小时,“谈话内容涉及很广,包括国内外政治、经济以至宇宙人生等等。”“谈到音韵改革问题,主张以曲韵代诗韵,以新的文学艺术代替‘高文典册’与宫廷文学。在旧文学著作中,我们对于离骚颇感兴趣,曾主张对离骚赋予新评价。关于治学问题,润之认为,对于宇宙,对于人生,对于祖国,对于教育,均属茫然!主张在学问方面用全副力量向宇宙、祖国、社会做穷源竟委的探讨,研究有得,便可解释一切。关于生活方面所涉及较少(罗章龙语——笔者注)。” 

这次谈话直到图书馆中午休息时方止,临别时,毛泽东对罗章龙说:我们谈得很好,“愿结管鲍之谊”。他还叮嘱罗章龙,以后要常见面。

第二天,罗章龙和同学黄昆吾及一师的彭道良谈到了在图书馆与毛泽东会晤一事。

彭道良,字则厚,浏阳人,与张昆弟同是湖南第一师范学校6班的学生。他听罢罗章龙绘声绘色的描述,笑道:

“昨日之事可称三奇会。”

罗章龙不解地问:

“何为三奇会?”

彭道良说:

“圣皋与兄为联中二奇,益以毛奇,岂非三奇?”

黄昆吾问起“毛奇”一名的缘由,彭道良从容解释说:

“我与二十八画生乃同学好友,颇知其为人品学兼优,且具特立独行之性格。他常语人:‘男子要为天下奇(此乃毛泽东引用宋代王庭珪送胡邦衡诗句——笔者注),即读奇书,交奇友,著奇文,创奇迹,做个奇男子。’此君可谓奇特之士,因此同学中戏称他毛奇。”

罗章龙以彭道良所言,又求证于他的同乡陈绍休(字赞周,亦陈圣皋之弟),陈绍休说:

“润之气质沉雄,确为我校一奇士,但他择友甚严,居恒鹜高远而卑流俗,有九天俯视之慨。观其所为诗文戛戛独造,言为心声,非修养有素不克臻此。直谅多闻,堪称益友。”

自此以后,罗章龙和毛泽东每到周末常常约定到天心阁相会,他们绕着旧垒城堞散步;或到城南书院、长郡中学、韩玄墓、杨昌济寓所晤谈;有时又相约去郊外云麓宫、自卑亭、水陆洲、猴子石、东南渡等处远足游览。每次晤面,二人多是检讨所思所学,析疑问难,究所未知。

且说1915年7月5日,杨昌济全家迁往湘江西岳麓山下,为了方便毛泽东等人来家里学习和讨论,他在家中专门辟了一间客房。

毛泽东在杨昌济搬家后,应数学老师王正枢之邀,来到他在长沙高正街19号的居所住了一个假期,给老师家添了不少麻烦。

从高正街19号可以直通长沙的古城墙,毛泽东时常带着王正枢的儿子王人路兄弟几人登城到天心阁观光游玩,他还曾为王家兄弟吟诵了不久前写的悼念学友易昌陶的诗,抒发自己的爱国情怀,并以此激励王氏兄弟。

毛泽东有时也到杨昌济家里看望先生,师生们在一起无所不谈,社会、政治、学术、理想、人生等等,有时一谈就是几个小时。吃饭时,大家坐在一起,不分彼此,亲如一家。

7月末,黎锦熙记载假期中他和毛泽东第8次接触的日记中写道:

“7月31日星期六,晚,在润之处观其日记,甚切实,文理优于章甫,笃行两人略同。皆可大造,宜示之以方也。”

黎锦熙所说的“章甫”就是陈昌。

在1915年8月底,黎锦熙日记中还有这样的追记:从4月4日至8月29日,毛泽东在星期六和星期日到芋园黎锦熙住处拜访求教近20次,内容包括“读书方法”、“研究科学之术”、“改造社会事”、“学与政”等等。

这正是:\begin{xemph}群贤毕集在名苑,读书论道赴洞天。\end{xemph}


胸有半壁见海日,开口便是三不谈。

再说在1915年8月间,萧子升从一师毕业后被聘到长沙楚怡学校任教,毛泽东在给他的信中说:

“人获一珠,家藏半璧,欲不互质参观,安由博征而广识哉?”“不先有言,何以知失?知失则得,非言之功乎?”

因此,他主张朋友间充分交换意见,“互质参观”,提倡说话,反对讳言,通过交流讨论取得“真知”、“真理”。在讨论中,既要虚心听取别人意见,又要坚持自己正确的观点。他说:

“夫人之生所遭不齐,惟豪杰之士知殊趋而同至。不强人以合吾之轨,亦不迁己轨以合人之型,以诚至公彻理之谈也。”

8月间,毛泽东还在给萧子升的信中抄录了他最近写的一篇名曰《自讼》的日记:

“一伎粗伸,即欲献于人也,一善未达,即欲号于众也,招朋引类,耸袂轩眉,无静澹之容,有浮嚣之气,姝姝自悦,曾不知耻,虽强其外,实干其中,名利不毁,耆欲日深,道听途说,搅神丧日,而自以为欣。日学牡丹之所为,将无实之可望,……牡丹先盛而后衰,匏瓜先衰而后盛,一者无终,一者有卒,有卒是取,其匏瓜乎?”

1915年9月,黎锦熙已经调到北京教育部任编纂处编纂员,毛泽东与先生仍时有书信往来。

9月6日,毛泽东在给萧子升的信中,说到了黎锦熙,他写道:

“闻黎君邵西好学,乃往询之,其言若合,而条理加详密焉,入手之法,又甚备而完。吾于黎君,感之最深,盖自有生至今,能如是道者,一焉而已。”“仆问邵西,学乌乎求?学校浊败,舍之以就深山幽泉,读古坟籍,以建基础,效康氏、梁任公之所为,然后下山而涉其新。”

毛泽东在信中又说,黎锦熙在回信中批评他这种先古后新的次序是“先后倒置”,黎锦熙说:“盖通为专之基,新为旧之基,若政家、事功家之学,尤贵肆应曲当。”

毛泽东还说,他认识到不能舍通识而专攻,乃“系其心于学校,惟通识之是求”。但他仍然认为:“吾人所最急者,国学常识也。”

1915年9月15日,陈独秀主编的《新青年》杂志在上海创刊,标志着新文化运动的开始。

杨昌济自己出钱订阅了几份《新青年》,除了自己阅读,还分赠给他的得意门生毛泽东等人。

《新青年》高举民主与科学的大旗,唤起了人们强烈的爱国主义精神,点燃了追求真理的火焰,在知识界和青年学生中引起了强烈的反响。一向对新鲜事物极为敏感和不断追求真理的毛泽东,一开始就认真地阅读它,并热心地宣传介绍它。在他的思想意识里,《新青年》取代了《新民丛报》;陈独秀和胡适取代了康有为与梁启超,成为他心目中崇拜的偶像。

毛泽东曾经说过:“我在师范学校学习的时候,就开始读这个杂志了。我非常钦佩胡适和陈独秀的文章。他们代替了已经被我抛弃的梁启超和康有为,一时成了我的楷模。”

毛泽东还曾经说过陈独秀对他的影响,他说:“他对我的影响,也许超过其他任何人。”

陈独秀,谱名同庆,学名乾生,字仲甫,1879年出生于安徽省怀宁县。

9月27日,毛泽东致信萧子升说:

“此日如金,甚可爱惜!仆自克之力甚薄,欲借外界以为策励,故求友之心甚热。如足下,诚能策励我者也。”

“近以友不博则见不广,少年学问寡成,壮岁事功难立,乃发内宣,仆无他长处,惟守‘善与人同’、‘取人为善’二语。故已有得,未尝敢不告示与人;有人善,虽千里吾求之。所以,效嘤鸣而求其友声。至今数日,应者尚寡。兹附上一纸,贵校有贤者,可为介绍。”

毛泽东也经常到楚怡学校萧子升处,通过萧子升认识了曾经先后两次同校同学而不相识的何叔衡。

何叔衡,字玉衡,号琥璜,学名瞻岵,1876年出生于湖南宁乡县杓子冲,成年后考中晚清最后一批秀才。1913年春,37岁的何叔衡考入湖南省立第四师范,与比他小17岁的毛泽东同校,后来他们又一起转入一师。1914年7月,何叔衡在一师只读了半年就退学了,受聘于长沙楚怡学校任主任教员,教高年级国文课。

毛泽东认识何叔衡后,没多少时间便对他产生了敬意,特别为他的办事热情、感情热烈所吸引。毛泽东常说“何胡子”是一头牛,是“感情一堆”。何叔衡对毛泽东的学识也非常钦佩,常向人介绍,“毛润之是个了不起的人物”,“是后起之秀”。毛泽东曾当面评价何叔衡:“不能谋则能断”。何叔衡心悦诚服地说:“润之说我不能谋则能断,这话是道着了。”

1915年11月9日,毛泽东给在北京的黎锦熙写了一封信,他写道:
\begin{xquote}

邵西仁兄足下: 

前月从熊君传来足下一书,教诲良多。兹有欲为足下言者:方今恶声日高,正义蒙塞,士人丁此大厄,正当龙潜不见,以待有为,不可急图进取。如足下之事,乃至崇之业。然彼方以术愚人,今反进以智人之术,其可合邪?收揽名士政策,日起日巧,有自欲用天下之志者,乃反为人所用欤!元凯臣舜,服善也;扬刘臣莽,附势也。辨夫今之为舜欤抑莽欤者,则所以自处明矣!北京如冶炉,所过必化。弟闻人言,辄用心悸。来书言速归讲学,并言北京臭腐,不可久居,至今不见征轺之返;又闻将有所为,于此久居不去。窃大惑不可解,故不敢不言,望察焉,急归无恋也。

弟在学校,依兄所教言,孳孶不敢叛。然性不好束缚,终见此非读书之地,意志不自由,程度太低,俦侣太恶,有用之身,宝贵之时日,逐渐催落,以衰以逝,心中实大悲伤。昔朱子谓:“不能使船者嫌溪曲。”弟诚不能为古人所为,宜为其所讥,然亦有“幽谷乔木”之训。如此等学校者,直下下之幽谷也。必欲弃去,就良图,立远志,渴望兄归,一商筹之。生平不见良师友,得吾兄恨晚,甚愿日日趋前请教。两年以来,求友之心甚炽,夏假后,乃作一启事,张之各校,应者亦五六人。近日心情稍快惟此耳。岁将晏,气候日寒,起居注意,道路珍摄。不复一一。

\sign{润之\quad 弟毛泽东顿首}

\end{xquote}

正所谓功夫不负有心人,经过毛泽东长期访求和多方通信联系,在他的周围终于逐渐聚集了一批追求进步的青年,他们多数是第一师范和各中等学校的学生,也有少数是长沙市中小学的青年教师。在这一批青年中,除了前边已经说到的蔡和森、何叔衡、萧子升、张昆弟、罗学瓒、陈昌、陈书农(别名陈启民)、罗章龙等十余人外,还有一个人不能不提到,他就是邓中夏。

原来,蔡和森在1915年秋考入了湖南高等师范,结识了邓中夏。

邓中夏,1894年出生于湖南宜章,1915年考入湖南高等师范文史专修科,与蔡和森同班。蔡和森沉默寡言,终日伏案用功;邓中夏则性情豪爽,喜交游,善谈论,温和中带有一种刚毅之气。

毛泽东在杨昌济先生家里经蔡和森介绍,认识了邓中夏。三人志同道合,很快就成了真挚的朋友。

毛泽东和这一批志向远大的青年经常在岳麓山、橘子洲、平浪宫等处聚会,臧否人物,畅谈国事。

1936年,毛泽东同斯诺谈道:“我逐渐在自己周围团结了一批学生,这是一小批态度严肃的人,他们人数不多,但都是思想上很认真的人,不屑于议论身边琐事。他们所做和所说的每一件事,都一定要有一个目的。他们没有时间谈情说爱,认为时局是如此危急,求知的需要是如此迫切,没有时间去谈论女人或私人问题。”“我的朋友们和我只乐于谈论大事——人的性质,人类社会,中国,世界,宇宙!”

毛泽东除了广泛结友外,还积极地参与学校的学友会活动。

在11月间,毛泽东被选为学友会文牍,负责起草报告、造具表册和会议速记。这一职务他一连任了四届。

第一师范的学友会始创于1913年9月,初名“技能会”,以培养学生的生活技能为宗旨;1914年改名为自进会;1915年秋正式定名为学友会。

学友会的宗旨是:“砥砺道德,研究教育,增进学识,养成职业,锻炼身体,联络感情。”学友会的成员包括在校学生和已经毕业的校友。学友会下设教育研究、文学、演讲、竞技等15个部。会中设有会长一人,由校长兼任,总理一切事务;总务一人,由学校学监兼任,主持日常会务;各部部长各一人,另有庶务、会计、文牍各一人。

1915年冬,毛泽东以学友会的名义,将昔日保皇派中的汤化龙、康有为、梁启超3人有关反对袁世凯称帝的言论编印成册。毛泽东知道萧子升的书法比较好,就给他写了一封信,请他为这本书题写书名,信中写道:

“近校中印发汤康梁三先生书文,封面当签署‘汤康梁三先生之时局痛言’十一字。大小真草,随兄为之。需此甚急,可否明天上午赐来?”

萧子升慨然应约,立即写好送来了。

小册子印成后,毛泽东和学友会成员们在校内外广泛散发,产生了强烈的反响。

陈昌在日记中写道:“上午8时,接润之兄书,并承赐《汤康梁三先生之时局痛言》一本。夫康氏素排议共和,今又出而讥帝制,其所谓时中之圣。斯人若出,民国亦云幸矣!”

这些课外活动并没有影响毛泽东的学习,他每天总是在天色未明时就起床,晚上熄灯后还要借着外面的一点微弱灯光苦读,这一良好习惯始终能够持之以恒,从不稍怠。

1916年2月29日,毛泽东在给萧子升的信中说:

“经之类13种,史之类16种,子之类22种,集之类26种,合七十有七种。据现在眼光观之,以为中国应读之书止乎此。苟有志于学问,此实必读而不可缺。然读之非十年莫完,购之非二百金莫办。昨承告以赠书,大不敢当。一则赠而不读,读而无得,有负盛心;二则吾兄经济未裕,不可徒耗。”

1916年5月7日,毛泽东在“国耻日”一周年追悼大会上为吴竹圃题写了一副挽联。

吴竹圃,湖南汨罗市桃林寺镇青山人,他是毛泽东的同窗好友,各科成绩出类拔萃,而且关心时政,指斥时政,与毛泽东志同道合。吴竹圃于4月间一病不起,年方20岁便撒手人寰。毛泽东为他题写的挽联是:

\begin{couplet}
吴夫子英气可穿虹,夭阙早知,胡不向边场战死?

贾长沙胜俦堪慰梦,永生何乐,须思道大数方深!
\end{couplet}

1916年6月6日,袁世凯在全国人民的讨伐声中死去了。

6月24日,一师已经放暑假了,毛泽东在给萧子升的信中说:

“话别之后,滞于雨,不得归;又以萑苻不靖,烽火四起,益不敢冒险行也。五六日来,阅报读书,亦云有所事事。然母病在庐,倚望为劳,游子何心,能不伤感!加之校中放假,同学相携归去,余子碌碌,无可与语。早起晚宿,三饭相叠。平居一日憎长,今如瞬息,寂历之景,对之惨然。……明日开霁,决行返舍。”

6月25日,毛泽东离开长沙踏上了返乡的路程;26日9时,到达湘潭,又步行70里,夜宿离韶山30里的银田寺。沿途的大好景色和乱世景象,使他感慨不已。当晚,他忍不住“捉管为书”,给萧子升写了一封信,述说途中的所见所闻。他写道:

“一路景色,弥望青碧,池水清涟,田苗秀蔚,日隐烟斜之际,清露下洒,暖气上蒸,岚采舒发,云霞掩映,极目遐迩,有如画图。”“近城之处,驻有桂军,招摇道途,侧目而横睨,与诸无赖集博通衢大街,逻卒熟视不敢问。联手成群,猬居饭店,吃饭不偿值,无不怨之。细询其人,殊觉可怜,盖盼望给资遣散而不得者也。其官长亦居饭铺,榜其门曰某官,所张檄告,介乎通不通之间焉。”

6月27日上午,毛泽东回到了故乡韶山,回到了母亲的病榻之旁,一颗“游子之心”得到了极大的宽慰。然而,他仍系念着中国的战局,尤其是湖南的局势,半月之后,便恋恋不舍地告别了母亲。

7月12日,毛泽东在返校途中到了湘潭,又忍不住给萧子升写信,述说感受。

不久,毛泽东回到了湖南一师,由杨昌济先生介绍,与其他同学一起寓居在岳麓书院半学斋内学习。

7月18日,毛泽东给萧子升写了一封长信。他对湖南军阀和都督走马灯似的更替,由此而给人民所造成的灾难,“愤愤不能平于心”,于是便激扬文字,针砭时弊,指点江山,申述自己对时局的看法和主张。他在信中写道:

“湖南问题,弟向持汤督(芗铭)不可去,其被逐也,颇为冤之,今现象益紊矣。何以云其冤也?汤在此3年,以严刑峻法为治,一洗从前鸱张暴戾之气,而镇静辑睦之,秩序整肃,几复承平之旧。其治军也,严而有纪,虽袁氏厄之,而能暗计扩张,及于独立,数在万五千以外,用能内固省城,外御岳鄂,旁顾各县,而属之镇守使者不与焉,非甚明干,能至是乎?任张树勋警察长,长沙一埠,道不拾遗,鸡犬无惊,市政之饬,冠于各省,询之武汉来者,皆言不及湖南百一也。”

在这一个时期,毛泽东不断看到《大公报》上所载的日、俄在1916年7月间为再次瓜分在中国满蒙的权益而签订协约的消息。报端还披露:日本内阁行将改组,因此,不少中国人就寄希望于炮制《二十一条》的罪魁祸首大隈重信首相下野后,日本对华关系有望趋于缓和。

7月25日,毛泽东给萧子升写了一封长信,主张对曾经附和帝制的罪魁祸首严加惩治,“庶几震竦天下之耳目,而扫绝风霾腥秽之气。”他在信中还说,新大总统黎元洪下令惩办洪宪帝制祸首杨度等8人,是“人心奇快”。“此衮衮诸公,昔日势焰熏灼,炙手可热,而今乃有此下场!”“故最愚者袁世凯,而8人者皆次也。”

毛泽东对新总统黎元洪也表示了不满,他写道:

“此次惩办,武人未及,如段芝贵、倪嗣冲、吴炳湘等,皆不与于罪人之数,舆论非之,即8人者,闻亦多逃矣。”

接下来,他分析和预测了中日之间的关系,他写道:

“此约业已成立。两国各尊重在满蒙之权利外,俄让长春滨江间铁路及松花江航权,而且助俄以枪械弹药战争之物。今所明布者犹轻,其重且要者,密之不令见人也。”

“大隈阁有动摇之说,然无论何人执政,其对我政策不易。思之思之,日人诚我国劲敌!感以纵横万里而屈于三岛,民数号四万万,而对此三千万者为之奴!满蒙去而北边动,胡马骎骎入中原。况山东已失,开济之路已为攫去,则入河南矣。20年内,非一战不足以图存!而国人犹沉酣未觉,注意东事少。愚意吾侪无它事可做,欲完自身以保子孙,只有磨砺以待日本。吾之内情,彼尽知之,而吾人有不知者;彼之内状,吾人寡有知者焉。吾在校颇有奋发踔厉之慨,从早至晚,读书不休。”“吾愿足下看书报,注意东事,祈共勉之。谓可乎?”

这正是:\begin{xemph}忧国忧民开慧眼,未卜先知超先贤。\end{xemph}


交友结党小试手,道义未担练铁肩。

欲知毛泽东后来如何将自身历练锻炼,请看下章内容。

东方翁曰:关于中日两国之间关系之发展,正如毛泽东所预料的那样发生了;而他那“20年内,非一战不足以图存”的预言,后来也果真成了事实:15年之后,“九一八事变”发生了,其结果是日本占领了中国的东三省;21年之后,日本又挑起了卢沟桥“七七事变”,发动了全面的侵华战争。斯时尚不足23岁的毛泽东,岂不是料事如神么?但毛泽东可能没有想到的是,在20年之后动员全国老百姓奋起抗战、统帅千军万马、运筹帷幄于抗日疆场之上的领袖,竟然会是他自己。
\end{document}
