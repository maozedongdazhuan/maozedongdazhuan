\documentclass[../../dazhuan.tex]{subfiles}
% 寸草春晖
\begin{document}
    \chapter*{我生之初}
    \addcontentsline{tod}{chapter}{我生之初}
    \pdfbookmark{我生之初}{Z02C01}

我的年龄应该从1948年秋末算起。在母亲的肚子里,我听到了解放宛城(1948年11月4日)和淮海战役(1948年11月6日-1949年1月10日)那渐去渐远的隆隆炮声。在那个刚刚结束的战争年代,母亲曾经站在我家前边那条河沟南边的官道旁,和乡亲们一起,一瓢一瓢地舀着茶水,让身背背包、腰扎十字攀肩带子(扎带子据说是受伤或死亡便于让救护人插上杠子立即抬走)的陈谢大军战士们解渴,目送着奔赴淮海战役战场的战士们,送走了一批又一批。在我出生的那一刻,已经是公元1949年7月15日(农历己丑年六月二十日)午时了。

我天生丑陋,父母已经有了三个男孩,原本想要一个闺女,一看我又是一个黑小子,便无奈地说,哎,这个小东西一向很安分,不踢不闹的,应该是个女孩,怎么又是一个“臭包子”呢?自此,“臭包”就成了我的乳名。父母请先生为我算命,先生让报上我的生辰八字,然后,他装模作样地掐算了一会儿,说道,这孩子是火命(旧黄历干支己丑在五行中为“火”,在属相中为“牛”。算命者说“火命”纯粹是骗人的),他的命很硬,天生就占了三个火——出生的年份火(如上),月份火(六月流金七月流火),时辰火(午时在五行中属火),长大成人,还不知道有多火呢!先生算命,原本是为了让求卦人听着耳顺,多付一点卦资,所以就只捡好听的话信口开河。

没承想那先生倒真的蒙上了一点点,我的确火了几把——十七岁在初中同班同学中组织七人“毛泽东小组”,秘密造反,争得了一个赴京学生代表名额,在毛主席第六次接见红卫兵时见到了他老人家。不管全中国全世界到底有多少人见过毛主席,我毕竟是那少数人中的一分子。十九岁坐上了母校初级中学领导核心革命委员会的头把交椅。25岁进入大学,被选为校团委委员,第二年成为历史系职位最高的学生头头——学生会主席兼团总支书记。年过不惑以唯一的党外副校长身份,成为县级卧龙区“人大”民主人士副主任候选人。后半生还写出了一部七百万字的传世之作——《毛泽东大传》。看官千万不要误会我是在宣扬唯心主义的宿命论。我从不信命。我不过是学一学曹雪芹在《红楼梦》中使用的“冷子兴演说荣国府”的“假语村言”手法,烘托一下气氛,给各位读者提一提精神罢了。

那时的父母亲也没有看出我会有什么火的迹象,倒是一副傻傻的样子,还有一个倔强的臭脾气,稍不如意就甩脸子,不吃不喝,少言寡语,坐在蒲团上一动不动。直到四岁,我才离开蒲团,会走路了。你说怎么这样磨人气人呢?用我们家乡话,父母一直叫我“伈子(眼睛不敢睁大、不敢抬头、小心恐惧貌)”,就是傻子的意思。所以,我不但有乳名,而且还有“伈子”这个绰号。一直到九岁那一年,1958年9月初,该上初级小学(1-3年级)了,我才用上了百家姓上的弓长氏,挎着母亲用像样一点的旧布缝制的“书包”,步入了半里开外的一个邻村旁边的初级小学堂。

\end{document}