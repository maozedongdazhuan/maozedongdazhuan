\documentclass[../../dazhuan.tex]{subfiles}
% 寸草春晖
\begin{document}
    \chapter*{小序}
    \addcontentsline{tod}{chapter}{小序}
    \pdfbookmark{小序}{Z02C00}

我在小学、初中学习期间,并不喜欢诗词。那时候,课本上的教材,老师领着读,压着背,成了一种负担。我爱上诗词,是在1968年冬到1969年初春。那时候,是我一生中最困顿的日子。在百无聊赖中,独自一人在茅屋里一个暗淡的角落,开始阅读《唐诗三百首》、《宋词三百首》,哪知一上手就放不下了。我发现自己与那些古人有不少共鸣,就像哥伦布发现新大陆一样,发现了自己的一片星空。于是就一发不可收拾地爱上了诗词,甚至也尝试着用一些现在看来不成样子的句子表达自己的情感。以至于后来每每有所感悟,不计工拙,不论格律,信手涂鸦,不管在什么时间,什么地方,都要记下来。哪怕是在被窝里,也要披衣下床记下来,竟养成了终生习惯。

我于诗词,爱其博大精深,高雅瑰丽。惜素无深学,只得皮毛。我的作品,说不上是诗词,充其量不过是一种带有韵味的日记体的长短句子而已。在上个世纪的九十年代,我将一些旧作整理成册,名曰《寸草集》。曾作小序云:“仆本宛东田舍翁后裔也,曾历十年寒窗,数载农耕,三年军旅生涯,又三年大学苦读。沉浮几度,至今与学子为伍,相濡以沫,别无他求。数十年来,吾不喜务生计,唯嗜烟酒,好读杂书,爱诗词,欣赏书法,略知音律,偶尔亢奋,放声一歌。乃当今社会精神贵族之流者。斯世也,吾心比天高,虑比河长,然自少及壮,时运乖蹇,且性情疏懒,终至今日,一无所成。噫嘻!”
    
时光荏苒,光阴如梭,转眼间吾已垂垂老矣!正是老病有闲,回顾平生,颇有感触,便将那些萍踪般的旧事写下来,以《寸草集》为经,以片忆为纬,编成8万言的不伦不类的回忆性文字,冠名为《寸草春晖》,展示鄙人七十余年以来的心路历程。
   
\end{document}