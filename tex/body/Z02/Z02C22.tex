\documentclass[../../dazhuan.tex]{subfiles}
% 寸草春晖
\begin{document}
	\chapter*{萍踪心语}
	\addcontentsline{tod}{chapter}{萍踪心语}
	\pdfbookmark{萍踪心语}{z02c22}
	
\begin{poem}{天安门感赋}[1982年7月21日]
\begin{xpref}
《中国青年》1982年第3期载文云:永乐十五年,明成祖任蒯祥为皇宫设计师,设计建造天安门。时人制有北京宫殿图,图中有蒯祥的影像,以表彰他的功绩。联想到1966年11月3日,我近距离地在天安门前接受毛主席(第六次接见红卫兵)检阅,心潮难平,遂吟小诗抒怀。
\end{xpref}
天安门楼巍而壮,丹青图影说蒯祥

铁血风雨劫难后,黎民犹思红海洋。
\end{poem}

\begin{poem}{忆旧二首}[1982年9月3日]
\subtitle{一\\改苏轼题西林壁抒怀}

横看成岭侧成峰,远近高低各不同。

今识庐山真面目,只缘身在此山顶。

\subtitle{二}

几经周折疑无路,壮心不已已伏枥。

未老忍作釜中味,恨无东风蹄不疾。
\end{poem}

\clearpage

\begin{poem}*{反思}[1983年1月26日]
	
且怨且叹非一端,谁记狼虎谷口前\footnote{唐末农民起义军领袖黄巢殉难处,起义失败。既有被剿杀因素,也有自杀因素。}。

非是坐成无深谋,凭谁会信不永年。

竖子翻手云复雨,反水愧使蒋汗颜\footnote{四一二反革命政变后,宋庆龄领衔发表讨蒋檄文。蒋一直称之为国母。华汪叶在毛主席逝世后不足一月,仅仅用8341几个卫士便一举擒获了党的另一副主席,一常委,二政治局委员。其中包括与毛主席结婚38年的妻子江青。}。

红黑与否有公论,留与后人发笔端。
\end{poem}

\begin{poem}{静思}[1983年1月30日]
\begin{xpref}
讲完高中历史课,让学生看资料。余徘徊在教室后边,偶得句,课后记之。
\end{xpref}

大河泥沙入海流,还有风浪在后头。

中华不乏热血儿,船破何须吾人忧。
\end{poem}


\begin{poem}*[8em]{我是一滴水}[1983年5月10日]
我是一滴水,多么渺小。

随着闪电,汇入滂沱大雨,

冲刷着浮尘,清洗人间正道。

我们汇集成江河,奔腾,呼啸,

向高山峻岭撞击,哪怕粉身碎骨,

卷走残渣污秽,吞噬一切怪石和礁。
\end{poem}

\begin{poem}{读《李斯列传》}[1989年3月30日]

堪惜孙卿\footnote{荀子又名孙卿。}教孺子,功成由斯败亦斯。

才忍塞外死扶苏,再欲牵黄\footnote{《史记·李斯列传》:“斯出狱,与其中子俱执,顾谓其中子曰:‘吾欲与若復牵黄犬俱出上蔡东门逐狡兔,岂可得乎?’”}何其迟。
\end{poem}

\begin{poem}*{重读《淮阴侯列传》感时}[1989年3月31日]

漂饭胯辱见英雄,依刘亡项百战功。

尚贤非兵下燕赵\footnote{尚贤,指齐广武君李左车被擒,韩信释而师之。},论将绝唱留汗青。

蒯通一言烹食其,良平蹑足汉高封\footnote{韩信求封假齐王,刘邦骂之,张良、陈平轻踩一下刘邦的脚,又附耳暗喻之,刘邦悟,笑骂韩信,乃封其为齐王。}。

长乐引颈怨女子,遂使吕雉成英名\footnote{韩信谋反,被吕雉邀至长乐宫宴饮,诛之。至今还有人为韩信一类人鸣不平,甚至要翻案。}。
\end{poem}

\begin{poem}*{诗二首}[1989年4月19日夜]
\begin{xpref}
乍闻被迫退位之胡列席会议,因情绪激动,心脏骤停,抢救无效,颇觉意外。感于时事,思绪万千,联想到在广州做寓公的叶,遂有所得,披衣记之。
\end{xpref}

\newpage
\subtitle{一\\悼胡}

少小甘为马前卒,老来轻薄才不足\footnote{胡上位,作报告时而站起,时而坐下,人笑曰“坐不住”。}。

自称读书破万卷,问尔可曾读《汉书》\footnote{汉书王莽传,更始元年七月,王莽以谋反罪迫国师公刘歆、曲阳侯王根之子王涉自杀。刘、王二人皆莽之鹰犬也。}。

\subtitle{二\\忆叶}

满腹“韬略”满腹诗,半生倥偬半生虚。

曾得高誉比诸葛,吕\footnote{吕端有大事不糊涂之誉。}陈\footnote{陈平有平吕(雉)安汉之功。}蒙羞为尔师。
\end{poem}

\begin{poem}*[8.25em]{感时题王莽五言诗三首}[1989年4月26日]
\subtitle{一}

尔本余孽枝\footnote{封建官僚之后裔。},妄自充栋梁。

一从改令后,十年\footnote{指“黄金十年”。}乱玄黄。

黎民不可愚,绿林有二王\footnote{指王匡、王风。}。

八方来风雨,一梦成黄粱。

\subtitle{二}

尔本山中猴,沐冠便称王。

小惠买人心,大盗愈张狂\footnote{“小盗窃钩,大盗窃国”。}。

千人竞发指,一夫独傍徨。

名欲垂青史,街头碎尸扬。

\subtitle{三}

尔本掌中物,沉浮赖沧桑。

三垂淯水钓\footnote{王莽在宦海三起三落,数度隐于南阳。},鸟尽良弓藏\footnote{“飞鸟尽,良弓藏。狐兔死,走狗烹”。王莽以谋反罪诛鹰犬刘歆、王涉。}。

名谋万民福,实求私欲张\footnote{名曰改制,实为掠夺。}。

天道不可违,戏场变刑场\footnote{王莽暴尸街头。}。

\end{poem}

\mbox{}\\
\\
\\

………………………………


读彭雪枫家书

1989年8月31日午时

将军享年37岁,予今步入不惑之年。读是书,方知人间事何其相似尔!遂吟以为志。

将军征战二十年,强虏烟灭指顾间。

高谈阔论天下事,文韬武略凝笔端。

家事不如英雄愿,热情散尽意未阑。

道是无情终有情,留得遗憾在人间。



武则天无字碑

1989年8月31日午时

随心所欲放光辉(1),敢作千古第一贵。

女皇不顾身后事,身后唯有无字碑。

功过是非自不论,女娲女祸任相委。

依稀知道千秋后,自有高人说崔巍(2)。

注(1)武则天自造字名曌,日月当空也。(2)毛泽东赞誉武则天。



毛泽东颂

1989年8月31日

余购得权延赤、李银桥合著《走下神坛的毛泽东》一书,拭泪观之,掩卷之余,犹唏嘘不止,难以自已,遂吟成此篇,以志对伟大的无产阶级革命家毛主席的深切缅怀。

湘江之水育英才,日出韶山驱阴霾。

敢上九天揽明月,五湖四海是胸怀。

一从长沙失利后,战场便向山村开。

三反“围剿”真神算,四渡赤水任去来。

长征前后临战垒,雄关漫道履平台。

徐说持久驳速胜,长缨缚龙纾国災。

枪林昂首退群虏,临危安坐何悠哉!

东北关门痛打狗,老蒋败退走蓬莱。

美帝签字板门店,一拳开免百拳来。

莫道不怜小儿女(1),男儿有泪哭民災。

蔽衣褴褛人不知,不摸金钱爱雪白。

字惊书圣走龙蛇,词引诗仙把酒来。

自从战事平息后,几经坎坷到鬓白。

统帅遗志凭谁问,我哭导师乱灵台。

注(1)朝鲜战争爆发,毛主席送长子毛岸英参战。小女儿李讷年届三十,先后到“五七干校”和农村劳动锻炼,看病都不回北京。



愁

1989年9月4日夜半

人生在世,苦辣酸甜,百味俱尝。悲欢离合,生老病死,在所难免。因此,人皆可以言志,可以说愁,伟人亦不例外。言志者,未必都是无产阶级意趣,说愁者,未必都是小资产阶级情调。志者,心所向也,愁者,心所苦也,皆不过是一时心情之宣泄矣。

一

少年说愁不识愁,中年说愁愁白头。

待到不再说愁时,一抔黄土葬风流。

二

自古人说诗言志,我却以诗解愁肠。

以诗解愁愁难解,欲把稿酬换杜康。



献给中华人民共和国四十周年

1989年9月24-27日

手持权杖的人,可以指鹿为马,指主为奴;可使江山变色,把天地翻覆。

拥有金钱的人,可以镶金错银,招摇过市;可下酒海使神弄鬼,可入肉林拥翠抚姝。

我,一个小小的书蠹,有人戏称精神贵族。

属于我的,是一管秃笔,几卷破书。

不过,我可以自豪地宣称,我拥有的是一个世界。

那世界,比天高,比海阔,

在那里,我可以飞翔,可以小憩,

可以长啸,可以痛哭,

可以写出春夏秋冬,可以画出美丑善恶。

该歌者我歌,该诅者我诅,永远充实、自由、满足。

我曾经泪水长流,我还要大声疾呼:

当今世界,人与人之间,应该互相制约。

只有制约,才有平衡,

只有制约,才有民主,

只有制约,才是通途。

那些反对制约的人,不是既得利益,便是居心叵测。

正是缺乏制约,才有魍魉畅笑,民怨塞途。

太阳啊,黑子果真能遮住你的光辉吗?

镰刀斧头啊,两个二十八年的风雨,果真销磨了你的锋芒?

既然创造了一个世界,就奈何不了小小的寄生虫?

不幸的预言,果真要应验吗?

为何一忍再忍,让那些自称公仆的人,明火执仗欺主。

快快打开地狱之门吧,把那些披着人皮的豺狼,把那些民贼国蠹,永远钉在历史的耻辱柱上。

让那美好的世界,重现光华,青春永驻!



无题

1989年11月2日

岁在不惑之年,喜听活诸葛申凤梅越调唱腔录音,忆旧抒怀。

束发投笔从元戎,十载沉浮幻千重。

忍看万众争一粟,一夫独向首阳(1)行。

难哺子女囊中羞,自甘淡泊两袖风。

幽居不思天下事,半掩房门听申声。

注(1)首阳山,伯夷、叔齐隐于此,不食周粟,采薇而食之。



燕子

1990年6月7日上午

呢喃双双飞,来去复匆匆。

筑巢旧时屋,不入帝王宫。



梦

1990年7月8日

接二女放学途中,与褚君谈及人生沉浮,宦海风云,多有感慨,偶成数句。

梦是什么?

梦是贪婪,是占有,是阴谋,是掠夺。

旧梦破灭了,新梦重新来过。

从王莽到“日不落”(1),

从拿破仑到希特勒,

一部历史,就是一串零的长河。



梦是什么?

梦是理想,是实践,是奋进,是超越。

旧梦失败了,新梦从头求索。

从万户(2)到加加林(3)

从莫尔(4)到马克思,

一部历史,恰似天际虹桥一座。

注(1)16世纪西班牙国王自称“日不落帝国”。(2)明代幻想家,乘自制火箭升天。(3)第一个进入太空的苏联宇航员。(4)空想社会主义奠基人,著有《乌托邦》一书。



冬月

1990年11月24日

是晚停电,与家人漫步庭前,妻儿见新月清光如洗,稚兴勃发,竞相吟诵嬉戏。妻笑谓余曰,古人作七步诗,汝能否?余沉吟少顷,勉力凑成数句。

半轮冰盘悬南天,吴刚玉兔只不见。

残桂飘零不堪说,欲睹芳华待来年。



感时

1991年1月4日午前

前日午后,闻宛城专区李某晋升浙江省要职,感慨良多。晚上,酣睡后起而方便,偶有数句入脑,诧异之至,盖因心仍系之之故。天寒惰于笔记,清晨只忆得首、2、7句。4日有客,未暇思之,客走,不忍弃之,搜索枯肠,乃成此篇。

老圃无芝(1)夸浮风,小李(2)有花(3)花杭城。

炎日昭昭尚不察,冷月昏昏更难明。

中原自古多术士,灵旗变幻妙无穷。

未忘信阳有饿殍,忍看一鼠戏生灵。

注(1)吴芝圃。(2)靠副县长丈母上位的农家大学生。(3)吹牛皮。



无花果

1993年8月23日

在你的世界里,

没有喧嚣,没有炫耀。

却总是悄没声的,

敞开你的心扉(1),

给人一个惊喜,

和你那最甜蜜的微笑。

注(1)无花果成熟时,顶部裂开,露出鲜红而甜蜜的果肉。



参加济源会议途径洛阳

游白马寺

1994年11月12日

洛阳马寺越千年,重殿叠阁钟声(1)喧。

山门左右埋二祖(2),文化东西一马牵。

善男信女纷沓至,心香愿表化青烟。

佛陀高踞开口笑,施主可曾知佛源(3)?

注(1)马寺钟声是洛阳八大景之一。(2)指印度来华二僧,埋骨寺门内两侧。(3)白马寺有“祖庭佛源”之称,这里指佛教的产生和佛教的本质。



随学校组团出游旅途组诗

1995年7月12日-14日

一

过武关思古

车跃商洛觅旧踪,闯王何处曾屯兵?

若非劫后再秣马,焉能宛洛振雄风。

二

商洛行

陕边无数山,满眼不毛地。

多少雄杰血,不染关山碧。

三

潼关吟

潼关埋白骨,几多逐鹿魂。

雄姿依旧在,不见夺关人。

四

登华山

少有涉险志,无缘踏青云。

遍寻绝顶处,可有论剑人?

五

西安长歌

多年梦求,纸上谈论,今日终上三秦。

问汉祖何在,唐宗何在,黄王何在?

持一管秃笔,二掬心酸,三杯啤酒,祭我中华先烈。

遥想当年,周家天子,曾言率河之滨,莫非王土。

奈何博千金一笑,烽火几明灭?

刘季提三尺剑入城,阔论兄弟赀财。

可知《大风》(1)难继,平帝饲莽孽(2)。

贞观君臣,舟水绝唱,后来凭谁曾记?

黄王雄师初来,破朱户,齐天下,空前壮烈。

恨竖子(3)反水,泰山饮刃,何等惨切!

噫唏,泾水逝兮渭水流,不复见当年雄姿英杰。

倩何人重倡《大言》(4)早除蒙尘,再现伟业。

注(1)刘邦《大风歌》。(2)14岁的汉平帝刘衎被外戚王莽(皇后之父)毒杀毙命。民间传言:“刘邦斩蛇起义,蛇魂索命:‘还我命来。’刘邦答曰:‘山上哪有命还,到平地再还。’平帝果然偿还了性命。”(3)叛徒朱温。(4)宋玉《大言赋》:方地为舆,圆天为盖,长剑耿介,倚天之外。



晨练偶思

1995年11月10日

一

煤渣

曾经燃起一片火海,

那是你一生中最辉煌的时刻。

而今,你静静地躺在这里,托起人们的脚步,

朝前迈,朝前迈……

二

跑道

四百米,远不止四百米。

因为人们的脚步,永远是向前的,

五十年,一百年,

走下去,走下去,沿着你的轨迹。

三

生命

几十年,仅仅几十年,至多百余年。

人的生命,虽然可以延续,

子孙繁衍,生生不息,

但生命永远不会周而复始。



枕上听雨得句

1996年3月29日

未见天公着颜色,轻风细雨绿枝头。



随学校组团

游石人山少林寺组诗

1996年5月1日

一

游石人山追梦

少时遥望西北山(1),雨后夕阳彩霞间。

日思山中多绮丽,夜梦飞越似谪仙(2)。

重峦崔巍尚有峰,叠嶂迷蒙隐林苑。

步入伏牛东麓下,方知还有石人山。

注(1)宛城在西北山东南方向。(2)李白。

二

少林寺

少林拳棒甲天下,而今刀子亦惊人。

门票吓煞千里客,驻足山门慨清贫。

三

自慰

千里旅游不为游,抛弃俗念爬山头。

追寻灵感求自慰,不计工拙不为酬。

四

过叶县

今日叶县非昆阳,不见萧王曾缒墙。

非是舍生集旧部,焉能决战歼逆莽。

五

心迹

古刹名观多游玩,只羡道人不羡仙。

他年我若抛俗务,青袍(1)醉伴孤灯眠。

注(1)道袍颜色分七个等级,青袍为中等法师所着。



梦忆

1996年9月19日夜

凌晨3时披衣记梦中残句

夜半最忆是毛公,混沌初开梦难成。

少年已折凌云翅,半生不敢唱《大风》(1)。

注(1)刘邦《大风歌》。



南街颂歌

1996年9月22日

在郑州省教师进修学院“中学校长培训班”学习期间,首次到临颍县素有“共产主义小社区”、“东方巴黎公社”、“小延安”的南街村参观学习。

西风烈,虎头(1)歌歇。

惊看东方巴黎公社,弹丸之地,别有一番景象,小村南街。

外圆内方(2),老歌(3)新唱,两手堪称第一(4)。

干群同心,创建共产主义社区,锐意未竭。

底是何等精神?走的还是“乡村包围城市”的道路,

村中毛公(5)还在,物质不灭(6)。

注(1)昔阳县大寨虎头山。(2)对外与市场经济社会对接,内部实行社会主义计划经济公有制。(3)毛泽东时代的红歌。(4)一手抓革命,一手促生产。(5)临颍大道口矗立着高大的毛主席塑像。(6)“物质不灭”是毛主席的著名论断。



黄河行吟

1996年9月23日

从省教师进修学院到黄河景区

一

我本散淡人,好游大山名谷。

攀华山涉险,入皇城散步。

登黄山捧日,坐邙山濯足。

西望黄河来路,仰天狂笑,心中长哭。

二

万事本无律,后人自做桎梏。

作诗词定格,造神仙自欺,

蒙昧招灾,愚人自误。

何时棒喝当头,天道人心,归真返璞。



罢官乐

1997年11月13日

自1997年1月6日任某初中校长,至1997年11月11日因故免职,计10个月零5天。

九七一一复一一,十月风雨如樊牢。

直面历史心无愧,脱去金钩不弄潮(1)。

注(1)此后一直赋闲在教育局多时。



酒后看电视剧

1998年4月26日

男儿泪无多,人间情几许?

偶有动心处,一任奔流去。



无题

1998年春,区政府组织“灭火队”入驻辖区内“老大难”行政村。我们这个队的成员有教育局副局长、房管局副局长、土地局副局长,还有一个是镇税务所所长。我们一行5人,到了安皋镇一个两派长期对立的行政村,驻队3个月,其间作五言诗一首。

男儿泪可贵,旅途多忧愁。

一到伤心处,无由不奔流。



送二爱女分赴许昌、北京

1998年9月13日,送长女入许昌师范专科学校,9月16日送次女入北京北方工业大学。

寒窗苦读终有期,日盼录讯乍闻之。

老妻整装吾筹款(1),送儿启程展翅飞。

廿年依依爱怜事,最是呀呀学语时。

寄语二女学业就,代父再吟思毛诗。

注(1)恰逢高等院校招生首次实行双轨制,是年新生入学需交高额学费、书本材料费、住宿费等。



解梦

1998年11月4日夜

旧时村落沉脑海,土垣茅屋总关爱。

十之八九归故里,亲情随夜入梦来。



宴乐

1998年12月4日

自“改开”以来,上流社会吃喝之风日盛,奢靡空前,好一番“盛世景象”。

熊冷鲍腥(1)荠蒿(2)香,灯红酒绿日月黄。

娲皇造人有良莠,神农(3)施教悔断肠。

注(1)熊掌、鲍鱼,泛指山珍海味。(2)荠菜、茼蒿,泛指山野菜。(3)女娲,补天造人。(4)神农氏是传说中三皇之一,教人渔猎农耕。



赋闲吟

1998年12月18日

自从离开初级中学以后,一直在教育局一办公室闲坐,无事可做,便重读《三国演义》、(水浒传)、《红楼梦》等书。在家时,继续购买、搜寻、研读、抄录、整理有关毛主席的史料。有时亦练练书法。

二十载耕耘未抬头,偏逢鬼蜮耻蒙羞。

一怒愤作百字对(1),年余闲煞六月牛。

日日举杯笑看《蜀道难》,夜夜邀月共赏《饮中八仙歌》。

地板做纸水为墨,五指捻管临颜柳。

摹《二表》(2),学张颠,拥衾膝上犹画钩。

门前冷落不见客,愁眠醒来读《春秋》。

妻儿不知我无力,耳提信催搏激流。

原以为贫贱之交似管鲍(3),到如今中山狼舞牝鸡讴。

此生最爱是毛公,不拜父母不弯腰,安能使我摧眉事沐猴。

君不见贾岛枉磨十年剑,“经理满街走,教授多如狗”(4)。

君不见荒原豺狼笑,华宴巴儿高举箸。

天地为之失颜色,世上无男儿,妆台正粉头。

自古一石难补天,我只低眉读《国风》(5),抚吴钩(6)。

注(1)1997年曾作一副百字对联以泄愤。(2)岳飞手书诸葛亮前后出师表。(3)管仲、鲍叔牙,有成语“管鲍之交”。(4)北京那个将毛泽东时代“五七干校”污为“牛棚”者的名言。其实贪官何尝不是多如狗,甚至比狗还多呢?(5)《国风》是《诗经》中的精华部分,反映了劳动人民真实生活,表达了他们对剥削者、压迫者的痛恨和争取美好生活的信念,还有不少民歌对统治阶级的荒淫无耻,予以有力的讽刺和鞭笞。(6)一种弯形刀,相传为吴王阖闾所做。后泛指锋利的宝刀。



冬日偕妻沿白河堤跑步

到淯阳桥头跳舞

1998年12月20日

老燕傍翅复呢喃,瘦月如钩意阑珊。

细看淯水有也无,流霭淡雾柳如烟。

林中少侣犹折柳,滩头翁媪慢挥拳。

相拥踏歌翩翩舞,暖阳出水看五环(1)。

注(1)桥有五孔,每孔上方有一大拱,倒映水中,形成五个环,像奥运会会标。



独酌

1998年12月21日

仨五花生米,一瓶二锅头。

化作四江水,和着电视流。



毛主席华诞日白河堤上

1998年12月26日

昨夜圣树(1)满城翠,疑是掳得金花(2)归。

东风亦知亡国恨,柳如酥手抆我泪。

无云二百一十日,气粗不怕遭天雷(3)。

淯水(4)若是似酒泉,射日(5)痛饮三万杯。

注(1)12月25日夜晚的圣诞树,满树华光。(2)八国联军攻占北京,总司令瓦德西召名妓夏金花侍寝。(3)全国多地大旱半年多,宛地亦旱7个月。“朱下岗”说大旱三年也有粮食吃。(4)白河又称白水、淯水,自东而西穿宛城而过。(5)传说后羿箭射九日,解除旱情。



无题

1999年2月10日

冲天遗恨狼虎恶(1),风摧浪噬任折磨。

敢叫路无不平事,当涂抱月又如何(2)!

注(1)黄巢曾称冲天大将军,后在狼虎谷被围自杀。(2)传说李白在当涂县好友家中饮酒后跳入水中捞月,溺亡。



正月十一日宛城火车站送曦儿赴京

1999年2月26日

夜半惊心爬车窗,恰似串联(1)走京广。

同齿(2)三十二年前,整装擎旗上井冈(3)。

宛汴气锐曾搏虎,京华观日(4)浴艳阳。

日暮摇笔志未老,千砺万磨盼晨光。

注(1)红卫兵大串联,归程在京广线上几次爬车窗。(2)32年前19岁,与儿今同岁。(3)1966年底至1967年初,负10斤重背包徒步28天到井冈山。(4)1966年11月3日,在天安门城楼前近距离接受毛主席检阅。



学书法感言

1999年3月4日

初摹毛主席墨宝,心旷神怡,再临乃惊魂动魄。遍观草书名家,未有若此大手笔者。走笔如神,非有大气度、大胸襟者不能为之也。纵览现代书坛,思虑良多,因有是感。

早识国宝未动心,初临已惊懵懂人。

最爱毛体有大气,信手点划皆通神。

家家珠玑痕似血,卷卷遗墨飘如云。

堪笑书坛现代派,创新岂可忘先人。



闻我驻南斯拉夫联盟大使馆被炸

1999年5月9日

从1999年3月24日起,以美国为首的北约凭借绝对的空中优势,使用大量高技术武器,对南联盟频繁发起悍然轰炸,史称“科索沃战争”。霎时间,北约各国的1100多架飞机铺天盖地侵入南联盟领空,对包括民用设施、客运汽车、住宅楼房、工厂企业在内的非军事目标实施丧心病狂的无差别轰炸。在78天的轰炸中,北约共造成南联盟境内2000多人死亡,6000多人受伤,近100万人沦为难民,2000多亿美元的财产付之一炬。据不完全统计,北约共炸毁25000多栋民宅、14个飞机场、18个幼儿园、69所学校,39所医院。

在此期间,中国官媒每天都如实报道南联盟遭轰炸的惨况,惹恼了美帝。5月8日晨6时,美军悍然轰炸中国驻南联盟大使馆,3枚导弹从一个弹孔穿透3层楼房到地下室,致3死20余伤。

美军轰炸我大使馆的消息传开后,社会各界人士无不感到愤怒,各高校广大学生更是群情激愤,他们手捧三位遇难英雄的遗像,走上街头,声嘶力竭地谴责美军的暴行,有学生捡起砖头向美驻华大使馆投掷,他们高呼:“坚决捍卫祖国主权,反对北约的人权侵犯!”他们心里滴血,脸上流泪,高唱着《义勇军进行曲》、《国际歌》、《团结就是力量》等歌曲,表达了对祖国尊严的坚守。余闻之,慨然作长短句抒怀。

美帝挥大棒,北约作祟,弹光横流,问世间公理可有?

小国苦撑持,谁肯援手则个,想必旧时友。

八外长重演慕尼黑(1),克林顿(2)气消未?

热脸儿(3)蹭个凉屁股,张伯伦(4)笑有同俦。

从来强国,文不爱钱,武不惜命,历史可曾改写?

捣你三五弹,看你还七讲八讲(5),欢颜吹唱,二胡演奏(6)。

今学子再发先声,安内乎?攘外乎?

钓鱼台上使人愁,雨凄凄,风满楼。

注(1)5月6日,西方七国集团外长和俄国外长签约,与北约5条件大同小异。重演二战前英法德意签订慕尼黑协定、出卖捷克斯洛伐克利益的一幕。(2)美国总统。(3)俄罗斯总统叶利钦,他于1999年12月辞职。(4)二战前推行绥靖政策的英国首相。(5)江“代表”高调:“五讲四美三热爱”。(6)“三个代表”4次访美,曾用英语发表演讲(吹嘘会英、俄、法、罗马尼亚4国语言),多次拉二胡,清唱京剧,也唱儿歌:“小也么小二郎,背着书包上学堂……”。有一次,正当他夸夸其谈时,一记者突然发问:“六四”事件后,你们有两千多人至今还被关押在监狱里?他一愣,有顷,说:“我不知道。”



学书法心得联字

1999年6月27日

白日习字梦里思字功在勤后;

心中有字眼里有字意生笔先。



楹联

1999年10月中旬-11月1日

一

题书法作品联

惊惊惊,惊百代遗痕凝如血;

喜喜喜,喜一时新墨飘似云。

二

题学苑联

(一)

育德育智育体美,育人功归庠序;

知天知地知古今,知本泽被家国。

(二)

雏凤清声盈庭院;老骥伏枥满桃李。

(三)

济济贤达,已成博采兼容气象;

殷殷学子,大有囊虫映雪遗风(1)。

注(1)晋代车胤借荧火虫之光读书,晋代孙康借雪光读书。

三

题彭雪枫纪念馆

马鞭指顾中,日伪顽灰飞烟灭;

帷幄笑谈间,豫皖苏雾散日出。



无题

2000年3月19日

蒙李金平先生赠印两方,一为“成悳”,一为“弓长成德印”,均为篆文。邀师友会于寒舍,席间得句,赠金平先生。

豪饮未必不俊才,我欲推杯浮大白。

青埂峰下石头记,可是仁兄亲手刻?



你是我生命的延续

——写给儿女们

2000年5月10日-11日

我从黑土地里走来,你是我生命的延续。

自从有了你,我的希望更加美丽。

你明白吗?孩子,你的所有就是我自己。

人的生命是短暂的,一生写出几节音符并不难,难的是谱就一曲优美的旋律。

儿不见,民主革命先行者,孙中山先生屡战屡败,中国照旧在黑暗中栖息。

二十世纪的文化旗手鲁迅先生,他最遗憾的是没有写出长篇文字,否则,我们看到的不止是《石头记》。

才学精神如毛泽东者,世间有几?

他那辉煌的业绩,也掩不住水晶棺里,作势欲起的身姿,和他那微微开启的双唇,似乎还有要说的话语。

孩子啊,不知你是否已读懂我的心曲,因为你毕竟不是我自己。

奋斗吧,别害怕荆棘阻路,成败利钝,多在人谋。

向前走,莫畏书山学海,锲而不舍,自得天酬。

为了你,为了我,为了无愧于这块丢不开扯不断,总也忘不了的黑土地。



千年初咏叹五叠韵

——从洛阳平安夜大火(1)说起

2001年1月3日

一

狂欢,狂欢!

大火烧红城半边。

三百新鬼惊回首,

忽一个声音说道:

防火要防患于未然(2)。

二

未然,未然?

怕是猪油蒙心眼。

好一片嘘声:

假币,假货,

毒大米,黑社会,巨贪。

三

巨贪,巨贪!

一股浊流浪滔天。

最是惊人处,

同僚雇凶,海关走私,

副省座(3)玩失踪,副国级(4)包“二奶”,

还有“三陪”部长,四“进宫”警官(5)。

四

警官,警官?

羞煞头上青天。

飞车撞人,枪杀无辜,

“旧船票”何尝是俊男(6)。

五

俊男,俊男,

皓首匹夫皆粉面(7)。

君不见,海峡东岸,百年铸造,

千年之末,暗礁已沉船(8)。

注(1)2000年12月25日晚,洛阳市老城区东都商厦发生特大火灾,在二楼狂欢的男男女女死亡309人,受伤7人。(2)江代表喊话。(3)副省长胡长清。(4)副委员长成克杰。(5)上述均有官媒报道。(6)某男歌手被刺,暴露出同性恋丑闻。(7)高官染发,装嫩。(8)台湾国民党败选,丧失统治权。



七言戒贪诗四首

2001年1月3日

一

贪官手

旗下曾誓为公谋,手握两弹(1)入浊流。

天网虽疏终不漏,陈毅有诗(2)在前头。

二

贪官脸

仰面是猫俯成虎,半阳半阴两张口(3)。

莫道吞吐凭卿意,法剑高悬能斩头。

三

贪官心

欲壑难填今难填,叛主(4)傍款(5)灭天伦。

时下莫唱“养廉”(6)曲,胡(7)成(8)便是掌嘴人。

四

劝贪官

谁信严嵩是饿殍(9),当知戈氏成烹狗(10)。

祸国并非立身计,身后徒留子孙羞。

注(1)“银弹”即票子、“肉弹”即女人。(2)陈毅诗句:“莫伸手,伸手必被捉。”(3)《语海》壮族民谣形容官相:“官字两个口,说话有两手。”(4)背叛劳动人民。(5)傍大款。(6)“高薪养廉”。(7)即副省长胡长清。(8)即副委员长成克杰。(9)明代权奸严嵩被抄家后沿街乞讨,饿毙街头。(10)指戈尔巴乔夫,背叛了共产主义运动,“狡兔死,走狗烹”,他的下场并不光彩。



王伟烈士殉难二十日祭

“四一”事件感言

2001年4月20日

撞机(1)谈判何时了?

桌上马拉松,南海美机扰。

抬手专打笑面人,国人气短美帝刁。

当重读,风流史,宋皇徽与高(2)。

谁言弱国无外交?

板门才签字(3),援越又试刀(4)。

毛公一怒发神弹,万米高空追尤儿(5)。

尼克松,田中相,访华乐陶陶。

注(1)4月1日,美军战机在我南海边撞毁王伟驾驶的战机,王伟壮烈牺牲。(2)宋徽宗擅长书画,成为金国囚徒。宋高宗擅长行书,以“莫须有”罪名杀害抗金名将岳飞,偏安一隅。(3)侵朝美军在板门店谈判中被迫签订《停战协定》。(4)援越抗法,援越抗美。(5)即美国U2高空侦察机和U2无人机。从1962年9月9日到1965年,我空军部队击落其侦察机5架,击落其无人机3架。



为南开附中南阳学校题

2002年2月6日

联津门名校延宛中耆宿卧龙惊雷开辟阳光春苑;

遵毛公教诲守先贤校训学界方寸化育金色秋实。



牵牛花(儿歌)

2002年2月7日

牵牛花,喇叭长,攀高枝,好向阳。

哥们组合朝天吹,不为功,不为赏,

吹个《百鸟朝凤》曲,呜呜呜,

眼前一片明光光。

再吹一个《太平年》,呜呜呜,

来日《泥马渡康王》。



2003年2月10日(正月初十)晨大雾,赴商丘分校任校长。

寓居归德(1)寄妻

2003年3月17日

相知半世长依傍,阴晴历历自难忘。

抱病犹为小儿计,归德孤影夜思乡。

注(1)商丘古称归德府。



看电视新闻有感于伊拉克战争

2003年3月20日,美国以伊拉克拒绝交出大规模杀伤性生化武器并暗中支持恐怖分子为借口,联合英国等多国军队对伊拉克发动侵略战争。此前,鲍威尔拿一瓶“白色粉末”,说伊拉克有大规模杀伤性武器,这些武器很不安全,只有找到并消灭它,才能保证伊拉克民众的安全。为此,美英绕开联合国对伊拉克实施军事打击。美国出动兵力19.2万人,英国出动兵力4.5万人。萨达姆政府军队45万人及预备役65万人应战。美帝以子虚乌有的战争借口打击对方,最终也没有找到大规模杀伤性武器,鲍威尔那瓶“白色粉末”成了笑话,不少政治人物讥讽为“洗衣粉”。

2003年3月22日晚

当今无公理,弱肉强者食。

两个男人玩游戏,

小布什(1)咄咄逼人,老萨(2)仗剑发誓,

“战斧”导弹(3)腾起。

眼睁睁猫儿戏老鼠,哗剌剌黑云欲摧城。

安南(4)老儿却道:

放下你的剑罢,献出你的石油,别让他老人家生气。

更那堪,东邻絮絮,西邻叨叨,

韬光乎,养晦乎,全是放屁。

什么“多元”,什么“和平发展”(5),

竖子无行,放纵一极。

曾记否,慕尼黑阴谋后,

“轴心国”肆虐,可怜见亿万生灵涂炭,

终叫那群雄牵手,一扫阴霾,海晏日霁。

注(1)美国总统,战争发动者。(2)萨达姆,伊拉克总统。2003年他的政权被美国领导的多国部队推翻,他本人在逃亡8个月后被美军擒获,2006年11月5日被判处绞刑,绞断了脖子。(3)美制巡航导弹。(4)联合国第七任秘书长。(5)邓不翻案在1985年提出“和平发展,合作是当今时代的主题”。



赞伊拉克军民抗战

2003年3月26日

天外飞来巡航弹,焦土頻频响警笛。

农民活捉大鹏鸟(1),将军提督四战区(2)。

古兰(3)奇香冲天网,卫国干城第一师。

试看布什(4)布莱尔(5),底是鹰隼底是鸡?

注(1)美国战机。(2)萨达姆把全国划分为四个战区:北方战区,南方战区,中央战区,中幼发拉底河战区,统归萨达姆指挥。(3)古兰经。(4)美国总统,小布什。(5)英国首相,美军的帮凶。



青少年自律歌

送初二班同学(1)

2003年3月28日

吾生有缘,修来一班。

外不人欺,内和一团。

家兮国兮,学习为先。

男当自强,女是木兰。

天下为任,修身维艰。

歌兮咏兮,心存高远。

注(1)校内问题班,几个刺头均是富家子弟,班主任、任课老师颇为头痛。



寓居归德寄儿女

2003年3月30日夜

梦中见长女儿时憨态,再不能寐,老泪纵横,披衣吟之。

一

赠颜儿

风雨归来觅桃种,麟儿接来赐芳名。

三岁已似牧羊犬(1),开蒙便识父女情(2)。

投笔击水(3)下南国,执手赠句学毛公(4)。

商战诡谲须智勇,来年喜报告乃翁。

二

赠曦儿

乃翁无意得和璧(5),我儿睿智初长成。

学海频频现身手,庠序魁元(6)我不能。

世态炎凉尚不知,道路艰辛未可轻。

寒舍陋室曾寄语,凌云阁上可留名。

三

赠棹儿

戏言有因孽缘生,飘萍无定屡心惊。

小儿数度脱藩篱,老叟几处觅劣踪。

神归心收入庠序,耳提面命见初功。

玉树已成临风势,前途未知是鹏程?

注(1)自幼领着妹妹玩耍。(2)上小学,与妹妹联名留言:“爸爸,请您吃饺子。”(3)大学毕业要去深圳做生意。(4)以毛泽东赠开辟大别山根据地部队的话,嘱咐女儿。(4)和氏璧。(5)榜上第一。



自白

——我心中的魔鬼

2003年4月间,一场突如其来的非典震动了全国,人心不安。为了学生安全,我和朋友一起去见宛籍市委书记刘,递交了一份将学生暂时疏散回家的申请书。5月1日下午,接到请求放假的批复。2日上午,我召集学校领导班子成员开会,决定放假半学期,部署班主任通知学生家长接学生回家。3日,我与宛籍职工同车回宛。心情极其激动,工作中的不快折磨着我,萌生了一种创作的欲望,随即向司机要了笔,拆开两个烟盒做纸,草出了一首《自白——我心中的魔鬼》,全文如下:

2003年5月3日

仁慈,自负,

你是我心中的魔鬼。

恨你,害怕你,

可又常常牵挂着你。

多少年了,食不甘味,

丧失理智,半辈子碌碌无为。

有了你,总把别人想得太好,

有了你,总把事情处理得很简单。

诅咒你啊,可你的名字偏偏叫不悔!

你曾经说过,

蔑视政治流氓那“宁负天下人”的无耻,

你始终认为,

理想主义信念是你心中的丰碑。

快快离我而去吧,

摆脱你的那一刻已经为期不远了。

到那时,连同我自己,

统统抛入炉火中,

燃烧,烧毁。



你是一丝清风

2003年6月17日

你是一丝清风,

拉起烦恼一路走去。

你是一阵细雨,

带来春的希望,夏的凉意。

你是一片彩云,

将无限遐思撒在大山的背上。

你是一缕晚霞,

把灿烂融入历史的心里。



我愿

2003年6月18日

我愿是一条小船,

随着清风去流浪。

我愿是一只雄鹰,

奋力展翅在太空翱翔。

我愿是一缕朝霞,

伴着太阳一同起床。

我愿是一片彩云,

合着落日进入梦乡。



纪念毛泽东同志诞辰110周年

——观河南省会“七一”广场歌会

2003年7月1日

我的心颤抖了,我的血液澎湃了,

我的眼睛模糊了,我的胸膛爆裂了。

啊!那是一种什么声音?

我急忙屏住呼吸:

“太阳最红,毛主席最亲……”

那是一首什么老歌?

“大海航行靠舵手,万物生长靠太阳。

雨露滋润禾苗壮,干革命靠的是毛泽东思想……”

好大的广场啊,好多的红旗啊!

那一张张沧桑的脸,

可以看出,他们是我的同类。

那一串串流淌的泪,

像一股电流,击中了我的心扉。

我也哭着,我也唱着,

好痛快呀,让眼泪肆意滂沱。

红歌,久违的歌!

你怎么这样苍凉,如泣如诉,曲调还是那样优美。

灌入我的耳鼓,却是如此的哀婉悲切。

你刺穿了我的心,你能告诉我什么?

噢,我明白了,

你是对一个时代的凭吊,

你是献给老人家的挽歌!

我哭什么,谁又能够明白,

我哭我曾经拥有的青春,

想起了《想念毛主席》那一首歌。

那是一个什么样的年代呀,

怎能让人如此眷恋!

他是一位怎样的老人呀,

二十七年后依然让人怀念!

至今不是有人诅咒他灭人欲吗?

好吧,那就放纵罢!

至今不是还有人诿罪于他吗?

那就请读一读黄克诚引用的古诗:

“尔曹身与名俱灭,不废江河万古流。”

历史啊,你是无情的裁判,

没有什么权势,

可以摧毁镌刻着“无私无畏”的丰碑。

没有什么利诱,

能够把所有人变成另类。

历史啊,你是最后的证明,而且已经证明了:

沤不腐的是人心,扯不断的是思恋,

砸不烂的是辉煌,灭不掉的是信念。



有感

2004年10月6日晚

山自常青水自流,艰难坎坷何足忧。

老牛不废平生愿,奋蹄搏击壮志酬。



有感

2004年10月7日午

花自凋零水自流,爱恨情仇何时休。

秋风吹我当归去,自有新绿在后头。



2004年末,我离开商丘,回到了故乡,休息了两个多月,又到了原来的私立学校原来的岗位,依然做后勤工作兼管外事活动。



对联

2005年1月29日晨,与孙校同车去南开附中途中,他嘱余为学校撰一春联。沉吟少顷,遂成。

金猴献瑞教坛处处皆祥瑞;

红梅报春校园年年满阳春。



三亚临海	

2005年5月1日

观琼山之巍峨兮,知有多少流放鬼。

念项羽之悲歌兮,直把来者心摧。

心摧,心摧,

抛一掬浊泪,

汇入天涯水。	



离琼过广州

2005年5月9日

万里奔波为底事?两过羊城不能留。

心悬十日难放下,思念新作(1)事未周。

注(1)《晋阳英烈》为处女作,《毛泽东大传》为新作。



旅途三叹

2005年8月14日

一

人在旅途意蹉跎,回车镇前思回车。

失意已成千古恨,前途苦海正风波。

二

甘苦自酿自斟酌,穷途末路又如何。

一着不慎何须悔,只将遗骸填沟壑。

三

久在江湖思素心,处处庙宇处处神。

相鼠有皮人无仪(1),我以我血荐红尘。

注(1)“相鼠有皮,人而无仪!人而无仪,不死何为?相鼠有齿,人而无止!人而无止,不死何俟?相鼠有体,人而无礼!人而无礼,胡不遄死?”这句话出自《诗经》中的《国风·鄘风·相鼠》篇。意思是说老鼠都有皮有齿有体,人却无仪无耻无礼,不去死还干什么呢?此语极具嘲讽,辛辣地批判了在位者缺乏礼义廉耻的丑行。



夜饮

2005年8月21日	

秋风秋雨连旬日,檐滴檐泣叩无眠。

天边雷动如闷鼓,榻前杯倾似鸣蝉。

曾记当年共甘苦,翁媪起舞雏燕喧。

忽忆前帆丝竹好,顿悟孤舟是漏船。



秋夜

2005年8月24日

凄风苦雨天初晴,起看残月晦不明。

腹中一肠直擂鼓,阶前百虫乱发声。

挥洒弱毫墨无序,检点报章字不清。

百窗明灭灯似火,斗室寒凉席如冰。



赴琼途经广西

2005年8月30日	

山似孤丘蔗成林,一处风物一处新。

万里驱驰再蹈海,何时何地是归身。



客机上观云海

2005年11月20日

似山似海脚下踩,似雪似棉半空悬。

碧窗透明一瞥处,山青水秀是乐园。

蝼蚁碌碌为生存,螳雀悄悄凝目观。

南国北国任来去,高空万米我为仙。



黑夜

2005年11月27日子时

你知道吗

黑夜有多么可怕

他一手遮天

吞噬了光明

尽管天上的星星眨着眼睛

月亮也躲在云层后面偷窥

可所有的罪恶

依然肆无忌惮地滋生

作奸犯科

杀人越货

尔虞我诈

贿赂公行

血液凝固了

神经麻木了

斗室里失去了温暖

唯有恐惧在徘徊

郁闷在蔓延

孤独在膨胀

死神在爬行

我不住地添加衣服

心里还是感觉冰冷

日出的时候多么美好啊

我诅咒黑夜

企盼光明



无题

2006年1月12日晨4时

可怜白发翁,心无一日宁。

斗室徒四壁,顾影悲自生。



诗三首

2006年1月16日2时

一

男儿枉佩三尺剑,心有魔障不能斩。

情天恨海谁无怨,毛公三叹亦枉然(1)。

注(1)请参看《毛泽东大传》毛主席在1937年挽留贺子珍、在庐山会见贺子珍两部分。

二

资质鲁钝性冥顽,枉我平生一寸丹(1)。

忍看燕雀欺大鸟,来世再把大海填(2)。

注(1)明于谦:《立春日感怀》“一寸丹心图报国,两行清泪为思亲。”(2)“精卫填海”出自《山海经·北山经》,这个故事讲的是精卫衔来木石,决心填平大海。比喻仇恨极深,立志报复。也有人比喻意志坚决,不畏艰难。

三

小子万言骂先生,先生一怒要撕梦。

笑看鲁迅著《奔月》,生儿莫如高长虹。

注(1)鲁迅有个忠实粉丝,是《莽原》周刊的编辑高长虹。高长虹在困难的时候,鲁迅伸出援手,多次帮助这位后进青年。他们像师生,像父子。有一次,高长虹和韦素园等人因刊发稿子,有了矛盾。高长虹想让鲁迅给评个理,可鲁迅不在现场,没法表态。高长虹怀恨在心,在1926年10月28日忽发万字长文,公开辱骂鲁迅,并声称他对鲁迅感到“瘟臭”,甚至想到就“呕吐”。鲁迅一看就懵了:这小子为什么骂我呀?韦素园写信告诉鲁迅,高长虹一直暗恋许广平,有他的“月亮诗”为证:“我在天涯行走,月儿向我点首,我是白日的儿子,月儿呵,请你住口……月儿我交给他了,我交给夜去消受。夜是阴冷黑暗,月儿逃出在白天,只剩着今日的形骸,失却了当年的风光。”韦素园解释,诗中的夜是指鲁迅,太阳是高长虹,月亮是许广平。高长虹是暗恋自己的师娘许广平,并借此迁怒不知情的鲁迅。高长虹还在另一篇文章中写道:“我对于鲁迅先生曾献过最大的让步,不只是思想上,而且是生活上。”鲁迅非常愤怒,写信告诉韦素园:“要细心研究他的梦,或者简直动手撕碎它,让他痛哭流涕!”这才有了这短篇小说《奔月》。细看《奔月》,你会发现那个暗算后羿的逢蒙就是高长虹。小说中,后羿教训施放暗箭的逢蒙说:“你真是白来了一百多回。”说的就是当年的高长虹,为了请教鲁迅,高长虹经常去鲁家,一周三四次,三个多月算起来,正好一百多次。逢蒙在和后羿打斗时,还诅咒后羿“打了丧钟”。在《奔月》中,嫦娥对后羿说过:“若以老人自居,是思想的堕落”。这也是高长虹攻击鲁迅的话。由此看来,《奔月》中背叛后羿的嫦娥,也是暗喻高长虹。后来有人传言,高长虹回了东北,疯了。也有人说他死了,究竟如何,不知所终。

以上抄录,如能助你对鲁迅对《奔月》加深理解,也算不枉我一番心意了。



从豫东到宛东

2006年4月2日

一毯如碧,嵌黄花,锦绣千里。

靠良种化肥农药,还有合理密植(1),又是一个丰收季。

堪慰先贤,当年宏愿不虚。

惜年年丰收,红纱炭值(2)。

更那堪硕鼠多多,稗草萋萋;打工为奴,富豪林立。

人道是,卫星上天,腥臊遍地。

注(1)毛主席在1958年提出“农业八字宪法”,即:土(改良土壤)、肥(合理施肥)、水(兴修水利)、种(培育和推广良种)、密(合理密植)、保(植物保护、防治病虫害)、管(田间管理)、工(工具改革)。(2)唐白居易《卖炭翁》:“卖炭得钱何所营?身上衣裳口中食。”“一车炭,千余斤,宫使驱将惜不得。半匹红纱一丈绫,系向牛头充炭值。”。



宛城漫步

2006年5月4日

多年蹉跎不识家,鬓毛已衰无闲暇。

老夫乘勇脚力健,一日看遍宛城花。

十里白河加岸柳,半间禅林忆名刹。

场馆巍峨金银砌,府衙狰狞好自夸。



五十八周岁自嘲

2007年7月14日夜

少小离家走天涯,卅年肝胆寄云霞。

斗室摇曳半烛泪,老骨零落一孤家。



利比亚战争

2011年3月19日,法国率先空袭利比亚,美国海军于深夜通过其部署在地中海上多艘军舰,向利比亚北部防空系统发动了导弹攻击,英国皇家空军派出多架战机参与空袭。3月20日晚,多国联军向利比亚发动第二波空袭,参加国增至7个。美军出动了王牌B2隐形轰炸机。3月21日,美英向利比亚发射12枚“战斧”巡航导弹,战机出动约七八十次。3月24日,美国对卡扎菲提出三项要求:停火、撤退、下台。4月1日,北约已有205架战机参与在利比亚的军事行动。此后,利比亚反政府武装在美英法等支持下,继续向政府军发动进攻,在10月20日擒获并残暴地杀死了卡扎菲。利比亚战争,是继20世纪90年代的科索沃战争(空袭南联盟)、本世纪的阿富汗战争和伊拉克战争后,西方国家军事联盟第四次对主权国家发动大规模军事打击。

卜算子·悼卡扎菲

2011年10月21日

反帝反封建,首倡结非盟(1)。

四十二年英雄结,一朝梦成空。



强虏大轰炸,老友成帮凶(2)。

孤旅抗暴真豪杰,洒酒祭鬼雄(3)。

注:(1)1999年,在卡扎菲倡议下,非洲第四届特别首脑会议通过《锡尔特宣言》,2001年非盟成立。(2)2011年3月17日,联合国总部投票通过第1973号决议,制裁利比亚卡扎菲政权。在投票中,俄罗斯和中国(胡和谐在位)投了弃权票,表示不反对美国提出的加大对利比亚卡扎菲政权的制裁力度。6月9日,利比亚反对派政府到北京商讨建立合作和外交事务。9月,中国外交部承认利比亚过渡政府为利比亚的合法政府。(3)昨日晨,卡扎菲被掳,反对派武装士兵用棍子捅他屁股,往他身上撒尿。卡扎菲高声呼号,奋力反抗,直至被活活打死。他的遗体没有按照伊斯兰惯例马上下葬,而是和接班人穆塔西姆的尸体一起被放在一个冻肉库里供市民参观。一代枭雄卡扎菲就是以这样屈辱的方式终结了自己的神话。



利比亚战争

2011年10月22日

英雄崛起,反封建,更敢举反帝旗。

一心要联苏联华,信仰社会主义。

解放妇女,发展经济,富甲一方里。

首倡非盟,列强侧目觊觎。



世界格局逆转,苏东剧变,红旗落地。

接受招安,尖端武器放弃(1),美英法战端重启。

北约狂轰滥炸,一片废墟中,壮士伤毙。

公理何存?朋友,你在哪里?

注(1)利比亚被美国列为支持恐怖主义国家多年。为了掌握更多对抗筹码,卡扎菲一直致力于研制核武器。2003年,卡扎菲突然单方面宣布放弃核武器计划。他一方面是出于对美帝的恐惧,另一方面也是出于对美国为首的西方国家的信任——他以为,只要真心投向西方,就安全了,这才下定了决心。卡扎菲在宣布弃核之后,还积极邀请国际核查员对弃核过程进行监督,并协助销毁利比亚的化学武器,销毁远程导弹。



与网友酬唱

笑对人生:

“忠心赤胆有心人,鸿篇巨著颂伟人,

内忧外患转为安,主席思想是灵魂!

2014年3月21日”

东方直心:

“阳光雨露育丹心,大风大浪练铁身。

一介翻身农家子,不揣冒昧图报恩。

2014年3月21日”

笑对人生:

“志同道合观点同,崇拜捍卫毛泽东。

君之巨著感染人,难能可贵精气神!

2014年4月22日”

东方直心:

“愚公亿万读伟人,道义共肩一条心。

莫道王屋压迫重,九州未来日日新。

2014年4月22日”



偶得

2015年10月21日

我本寻常人,近来烦扰频顾。

残躯何不自珍惜,孤影河边漫步。

桥头旁,笙歌处,

有雅韵清音相慰,忘却不快无数。



剥王维《相思》

2016年4月21日

红豆生南都,春来发几枝?

梦中常采撷,此物最相思。



一二·二六人民节纪实

2018年12月26日

百草凄凄枝叶衰,红旗猎猎地火飞。

金桂迎雪仰天笑,腊梅初绽竞芳菲。

注:是月十八日,256本书被有关方面以“敏感出版物”为借口“暂扣”拉走。未几,小雪。是日,余冒雪在淯水岸边湿地公园徜徉,用手机为四季桂拍照,吟诵“七言”抒怀。



《二0二0人民战争之封城战疫》开篇词

满江红·题自拍照

2020年2月

曾经SARS,今又见新冠天造(1)。

捂口鼻,人无面目,闭户为牢。

病毒重锁清明道,官讯灾难似惊涛。

知封城断路遍神州,心如捣。



遮望眼,不忍瞧。

敲键盘,直心剖。

盼人民战争,公心大振。

抗疫热土成净土,斯民十亿马列毛!

上太空再奏东方红,妖氛销。

注(1)新冠肺炎病毒疫情出现后,人为制造病毒的“阴谋论”甚嚣尘上。后来,“阴谋论”由民间发展到了大国之间,且持续了三年之久。



题梦郎画像

2024年7月下旬

成人童话(1)好,港岛影像(2)真。

七旬娃娃脸(3),东方不败(4)身。

注释(1)钱学森称金庸的新武侠小说为成人童话。(2)香港影视人物形象。(3)《天龙八部》中的童姥。(4)《笑傲江湖》中的武林高手,为修炼武功秘籍《葵花宝典》,自宫而蜕变为美妇人,且拥有了男宠。



备忘录一



备忘录二

2024年8月22日

坐观风云四十年,键盘声里度老残。

网监已成猫和狗,一窝耗子反了天。

报章撕来做手纸,电视转换断毒源。

最爱视频百家论,喜讯飞越万仞山。



答友人

2024年9月16日(中秋节前夕),微信得学友节日祝贺,且附宋苏轼诗一首:

暮云收尽溢清寒,银汉无声转玉盘。

此生此夜不长好,明月明年何处看。

遂答之曰:

一年一度又中秋,夜黑风高底是头?

何时盼得星月朗,把酒欢歌一醉休。


\end{document}